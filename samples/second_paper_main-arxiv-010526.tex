\documentclass[11pt]{article}
\usepackage{hyperref}
%
%
\usepackage{amsmath,amssymb,amsthm}
\usepackage[margin=1in]{geometry}
%
\usepackage{graphicx}
\usepackage{color}
\usepackage{bbm}
\usepackage{booktabs}
\usepackage{breakcites}
\usepackage{bm}
\usepackage[numbers,sort&compress]{natbib}
\usepackage{thm-restate}
\usepackage{color-edits}
%
%
%
\usepackage{thmtools}
\usepackage{mathtools}
%
\usepackage[inline, shortlabels]{enumitem}
\usepackage{enumitem}
\usepackage{dsfont}
\usepackage{xspace}
\usepackage{hyperref}
%
\usepackage[algo2e,linesnumbered,ruled,vlined,algosection]{algorithm2e}
%
\usepackage{algorithmic}
\usepackage{subcaption}
\usepackage{array}
\usepackage[dvipsnames]{xcolor}
\usepackage{cleveref}
\usepackage{makecell}
%

\usepackage{colortbl}

\theoremstyle{plain}
\declaretheorem[name=Theorem,numberwithin=section]{theorem}
%
\newtheorem{lemma}[theorem]{Lemma}
\newtheorem{proposition}[theorem]{Proposition}
\newtheorem{claim}[theorem]{Claim}
\newtheorem{cor}[theorem]{Corollary}
\newtheorem{corollary}[theorem]{Corollary}
\newtheorem{assumption}[theorem]{Assumption}
\newtheorem{prop}[theorem]{Proposition}
\newtheorem{fact}[theorem]{Fact}
\newtheorem{question}[theorem]{Question}
\newtheorem{goal}[theorem]{Goal}

\theoremstyle{definition}
\newtheorem{definition}[theorem]{Definition}
\newtheorem{example}[theorem]{Example}

\theoremstyle{remark}
\newtheorem{remark}[theorem]{Remark}
\newtheorem{observation}{Observation}

%

\renewcommand{\vec}[1]{{\bm{#1}}}

\newcommand{\bvec}[1]{{\bar{\vec{#1}}}}
\newcommand{\pvec}[1]{\vec{#1}'}
\newcommand{\ppvec}[1]{\vec{#1}''}
\newcommand{\tvec}[1]{{\tilde{\vec{#1}}}}

\newcommand{\paren}[1]{\left(#1 \right )}
\newcommand{\Paren}[1]{\left(#1 \right )}

\newcommand{\brac}[1]{[#1 ]}
\newcommand{\Brac}[1]{\left[#1\right]}

\newcommand{\Set}[1]{\left\{#1\right\}}
\newcommand{\set}[1]{\left\{#1\right\}}


\newcommand{\abs}[1]{\left\lvert#1\right\rvert}
\newcommand{\Abs}[1]{\left\lvert#1\right\rvert}

\newcommand{\ceil}[1]{\lceil #1 \rceil}
\newcommand{\floor}[1]{\lfloor #1 \rfloor}

%
\DeclarePairedDelimiter{\inceil}{\lceil}{\rceil}
\DeclarePairedDelimiter{\infloor}{\lfloor}{\rfloor}
%

\newcommand{\norm}[1]{\left\lVert#1\right\rVert}
\newcommand{\Norm}[1]{\left\lVert#1\right\rVert}
\newcommand{\fnorm}[1]{\norm{#1}_\mathrm{F}}
\newcommand{\opnorm}[1]{\norm{#1}_\mathrm{op}}
\newcommand{\spnorm}[1]{\norm{#1}_\mathrm{2}}
\newcommand{\nuclear}[1]{\norm{#1}_{*}}


\newcommand{\nnz}[1]{\mathrm{nnz}(#1)}


%
\newcommand{\defeq}{\coloneqq}
\newcommand{\seteq}{\mathrel{\mathop:}=}
\newcommand{\iseq}{\stackrel{\textup{?}}{=}}
\newcommand{\isgeq}{\stackrel{\textup{?}}{\geq}}
\newcommand{\isleq}{\stackrel{\textup{?}}{\leq}}

\newcommand{\vbig}{\vphantom{\bigoplus}}


\newcommand{\inprod}[1]{\left\langle #1\right\rangle}


\newcommand{\Tr}[1]{\mathrm {Tr}\paren{#1}}



\newcommand{\snorm}[1]{\norm{#1}^2}
\newcommand{\dist}[1]{{\sf dist }\paren{#1}}


\newcommand{\normt}[1]{\norm{#1}_{\scriptstyle 2}}
\newcommand{\snormt}[1]{\norm{#1}^2_2}


\newcommand{\projsymb}{{\sf Proj}}
\newcommand{\proj}[2]{\projsymb_{#1}\paren{#2}}
\newcommand{\projperp}[2]{\projsymb_{#1}^{\perp}\paren{#2}}
\newcommand{\projpar}[2]{\projsymb_{#1}^{\parallel}\paren{#2}}


\newcommand{\normo}[1]{\norm{#1}_{\scriptstyle 1}}
\newcommand{\normi}[1]{\norm{#1}_{\scriptstyle \infty}}
\newcommand{\normb}[1]{\norm{#1}_{\scriptstyle \square}}
\newcommand{\normInline}[1]{\lVert#1\rVert}

\newcommand{\Z}{{\mathbb Z}}
\newcommand{\N}{{\mathbb Z}_{\geq 0}}
\newcommand{\natnums}{\mathbb{N}}
\newcommand{\R}{\mathbb R}
\newcommand{\Q}{\mathbb Q}
\newcommand{\Rnn}{\R_+}

\newcommand{\subjectto}{\text{subject to}}
\newcommand{\suchthat}{~:~}


\newcommand{\Esymb}{\mathbb{E}}
\newcommand{\Psymb}{\mathbb{P}}
\newcommand{\Vsymb}{\mathbb{V}}
\newcommand{\Isymb}{\mathbb{I}}
\DeclareMathOperator*{\E}{\Esymb}
\DeclareMathOperator*{\Var}{{ Var}}
\DeclareMathOperator*{\ProbOp}{\Psymb}
\newcommand{\var}[1]{\Var \paren{#1}}


\newcommand{\given}{\mathrel{}\middle|\mathrel{}}
\newcommand{\Given}{\given}


\newcommand{\prob}[1]{\ProbOp\Set{#1}}
\newcommand{\ber}{\mathsf{Ber}}
\newcommand{\Prob}{\mathbb{P}}
\newcommand{\probSub}[2]{\mathbb{P}_{#2}\Set{#1}}
\newcommand{\probSubInline}[2]{\mathbb{P}_{#2}\{#1\}}

\newcommand{\indicator}[1]{\mathds{1}{\set{#1}}}
\newcommand{\1}[1]{\mathds{1}_{\set{#1}}}


\newcommand{\ex}[1]{\E\brac{#1}}
\newcommand{\Ex}[1]{\E\Brac{#1}}
\newcommand{\exSub}[2]{\E_{#2}\brac{#1}}


\renewcommand{\Pr}[1]{\ProbOp\Brac{#1}}
\newcommand{\pr}[2]{\ProbOp_{#1}\Set{#2}}


\newcommand{\ind}[2]{\Isymb_{#1}\brac{#2}}
\newcommand{\Ind}[1]{\Isymb\Brac{#1}}

\newcommand{\varex}[1]{\E\paren{#1}}
\newcommand{\varEx}[1]{\E\Paren{#1}}
\newcommand{\eset}{\emptyset}
\newcommand{\e}{\epsilon}
\newcommand{\ebf}{\mathbf{e}}
\newcommand{\Abar}{\bar{A}}
\newcommand{\Dbar}{\bar{D}}
\newcommand{\Ebar}{\bar{E}}
\newcommand{\zero}{\mathbf{0}}


\newcommand{\bern}[1]{\ensuremath{\operatorname{Bern}\paren{ #1 } }}
\newcommand{\unif}{\ensuremath{\operatorname{Unif}}}



\newcommand{\super}[2]{#1^{\paren{#2}}}


\newcommand{\bits}{\{0,1\}}

\let\e\varepsilon


\newcommand{\cA}{\mathcal A}
\newcommand{\cB}{\mathcal B}
\newcommand{\cC}{\mathcal C}
\newcommand{\cD}{\mathcal D}
\newcommand{\cE}{\mathcal E}
\newcommand{\cF}{\mathcal F}
\newcommand{\cG}{\mathcal G}
\newcommand{\cH}{\mathcal H}
\newcommand{\cI}{\mathcal I}
\newcommand{\cJ}{\mathcal J}
\newcommand{\cK}{\mathcal K}
\newcommand{\cL}{\mathcal L}
\newcommand{\cM}{\mathcal M}
\newcommand{\cN}{\mathcal N}
\newcommand{\cO}{\mathcal O}
\newcommand{\cP}{\mathcal P}
\newcommand{\cQ}{\mathcal Q}
\newcommand{\cR}{\mathcal R}
\newcommand{\cS}{\mathcal S}
\newcommand{\cT}{\mathcal T}
\newcommand{\cU}{\mathcal U}
\newcommand{\cV}{\mathcal V}
\newcommand{\cW}{\mathcal W}
\newcommand{\cX}{\mathcal X}
\newcommand{\cY}{\mathcal Y}
\newcommand{\cZ}{\mathcal Z}
\newcommand{\hatN}{\hat{N}}

\newcommand{\bbB}{\mathbb B}
\newcommand{\bbS}{\mathbb S}
\newcommand{\bbR}{\mathbb R}
\newcommand{\bbZ}{\mathbb Z}
\newcommand{\bbI}{\mathbb I}
\newcommand{\bbQ}{\mathbb Q}
\newcommand{\bbP}{\mathbb P}
\newcommand{\bbE}{\mathbb E}
\newcommand{\bbC}{\mathbb C}

\newcommand{\sfE}{\mathsf E}


\newcommand{\Erdos}{Erd\H{o}s\xspace}
\newcommand{\Renyi}{R\'enyi\xspace}
\newcommand{\Lovasz}{Lov\'asz\xspace}
\newcommand{\cdeg}{\mathrm{cdeg}}
\newcommand{\bigO}{O}
\newcommand{\bigo}[1]{\bigO\!\left(#1\right)}
\newcommand{\tbigO}{\tilde{\mathcal{O}}}
\newcommand{\tbigo}[1]{\tbigO\!\left(#1\right)}
\newcommand{\tensor}{\otimes}
\newcommand{\eigvec}{{\sf v}}

\DeclareMathOperator*{\argmin}{argmin} 
\DeclareMathOperator*{\argmax}{argmax} 
\newcommand{\poly}{{\sf poly}}
\newcommand{\polylog}{{\sf polylog}}
\newcommand{\supp}{{\sf supp}}

\newcommand{\rank}{{\sf rank}}
\newcommand{\tk}{t_{1/k}} 
\newcommand{\rd}{{\sf d}}

\newcommand{\specialcell}[2][c]{%
  \begin{tabular}[#1]{@{}l@{}}#2\end{tabular}}


\allowdisplaybreaks
\definecolor{violet}{RGB}{148, 0, 211}
\hypersetup{
    colorlinks=true,
    linkcolor=blue!70!black, %
    citecolor=violet, %
    filecolor=blue!70!black, %
    urlcolor=magenta
}

\newcommand{\bv}{\bm{v}}
\newcommand{\bz}{\bm{z}}
\newcommand{\by}{\bm{y}}
\newcommand{\bxi}{\bm{\xi}}
\newcommand{\bx}{\bm{x}}
\newcommand{\bg}{\bm{g}}
\newcommand{\br}{\bm{r}}
\newcommand{\bP}{\bm{P}}
\newcommand{\bu}{\bm{u}}
\newcommand{\bsigma}{\bm{\sigma}}
\newcommand{\bmp}{\bm{p}}
\newcommand{\Atot}{\cA_{\mathrm{tot}}}%
\newcommand{\Aset}{\cA}
\newcommand{\vones}{\mathbf{1}}
\newcommand{\vzero}{\mathbf{0}}
\newcommand{\tmix}{t_{\mathrm{mix}}}

\newcommand{\reward}{\bm{r}}
\newcommand{\transProb}{\bm{p}}
\newcommand{\otilde}{\tilde{O}}
\newcommand{\Otilde}{\otilde}
\newcommand{\transMat}{\mathbf{P}}

\newcommand{\vals}{\bv}
\newcommand{\optVals}{\bv^*}
\newcommand{\diffVec}{\bm{\Delta}}

%

\newcommand{\code}[1]{\textnormal{\texttt{#1}}}
\SetKwInput{KwInput}{Input}
\SetKwInput{KwOutput}{Output}
\SetKwInput{KwParameter}{Parameter}
%
\SetKwProg{Function}{function}{}{}
\SetKwIF{IfNot}{ElseIfNot}{}{if not}{then}{else if not}{}{} %
%
\newcommand\mycommfont[1]{\footnotesize\textcolor{gray}{#1}}
\SetCommentSty{mycommfont}
\SetKwFor{Repeat}{repeat}{}{}
\SetKwRepeat{Do}{do}{while} %
%
%

\usepackage{algorithmic}

\newcommand{\VarOf}[1]{\textnormal{Var}\Brac{#1}}

\usepackage{booktabs} %
\usepackage{array} %

\newcommand{\Igamma}{(\bm{I} - \gamma \bP^{\star})^{-1}}
\newcommand{\normsigma}{\norm{(\bm{I} - \gamma \bP^{\star})^{-1}\sqrt{\bsigma_{\bv^\star}}}_\infty}
\newcommand{\tv}[2]{d_{TV}\paren{#1, #2}}
\newcommand{\tvInline}[2]{d_{TV}({#1, #2})}
\newcommand{\tvTwo}[2]{d_{TV}({#1, #2})}
\newcommand{\bI}{\bm{I}}

\newcommand{\nOuter}{n_{\mathrm{outer}}}
\newcommand{\xParam}{x^{\mathrm{param}}}
\newcommand{\nBatchInner}{n_{\mathrm{batch}}}
\newcommand{\nSampleInner}{n_{\mathrm{sample}}}
\newcommand{\UniformDist}{\mathsf{Uniform}}
\newcommand{\ellOneEllOne}{\ell_1\text{-}\ell_1}
\newcommand{\ellTwoEllOne}{\ell_2\text{-}\ell_1}
%
%
\newcommand{\normATwoToInf}[1]{\Vert #1 \Vert_{2 \rightarrow \infty}}
\newcommand{\nSampleInnerDependent}{n_{\mathrm{sampleAdaptive}}}
\newcommand{\nSampleInnerOblivious}{n_{\mathrm{sampleOblivious}}}
\newcommand{\sr}{\mathrm{sr}}
\newcommand{\bSigma}{\bm{\Sigma}}
\newcommand{\gap}{\mathrm{gap}}


\newcommand{\APP}{\code{APP}}
\newcommand{\Solve}{\code{Solve}}
\newcommand{\SVRG}{\code{SVRG}}
\newcommand{\VRMD}{\code{VRMD}}
%

\newcommand{\Dentry}{\cD_{\mathrm{entry}}}
\newcommand{\Drow}{\cD_{\mathrm{row}}}
\newcommand{\Dcol}{\cD_{\mathrm{col}}}

\newcommand{\minimize}{\mathrm{minimize}}
\newcommand{\maximize}{\mathrm{maximize}}
\newcommand{\tr}{\mathrm{tr}}


\newcommand{\prox}{\mathrm{prox}}
\newcommand{\proxStep}{\mathsf{SUGMStep}}
%
%
%
\newcommand{\modelUpdateStep}{\code{flag}}
\newcommand{\true}{\code{True}}
\newcommand{\false}{\code{False}}
\newcommand{\guilty}{\code{guilty}}
\newcommand{\validation}{\code{status}}
\newcommand{\smooth}{\code{smooth}}
\newcommand{\verdict}{\code{verdict}}
\newcommand{\Step}{\textsc{Step}}
\newcommand{\SUG}{\textsc{SUPGSolver}}
\newcommand{\Subsolver}{\mathsf{SubProblemSolver}}
\newcommand{\flag}{\code{flag}}
\newcommand{\none}{\code{None}}
%
%
\newcommand{\B}{\mathbb{B}}
\newcommand{\KL}{\mathrm{KL}}

%
%
%
%


%
\newcommand{\xset}{\cX}
\newcommand{\yset}{\cY}
\newcommand{\zset}{\cZ}
\newcommand{\simplex}{\Delta}
\usepackage{etoolbox}
\usepackage{xparse}
\usepackage{mathtools}
\DeclarePairedDelimiterXPP{\inangle}[1]{}{\langle}{\rangle}{}{#1}
\DeclarePairedDelimiterXPP{\inbraces}[1]{}{\{}{\}}{}{#1}
\DeclarePairedDelimiterXPP{\inparen}[1]{}{(}{)}{}{#1} %
%
\DeclarePairedDelimiterXPP{\insquare}[1]{}{[}{]}{}{#1}
\DeclarePairedDelimiter{\innorm}{\|}{\|}
\DeclarePairedDelimiter{\inabs}{\lvert}{\rvert}

%
%
%
%
%
%
%
%
%


%

%

%
%
\NewDocumentCommand\breg{s m m }{ %
  \IfBooleanTF{#1} %
    { V^{r}_{#2} \left( #3 \right) }
    { V^{r}_{#2} ( #3 ) }
}
%

%
%
%
\NewDocumentCommand\bregwr{s O{} m m }{ %
  \IfBooleanTF{#1} %
    { V^{#2}_{#3} \left( #4 \right) }
    { V^{#2}_{#3} ( #4 ) }
}
%

%

\newcommand{\rx}{r_\mathsf{x}}
\newcommand{\ry}{r_\mathsf{y}}

\NewDocumentCommand\xbreg{s m m }{
  \IfBooleanTF{#1} %
    { \bregwr*[\rx]{#2}{#3} }
    { \bregwr[\rx]{#2}{#3} }
}
\NewDocumentCommand\ybreg{s m m }{
  \IfBooleanTF{#1} %
    { \bregwr*[\ry]{#2}{#3} }
    { \bregwr[\ry]{#2}{#3} }
}

\newcommand{\gm}{\nabla_{\pm}}
\newcommand{\xnabla}{\nabla_{\mathsf{x}}}
\newcommand{\ynabla}{\nabla_{\mathsf{y}}}

\newcommand{\x}{_\mathsf{x}}
\newcommand{\y}{_\mathsf{y}}

\newcommand{\xsub}{\mathsf{x}}
\newcommand{\ysub}{\mathsf{y}}

%
%


\newcommand{\RPPS}{\textsc{RPPS}}
\newcommand{\zopt}{z^\star}
\newcommand{\zhat}{\hat{z}}
\newcommand{\grad}{\nabla}
\newcommand{\xz}{x^0}
\newcommand{\yz}{y^0}
\newcommand{\zz}{z^0}
\newcommand{\one}{^1}
\newcommand{\two}{^2}
%
%
\newcommand{\frange}{B_{\mathrm{range}}} %
%
\newcommand{\allpoly}{\poly(\cdots)}
\DeclarePairedDelimiterXPP{\dualnorm}[1]{}{\|}{\|}{_{*}}{#1}

\NewDocumentCommand\PS{m m}{
\textsc{PS}({#1}, {#2})
}
\newcommand{\AS}{\textsc{AS}}
\newcommand{\DB}{\textsc{DB}}
\newcommand{\DMP}{\textsc{DMP}}
%
\newcommand{\CWF}{\textsc{CWF}}
\newcommand{\GWF}{\textsc{GWF}}
%
\newcommand{\ztilde}{\tilde{z}}
%

\newcommand{\wbar}{\bar{w}}

\newcommand{\overeq}[1]{\overset{#1}{=}}
\newcommand{\overle}[1]{\overset{#1}{\le}}
\newcommand{\overge}[1]{\overset{#1}{\ge}}


\newcommand{\thirdind}{J_{\epsilon^{1/3}}}
\newcommand{\gethirdind}{J_{> \epsilon^{1/3}}}

%
\newcommand{\treed}{\mathcal{T}}
\newcommand{\pathd}{\mathcal{P}}

\newcommand{\Holder}{H\"{o}lder}


%
%
%
%

%
\newcommand{\proxStepFullZ}[6]{\prox^{#3, #4}_{#1 , #2}(#5; #6)}
\newcommand{\proxStepFull}[5]{\prox^{#3, #4}_{#1 , #2}(#5)}
\newcommand{\proxStepSimple}[3]{\prox^{#2}_{#1}(#3)}
\newcommand{\proxStepSimpleZ}[4]{\prox^{#2}_{#1}(#3; #4)}

%
\newcommand{\sfx}{\mathsf{x}}
\newcommand{\sfy}{\mathsf{y}}
\newcommand{\rsubx}{r_\sfx}
\newcommand{\rsuby}{r_\sfy}
\newcommand{\diag}{\mathrm{diag}}

\newcommand{\locnorm}[2]{\normInline{#1}_{#2}}

\newcommand{\evalStepballball}{\mathsf{FrobeniusJudge}}
\newcommand{\evalStepballballNuclear}{\mathsf{NuclearJudge}}
\newcommand{\judge}{\textsc{Judge}}
\newcommand{\jury}{\mathsf{Jury}}
\newcommand{\normalize}{\mathrm{unit}}
\newcommand{\frobJudge}{\mathsf{FrobeniusJudge}}
\newcommand{\xtrunc}{\xset_\nu}
\newcommand{\ytrunc}{\yset_\nu}
\newcommand{\ztrunc}{\zset_\nu}

\newcommand{\gaptrunc}{\gap_{\nu}}

\newcommand{\fprimal}{f_{\mathrm{pr}}}
\newcommand{\fdual}{f_{\mathrm{du}}}
\newcommand{\localize}[2]{(#1)_{#2}}
\newcommand{\ground}[2]{{(#1)}_{#2}}
\newcommand{\unground}[2]{{(#1)}_{#2, *}}
%
\newcommand{\SUGSMStep}{\mathsf{SUG}\text{-}\mathsf{SM}\text{-}\mathsf{Step}} %
%
\newcommand{\SUGStronglyMonotoneMirrorProx}{\mathsf{SUG}\text{-}\mathsf{SM}\text{-}\mathsf{MP}}

\newcommand{\trunc}{_{\nu}}
\newcommand{\htrunc}{h_{\nu}}
\newcommand{\gtrunc}{g_{\nu}}
\newcommand{\ball}{\B}
\DeclarePairedDelimiter{\inmaxnorm}{\|}{\|_{\mathrm{max}}}

\newcommand{\gradx}{\grad\x}
\newcommand{\grady}{\grad\y}

\newcommand{\regret}{\mathrm{regret}}

\newcommand{\htilde}{\tilde{h}}

\newcommand{\walpha}{w_\alpha}
\newcommand{\wbeta}{w_\beta}
\newcommand{\halpha}{h_\alpha}
\newcommand{\hbeta}{h_\beta}

\newcommand{\hess}{\nabla^2}
\newcommand{\showdiag}{\mathrm{diag}}

\newcommand{\inKL}[2]{\KL(#2||#1)} %

\newcommand{\bilinear}[1]{f_{#1}}

\newcommand{\utheta}{u_\theta}
\newcommand{\uxi}{u_\xi}

\newcommand{\xtilde}{\tilde{x}}
\newcommand{\ytilde}{\tilde{y}}

\newcommand{\wopt}{w^\star}


%
%
%

%
%
%
%

%
\newcommand{\zground}{z_\mathsf{center}}
\newcommand{\zgroundx}{z_{\mathsf{center} \mathsf{x}}}
\newcommand{\zgroundy}{z_{\mathsf{center} \mathsf{y}}}
\newcommand{\xground}{x_\mathsf{center}}
\newcommand{\zmean}{z_{\mathsf{mean}}}
\newcommand{\zhead}{z_{\mathsf{head}}}
\newcommand{\alphaopt}{\tilde{\alpha}}

\newcommand{\epsprim}{\epsilon'}

\newcommand{\sball}{\cB}

\newcommand{\gammapw}{\gamma_{\mathrm{pw}}}
\newcommand{\gammab}{\gamma_{\mathrm{b}}}
\newcommand{\gammagb}{\gamma_{\mathrm{gb}}}
\newcommand{\gammav}{\gamma_{\mathrm{v}}}


\newcommand{\passModel}{\code{passModel}}

%
%
%
\newcommand{\slower}{s^{\text{lower}}}
\newcommand{\supper}{s^{\text{upper}}}
\newcommand{\tlower}{t^{\text{lower}}}
\newcommand{\tupper}{t^{\text{upper}}}
\newcommand{\vlower}{v^{\text{lower}}}
\newcommand{\vupper}{v^{\text{upper}}}
\newcommand{\bestresponse}{\mathsf{map}}
\newcommand{\checkdiv}{\mathsf{InnerSolver}}
\newcommand{\bsearch}{\mathsf{B}\text{-}\mathsf{Search}}
\newcommand{\getbestresponse}{\mathsf{GetBestResponse}}
\newcommand{\checkcoords}{\mathsf{CheckCoords}}
\newcommand{\coordsInRange}{\code{coordsInRange}}
\newcommand{\tooBig}{\code{tooBig}}
\newcommand{\tooSmall}{\code{tooSmall}}
\newcommand{\justRight}{\code{justRight}}
\newcommand{\oracleSuccess}{\code{success}}
\newcommand{\oracleFailure}{\code{failure}}
\newcommand{\betaDivFlag}{\code{betaDivFlag}}
\newcommand{\alphaDivFlag}{\code{alphaDivFlag}}
%
%

\newcommand{\append}{\mathsf{Append}}
\newcommand{\getpath}{\textsc{GetPath}}
\newcommand{\MDMP}{\textsc{MDMP}}
\newcommand{\MDMPImp}{\textsc{MDMPSearch}}
\newcommand{\CCO}{\textsc{CCO}}
\newcommand{\Wrapper}{\textsc{Wrap}}
\newcommand{\oracle}{\cO_{\textsc{SEARCH}}}

\newcommand{\bregx}[2]{V^{\rx}_{#1}({#2})}
\newcommand{\bregy}[2]{V^{\ry}_{#1}({#2})}
\newcommand{\Validate}{\textsc{ConstrainedSolve}}
\newcommand{\geomean}[1]{{\mathsf{g}({#1})}}
\newcommand{\mean}[1]{{\mathsf{m}({#1})}}
\newcommand{\collapsed}[1]{{\mathsf{q}({#1})}} %


\newcommand{\cZint}{\cZ_{\mathrm{int}}}
%
\newcommand{\zoptbeta}{\zopt_\beta}
\newcommand{\zoptalpha}{\zopt_\alpha}
\newcommand{\zoptalphax}{\zopt_{\alpha \mathsf{x}}}
\newcommand{\zoptalphay}{\zopt_{\alpha \mathsf{y}}}
\newcommand{\zoptlocal}{\zopt_\ell}
\newcommand{\zoptlocalx}{\zopt_{\ell \mathsf{x}}}
\newcommand{\zoptlocaly}{\zopt_{\ell \mathsf{y}}}
\newcommand{\zoptalphaopt}{\zopt_{\alphaopt}}
\newcommand{\Aq}[1]{A(#1)}
%
\newcommand{\Uq}[2]{\innorm{#1}_{(#2)}} %


\newcommand{\zbar}{\bar{z}}


%
\usepackage{tikz}
\newcommand*\circled[1]{\tikz[baseline=(char.base)]{
    \node[shape=circle,draw,inner sep=1pt] (char) {#1};}}

\newcommand{\oneC}{\circled{1}}
\newcommand{\twoC}{\circled{2}}
\newcommand{\threeC}{\circled{3}}

\newcommand{\Range}{\Gamma}

\newcommand{\entropy}{e}

\newcommand{\project}{\mathrm{proj}}

\newcommand{\smax}{\mathrm{smax}}

\newcommand{\spaceeq}{\qquad}

\newcommand{\uset}{\mathcal{U}}
\newcommand{\idset}{\mathcal{I}}

\newcommand{\biset}{\mathcal{J}}

\DeclarePairedDelimiter{\xnorm}{\|}{\|_{\mathsf{x}}}
\DeclarePairedDelimiter{\ynorm}{\|}{\|_{\mathsf{y}}}

\newcommand{\size}{\mathrm{size}}

\newcommand{\dgfsetup}{\mathcal{S}}
%

%

%

\newcommand{\edset}{\mathcal{E}}


\newcommand{\xhat}{\hat{x}}
\newcommand{\yhat}{\hat{y}}

\newcommand{\Omegatilde}{\tilde{\Omega}}


\newcommand{\cautiousSearch}{\textsc{CautiousBisectionSearch}}
\newcommand{\ysetstable}{\yset_{\mathrm{stable}}}

\newcommand{\Min}{M_{\mathrm{in}}}
\newcommand{\Mout}{M_{\mathrm{out}}}

\newcommand{\bindecompset}{\mathcal{K}}

\newcommand{\Lmax}{L_{\mathrm{max}}}

\newcommand{\prodsetup}{\mathrm{prod}}

\newcommand{\ODMP}{\mathcal{O}_{\DMP}}
\newcommand{\OMDMP}{\mathcal{O}_{\MDMP}}
\newcommand{\OAPPROX}{\mathcal{O}_{\AS}}
\newcommand{\ODB}{\mathcal{O}_{\DB}}

\newcommand{\ktstar}{k_t^\star}

%

\newcommand{\qU}{\mathsf{q}(\uset)} %

\newcommand{\jspecial}{j^\star}

\newenvironment{assumptions}
  {
   \paragraph{Assumptions.}
  }
  {
   \par
   \medskip
  } \usepackage{microtype}
%
%
  %
  %
  \usepackage{nth}
  \usepackage{intcalc}

  \newcommand{\cSTOC}[1]{\nth{\intcalcSub{#1}{1968}}\ Annual\ ACM\ Symposium\ on\ Theory\ of\ Computing\ (STOC)}
  \newcommand{\cFSTTCS}[1]{\nth{\intcalcSub{#1}{1980}}\ International\ Conference\ on\ Foundations\ of\ Software\ Technology\ and\ Theoretical\ Computer\ Science\ (FSTTCS)}
  \newcommand{\cCCC}[1]{\nth{\intcalcSub{#1}{1985}}\ Annual\ IEEE\ Conference\ on\ Computational\ Complexity\ (CCC)}
  \newcommand{\cFOCS}[1]{\nth{\intcalcSub{#1}{1959}}\ Annual\ IEEE\ Symposium\ on\ Foundations\ of\ Computer\ Science\ (FOCS)}
  \newcommand{\cRANDOM}[1]{\nth{\intcalcSub{#1}{1996}}\ International\ Workshop\ on\ Randomization\ and\ Computation\ (RANDOM)}
  \newcommand{\cISSAC}[1]{#1\ International\ Symposium\ on\ Symbolic\ and\ Algebraic\ Computation\ (ISSAC)}
  \newcommand{\cICALP}[1]{\nth{\intcalcSub{#1}{1973}}\ International\ Colloquium\ on\ Automata,\ Languages and\ Programming\ (ICALP)}
  \newcommand{\cCOLT}[1]{\nth{\intcalcSub{#1}{1987}}\ Annual\ Conference\ on\ Computational\ Learning\ Theory\ (COLT)}
  \newcommand{\cCSR}[1]{\nth{\intcalcSub{#1}{2005}}\ International\ Computer\ Science\ Symposium\ in\ Russia\ (CSR)}
  \newcommand{\cMFCS}[1]{\nth{\intcalcSub{#1}{1975}}\ International\ Symposium\ on\ the\ Mathematical\ Foundations\ of\ Computer\ Science\ (MFCS)}
  \newcommand{\cPODS}[1]{\nth{\intcalcSub{#1}{1981}}\ Symposium\ on\ Principles\ of\ Database\ Systems\ (PODS)}
  \newcommand{\cSODA}[1]{\nth{\intcalcSub{#1}{1989}}\ Annual\ ACM-SIAM\ Symposium\ on\ Discrete\ Algorithms\ (SODA)}
  \newcommand{\cNIPS}[1]{Advances\ in\ Neural\ Information\ Processing\ Systems\ \intcalcSub{#1}{1987} (NeurIPS)}
  \newcommand{\cWALCOM}[1]{\nth{\intcalcSub{#1}{2006}}\ International\ Workshop\ on\ Algorithms\ and\ Computation\ (WALCOM)}
  \newcommand{\cSoCG}[1]{\nth{\intcalcSub{#1}{1984}}\ Annual\ Symposium\ on\ Computational\ Geometry\ (SCG)}
  \newcommand{\cKDD}[1]{\nth{\intcalcSub{#1}{1994}}\ ACM\ SIGKDD\ International\ Conference\ on\ Knowledge\ Discovery\ and\ Data\ Mining\ (KDD)}
  \newcommand{\cICML}[1]{\nth{\intcalcSub{#1}{1983}}\ International\ Conference\ on\ Machine\ Learning\ (ICML)}
  \newcommand{\cAISTATS}[1]{\nth{\intcalcSub{#1}{1997}}\ International\ Conference\ on\ Artificial\ Intelligence\ and\ Statistics\ (AISTATS)}
  \newcommand{\cITCS}[1]{\nth{\intcalcSub{#1}{2009}}\ Conference\ on\ Innovations\ in\ Theoretical\ Computer\ Science\ (ITCS)}
  \newcommand{\cPODC}[1]{{#1}\ ACM\ Symposium\ on\ Principles\ of\ Distributed\ Computing\ (PODC)}
  \newcommand{\cAPPROX}[1]{\nth{\intcalcSub{#1}{1997}}\ International\ Workshop\ on\ Approximation\ Algorithms\ for\  Combinatorial\ Optimization\ Problems\ (APPROX)}
  \newcommand{\cSTACS}[1]{\nth{\intcalcSub{#1}{1983}}\ International\ Symposium\ on\ Theoretical\ Aspects\ of\  Computer\ Science\ (STACS)}
  \newcommand{\cMTNS}[1]{\nth{\intcalcSub{#1}{1991}}\ International\ Symposium\ on\ Mathematical\ Theory\ of\  Networks\ and\ Systems\ (MTNS)}
  \newcommand{\cICM}[1]{International\ Congress\ of\ Mathematicians\ {#1} (ICM)}
  \newcommand{\cWWW}[1]{\nth{\intcalcSub{#1}{1991}}\ International\ World\ Wide\ Web\ Conference\ (WWW)}
  \newcommand{\cICLR}[1]{\nth{\intcalcSub{#1}{2012}}\ International\ Conference\ on\ Learning\ Representations\ (ICLR)}
  \newcommand{\cICCV}[1]{\nth{\intcalcSub{#1}{1994}}\ IEEE\ International\ Conference\ on\ Computer\ Vision\ (ICCV)}
  \newcommand{\cICASSP}[1]{#1\ International\ Conference\ on\ Acoustics,\ Speech,\ and\ Signal\ Processing\ (ICASSP)}
  \newcommand{\cUAI}[1]{\nth{\intcalcSub{#1}{1984}}\ Annual\ Conference\ on\ Uncertainty\ in\ Artificial\ Intelligence\ (UAI)}
  \newcommand{\cSOSA}[1]{\nth{\intcalcSub{#1}{2017}}\ Symposium\ on\ Simplicity\ in\ Algorithms\ (SOSA)}
  \newcommand{\cISIT}[1]{#1\ IEEE\ International\ Symposium\ on\ Information\ Theory\ (ISIT)}  
  \newcommand{\cALT}[1]{\nth{\intcalcSub{#1}{1989}}\ International\ Conference\ on\ Algorithmic\ Learning\ Theory\ (ALT)} 
  \newcommand{\cWSDM}[1]{\nth{\intcalcSub{#1}{2007}}\ International\ Conference\ on\ Web\ Search\ and\ Data\ Mining\ (WSDM)} 
  \newcommand{\cICDM}[1]{#1\ IEEE\ International\ Conference\ on\ Data\ Mining\ (ICDM)}  
  \newcommand{\cEC}[1]{\nth{\intcalcSub{#1}{1999}}\ ACM\ Conference\ on\ Economics\ and\ Computation\ (EC)} 
  \newcommand{\cDISC}[1]{\nth{\intcalcSub{#1}{1986}}\ International\ Symposium\ on\ Distributed\ Computing\ (DISC)}
  \newcommand{\cSIGMOD}[1]{{#1} ACM SIGMOD International Conference on Management of Data}
  \newcommand{\cCIKM}[1]{\nth{\intcalcSub{#1}{1991}}\ ACM\ International\ Conference\ on\ Information\ and\ Knowledge\ Management\ (CIKM)}
  \newcommand{\cAAAI}[1]{AAAI\ Conference\ on\ Artificial (AAAI)}  
  \newcommand{\cICDE}[1]{\nth{\intcalcSub{#1}{1984}}\ IEEE\ International\ Conference\ on\ Data\ Engineering\ (ICDE)}  
  \newcommand{\cSPAA}[1]{\nth{\intcalcSub{#1}{1988}}\ ACM\ Symposium\ on\ Parallel\ Algorithms\ and\ Architectures\ (SPAA)}  

  \newcommand{\pSTOC}[1]{Preliminary\ version\ in\ the\ \cSTOC{#1}}
  \newcommand{\pFSTTCS}[1]{Preliminary\ version\ in\ the\ \cFSTTCS{#1}}
  \newcommand{\pCCC}[1]{Preliminary\ version\ in\ the\ \cCCC{#1}}
  \newcommand{\pFOCS}[1]{Preliminary\ version\ in\ the\ \cFOCS{#1}}
  \newcommand{\pRANDOM}[1]{Preliminary\ version\ in\ the\ \cRANDOM{#1}}
  \newcommand{\pISSAC}[1]{Preliminary\ version\ in\ the\ \cISSAC{#1}}
  \newcommand{\pICALP}[1]{Preliminary\ version\ in\ the\ \cICALP{#1}}
  \newcommand{\pCOLT}[1]{Preliminary\ version\ in\ the\ \cCOLT{#1}}
  \newcommand{\pCSR}[1]{Preliminary\ version\ in\ the\ \cCSR{#1}}
  \newcommand{\pMFCS}[1]{Preliminary\ version\ in\ the\ \cMFCS{#1}}
  \newcommand{\pPODS}[1]{Preliminary\ version\ in\ the\ \cPODS{#1}}
  \newcommand{\pSODA}[1]{Preliminary\ version\ in\ the\ \cSODA{#1}}
  \newcommand{\pNIPS}[1]{Preliminary\ version\ in\ \cNIPS{#1}}
  \newcommand{\pWALCOM}[1]{Preliminary\ version\ in\ the\ \cWALCOM{#1}}
  \newcommand{\pSoCG}[1]{Preliminary\ version\ in\ the\ \cSoCG{#1}}
  \newcommand{\pKDD}[1]{Preliminary\ version\ in\ the\ \cKDD{#1}}
  \newcommand{\pICML}[1]{Preliminary\ version\ in\ the\ \cICML{#1}}
  \newcommand{\pAISTATS}[1]{Preliminary\ version\ in\ the\ \cAISTATS{#1}}
  \newcommand{\pITCS}[1]{Preliminary\ version\ in\ the\ \cITCS{#1}}
  \newcommand{\pPODC}[1]{Preliminary\ version\ in\ the\ \cPODC{#1}}
  \newcommand{\pAPPROX}[1]{Preliminary\ version\ in\ the\ \cAPPROX{#1}}
  \newcommand{\pSTACS}[1]{Preliminary\ version\ in\ the\ \cSTACS{#1}}
  \newcommand{\pMTNS}[1]{Preliminary\ version\ in\ the\ \cMTNS{#1}}
  \newcommand{\pICM}[1]{Preliminary\ version\ in\ the\ \cICM{#1}}
  \newcommand{\pWWW}[1]{Preliminary\ version\ in\ the\ \cWWW{#1}}
  \newcommand{\pICLR}[1]{Preliminary\ version\ in\ the\ \cICLR{#1}}
  \newcommand{\pICCV}[1]{Preliminary\ version\ in\ the\ \cICCV{#1}}
  \newcommand{\pICASSP}[1]{Preliminary\ version\ in\ the\ \cICASSP{#1}}
  \newcommand{\pUAI}[1]{Preliminary\ version\ in\ the\ \cUAI{#1}, #1}
  \newcommand{\pSOSA}[1]{Preliminary\ version\ in\ the\ \cSOSA{#1}, #1}
  \newcommand{\pISIT}[1]{Preliminary\ version\ in\ the\ \cISIT{#1}}
  \newcommand{\pALT}[1]{Preliminary\ version\ in\ the\ \cALT{#1}}
  \newcommand{\pWSDM}[1]{Preliminary\ version\ in\ the\ \cWSDM{#1}}
  \newcommand{\pICDM}[1]{Preliminary\ version\ in\ the\ \cICDM{#1}}
  \newcommand{\pEC}[1]{Preliminary\ version\ in\ the\ \cEC{#1}}
  \newcommand{\pDISC}[1]{Preliminary\ version\ in\ the\ \cDISC{#1}, #1}
  \newcommand{\pSIGMOD}[1]{Preliminary\ version\ in\ the\ \cSIGMOD{#1}, #1}
  \newcommand{\pCIKM}[1]{Preliminary\ version\ in\ the\ \cCIKM{#1}, #1}
  \newcommand{\pAAAI}[1]{Preliminary\ version\ in\ the\ \cAAAI{#1}, #1}
  \newcommand{\pICDE}[1]{Preliminary\ version\ in\ the\ \cICDE{#1}, #1}
  \newcommand{\pSPAA}[1]{Preliminary\ version\ in\ the\ \cSPAA{#1}, #1}



  \newcommand{\STOC}[1]{Proceedings\ of\ the\ \cSTOC{#1}}
  \newcommand{\FSTTCS}[1]{Proceedings\ of\ the\ \cFSTTCS{#1}}
  \newcommand{\CCC}[1]{Proceedings\ of\ the\ \cCCC{#1}}
  \newcommand{\FOCS}[1]{Proceedings\ of\ the\ \cFOCS{#1}}
  \newcommand{\RANDOM}[1]{Proceedings\ of\ the\ \cRANDOM{#1}}
  \newcommand{\ISSAC}[1]{Proceedings\ of\ the\ \cISSAC{#1}}
  \newcommand{\ICALP}[1]{Proceedings\ of\ the\ \cICALP{#1}}
  \newcommand{\COLT}[1]{Proceedings\ of\ the\ \cCOLT{#1}}
  \newcommand{\CSR}[1]{Proceedings\ of\ the\ \cCSR{#1}}
  \newcommand{\MFCS}[1]{Proceedings\ of\ the\ \cMFCS{#1}}
  \newcommand{\PODS}[1]{Proceedings\ of\ the\ \cPODS{#1}}
  \newcommand{\SODA}[1]{Proceedings\ of\ the\ \cSODA{#1}}
  \newcommand{\NIPS}[1]{\cNIPS{#1}}
  \newcommand{\WALCOM}[1]{Proceedings\ of\ the\ \cWALCOM{#1}}
  \newcommand{\SoCG}[1]{Proceedings\ of\ the\ \cSoCG{#1}}
  \newcommand{\KDD}[1]{Proceedings\ of\ the\ \cKDD{#1}}
  \newcommand{\ICML}[1]{Proceedings\ of\ the\ \cICML{#1}}
  \newcommand{\AISTATS}[1]{Proceedings\ of\ the\ \cAISTATS{#1}}
  \newcommand{\ITCS}[1]{Proceedings\ of\ the\ \cITCS{#1}}
  \newcommand{\PODC}[1]{Proceedings\ of\ the\ \cPODC{#1}}
  \newcommand{\APPROX}[1]{Proceedings\ of\ the\ \cAPPROX{#1}}
  \newcommand{\STACS}[1]{Proceedings\ of\ the\ \cSTACS{#1}}
  \newcommand{\MTNS}[1]{Proceedings\ of\ the\ \cMTNS{#1}}
  \newcommand{\ICM}[1]{Proceedings\ of\ the\ \cICM{#1}}
  \newcommand{\WWW}[1]{Proceedings\ of\ the\ \cWWW{#1}}
  \newcommand{\ICLR}[1]{Proceedings\ of\ the\ \cICLR{#1}}
  \newcommand{\ICCV}[1]{Proceedings\ of\ the\ \cICCV{#1}}
  \newcommand{\ICASSP}[1]{Proceedings\ of\ the\ \cICASSP{#1}}
  \newcommand{\UAI}[1]{Proceedings\ of\ the\ \cUAI{#1}}
  \newcommand{\SOSA}[1]{Proceedings\ of\ the\ \cSOSA{#1}}
  \newcommand{\ISIT}[1]{Proceedings\ of\ the\ \cISIT{#1}}
  \newcommand{\ALT}[1]{Proceedings\ of\ the\ \cALT{#1}}
  \newcommand{\WSDM}[1]{Proceedings\ of\ the\ \cWSDM{#1}}
  \newcommand{\ICDM}[1]{Proceedings\ of\ the\ \cICDM{#1}}
  \newcommand{\EC}[1]{Proceedings\ of\ the\ \cEC{#1}}
  \newcommand{\DISC}[1]{Proceedings\ of\ the\ \cDISC{#1}}
  \newcommand{\SIGMOD}[1]{Proceedings\ of\ the\ \cSIGMOD{#1}}
  \newcommand{\CIKM}[1]{Proceedings\ of\ the\ \cCIKM{#1}}
  \newcommand{\AAAI}[1]{Proceedings\ of\ the\ \cAAAI{#1}}
  \newcommand{\ICDE}[1]{Proceedings\ of\ the\ \cICDE{#1}}
  \newcommand{\SPAA}[1]{Proceedings\ of\ the\ \cSPAA{#1}}


  \newcommand{\arXiv}[1]{\href{http://arxiv.org/abs/#1}{arXiv:#1}}
  \newcommand{\farXiv}[1]{Full\ version\ at\ \arXiv{#1}}
  \newcommand{\parXiv}[1]{Preliminary\ version\ at\ \arXiv{#1}}
  \newcommand{\CoRR}{Computing\ Research\ Repository\ (CoRR)}

  \newcommand{\cECCC}[2]{\href{http://eccc.hpi-web.de/report/20#1/#2/}{Electronic\ Colloquium\ on\ Computational\ Complexity\ (ECCC),\ Technical\ Report\ TR#1-#2}}
  \newcommand{\ECCC}{Electronic\ Colloquium\ on\ Computational\ Complexity\ (ECCC)}
  \newcommand{\fECCC}[2]{Full\ version\ in\ the\ \cECCC{#1}{#2}}
  \newcommand{\pECCC}[2]{Preliminary\ version\ in\ the\ \cECCC{#1}{#2}} 
%

%
%
%
%
%
%
\title{Solving Matrix Games with Even Fewer Matrix-Vector Products}
%
%
%
%
%
%
%
%
\author{%
    Ishani Karmarkar\thanks{Stanford University, \texttt{\string{ishanik,ocarroll,sidford\string}@stanford.edu}} 
    \and
    Liam O'Carroll\footnotemark[1] 
    \and
    Aaron Sidford\footnotemark[1] 
}



\setcounter{page}{1}

\begin{document}

\pagenumbering{gobble}

\maketitle
%

%
%
%


\begin{abstract}
We study the problem of computing an $\epsilon$-approximate Nash equilibrium of a two-player,  bilinear, zero-sum game with a bounded payoff matrix $A \in \R^{m \times n}$, when the players' strategies are constrained to lie in simple sets. We provide algorithms which solve this problem in $\tilde{O}(\epsilon^{-2/3})$ matrix-vector multiplies (matvecs) in two well-studied cases: $\ell_1$-$\ell_1$ games, where the players' strategies are both in the probability simplex, and $\ell_2$-$\ell_1$ games, where the players' strategies are in the unit Euclidean ball and probability simplex respectively. These results improve upon the previous state-of-the-art complexities of $\tilde{O}(\epsilon^{-8/9})$ for $\ell_1$-$\ell_1$ and of $\tilde{O}(\epsilon^{-7/9})$ for $\ell_2$-$\ell_1$ due to [KOS '25]. In particular, our result for $\ell_2$-$\ell_1$, which corresponds to hard-margin support vector machines (SVMs), matches the lower bound of [KS '25] up to polylogarithmic factors. 
%
%
%
%
%
\end{abstract}


\setcounter{tocdepth}{2}
\tableofcontents \clearpage
%

%

%

%

%
%

\pagenumbering{arabic}

%

%

%

%
%
%
%
%
%
%

\section{Introduction}
\label{sec:intro}



%

In this paper, we consider the fundamental problem of computing \emph{$\epsilon$-solutions of matrix games} \cite{nesterov2005smooth,nem04,carmon2019variance,karmarkar2025solvingzerosumgames,kornowski2024oracle,carmon2024whole,carmon2020coordinate,grigoriadis1995sublinear,clarkson2012sublinear}. In a \emph{matrix game} we must solve the following pair of minimax and maximin optimization problems for a matrix $A \in \R^{m \times n}$ and compact, convex $\xset \subset \R^n$ and $\yset \subset \R^m$:
\begin{equation}
\label{eq:intro-general-matrix-game}
\min_{x \in \xset} \max_{y \in \yset} y^\top A x
~~\text{and}~~
\max_{y \in \yset} \min_{x \in \xset} y^\top A x\,.
\end{equation}
We call $(\hat{x},\hat{y}) \in \xset \times \yset$ an \emph{$\epsilon$-solution} if it is an \emph{$\epsilon$-approximate Nash equilibrium} in the sense that
\[
\gap(\hat{x},\hat{y}) \leq \epsilon ~~\text{where}~~
\gap(\hat{x},\hat{y}) \defeq \max_{y \in \yset} y^\top A \hat{x} - \min_{x \in \xset} \hat{y}^\top A x\,.
\]
$\epsilon$-approximate solutions for matrix games always exist \cite{freund1999adaptive,nemirovskij1983problem,beck2003mirrordescent} and are a standard approximate solution concept. In particular, if $(\xhat, \yhat)$ is an $\epsilon$-solution, then $\xhat$ is an (additive) $\epsilon$-approximate minimizer of $\min_{x \in \xset} \max_{y \in \yset} y^\top A x$,\footnote{In other words, $\max_{y \in \yset} y^\top A \xhat \le \max_{y \in \yset} y^\top A x + \epsilon$ for all $x \in \xset$.} and $\yhat$ is an $\epsilon$-approximate maximizer of $\max_{y \in \yset} \min_{x \in \xset} y^\top A x$.

%

We focus on this problem of solving matrix games in two foundational, well-studied special cases described below. To define these cases (throughout the paper) we let $\Delta^{k} \defeq \{u \in \R^k_{\geq 0} : \innorm{u}_1 = 1 \}$ and $\ball^k \defeq \inbraces{u \in \R^k : \innorm{u}_2 \le 1}$ denote the $k$-dimensional probability simplex and unit Euclidean ball respectively (see Section~\ref{sec:prelims} for additional notation). 



\begin{itemize}
    \item \emph{$\ell_1$-$\ell_1$ games}: In this setting, $\xset = \simplex^n$, $\yset = \simplex^m$, and $|A_{ij}| \leq 1$ for all $i\in [m]$ and $j \in [n]$. Such games encompass solving normal-form zero-sum games \cite{vonNeumann1928} and linear programming \cite{Adler2013,Dantzig1953}.
    
    \item \emph{$\ell_2$-$\ell_1$ games}: In this setting, $\xset = \ball^n$, $\yset = \Delta^m$, and $\norm{A_{i, :}}_2 \leq 1$ for all $i \in [m]$. Such games encompass hard-margin support vector machines (SVMs) \cite{rosenblatt1958perceptron,shwartz2014understandingML,mcculloch1943logical,soheili2012smoothperceptron,yu2014saddlepointsacceleratedperceptron,wang2023accelerated}, namely, computing a maximum-margin linear classifier/separating hyperplane.\footnote{Formally, the $\ellTwoEllOne$ matrix game corresponds to computing a maximum-margin linear classifier through the origin.
    %
      However, this can be extended to capture arbitrary affine hyperplanes via standard reductions.}
    %
    %
    %
    %
\end{itemize}

We study these games under the assumptions that $n$ and $m$ are known, but $A$ is unknown and only accessible via \emph{matvec (queries)}, namely, matrix-vector multiplies of the form $(A^\top y, Ax)$ for an input $(x,y)\in\xset \times \yset$. In the context of zero-sum ($\ellOneEllOne$) games, this corresponds to both players observing the expected payoffs of each individual action, when the other player's strategy is fixed. In the context of SVMs ($\ellTwoEllOne$ games) where the rows of $A$ are data points (multiplied by the corresponding labels), this corresponds to taking linear combinations of data points $(A^\top y)$ and inner products with data points $(Ax)$. 

The fundamental question we study in this paper is:
\begin{center}
    \emph{How many matvecs are necessary to compute $\epsilon$-solutions of $\ellOneEllOne$ and $\ellTwoEllOne$ games?}
\end{center}
Until recently, the state-of-the-art query complexity for these problems was $\otilde(\epsilon^{-1})$ due to seminal, independent works of Nesterov and Nemirovski two decades ago \cite{nem04,nesterov2005smooth}.\footnote{Throughout the paper, we use $\otilde(\cdot)$ and $\Omegatilde(\cdot)$ to hide multiplicative polylogarithmic factors in $n$, $m$, and $\epsilon^{-1}$.} Despite extensive research and the development of alternative algorithms (see \Cref{table:complexities}), this $\Otilde(\epsilon^{-1})$ complexity was only recently improved by \citet{karmarkar2025solvingzerosumgames} to $\Otilde(\epsilon^{-8/9})$ for $\ellOneEllOne$ games and $\Otilde(\epsilon^{-7/9})$ for $\ellTwoEllOne$ games.


%


Excitingly, \cite{karmarkar2025solvingzerosumgames} showed that the $\otilde(\epsilon^{-1})$ query complexity could be improved. However, unfortunately the improved upper bounds of \cite{karmarkar2025solvingzerosumgames} failed to match state-of-the-art lower bounds. These lower bounds are due to recent work of \citet{kornowski2024oracle} which showed that deterministic algorithms require $\Omegatilde(\epsilon^{-2/5})$ queries to solve $\ellOneEllOne$ games and $\Omegatilde(\epsilon^{-2/3})$ queries to solve $\ellTwoEllOne$ games. 

The central goal of this paper is to make progress on closing the gap between upper and lower bounds for this problem. Given the fundamental and well-studied nature of this problem and recent progress of \cite{karmarkar2025solvingzerosumgames} and \cite{kornowski2024oracle}, we defer to these works for a more comprehensive motivation of and introduction to this problem, as well as additional discussion of related work.




%




%

%


%


%

%



\paragraph{Our results.}
The main result of this paper is a general framework for solving matrix games which improves the state-of-the-art deterministic query complexity for both problems to $\Otilde(\epsilon^{-2/3})$.

%

%
\begin{restatable}{theorem}{simplesmain}\label{thm:final-result-l1-l1-aka-zero-sum}
There is a \emph{deterministic} algorithm that computes an $\epsilon$-solution of any $\ell_1$-$\ell_1$ game with $\otilde(\epsilon^{-2/3})$ matvecs to $A$.
\end{restatable}

\begin{restatable}{theorem}{svmmain}
\label{thm:final-result-l2-l1-aka-SVM}
There is a \emph{deterministic} algorithm that computes an $\epsilon$-solution of any $\ell_2$-$\ell_1$ game with $\otilde(\epsilon^{-2/3})$ matvecs to $A$.
\end{restatable}

These results improve upon the prior state-of-the-art query complexities due to \cite{karmarkar2025solvingzerosumgames} by a factor of $\Omegatilde(\epsilon^{-2/9})$ for $\ellOneEllOne$ games and $\Omegatilde(\epsilon^{-1/9})$ for $\ellTwoEllOne$ games. Importantly, Theorem~\ref{thm:final-result-l2-l1-aka-SVM} resolves the complexity of $\ellTwoEllOne$ games up to polylogarathmic factors in light of the aforementioned $\Omegatilde(\epsilon^{-2/3})$ lower bound of \cite{kornowski2024oracle}. For $\ellOneEllOne$ games, there remains a $\Omegatilde(\epsilon^{-4/15})$ gap between the upper bound of Theorem~\ref{thm:final-result-l1-l1-aka-zero-sum} and the $\Omegatilde(\epsilon^{-2/5})$ lower bound of \cite{kornowski2024oracle}. Resolving this gap is the most immediate open problem suggested by this work. We summarize the state of progress for both problems in Table \ref{table:complexities}.\footnote{Independently, Arun Jambulapati has claimed improvements for this problem. We thank Arun for coordinating arXiv postings.}



%
%
%
%
%
%
%
%
%
%
%
%
%
%
%
%
%
%
%
%
%
%
%
%

\begin{table}[h]
   \centering
   %
   %
   \begin{tabular}{@{}p{8cm}p{2.5cm}p{2.5cm}@{}} %
   \toprule
    Method & $\ellOneEllOne$ & $\ellTwoEllOne$  \\ \midrule
   Accelerated gradient descent \cite{nesterov2005smooth}  & $\epsilon^{-1}$ & $\epsilon^{-1}$ \\
   Mirror prox \cite{nem04} & $\epsilon^{-1}$ & $\epsilon^{-1}$ \\
   Dual extrapolation \cite{Nesterov2007dualextrapolation} & $\epsilon^{-1}$ & $\epsilon^{-1}$ \\
   Optimistic mirror descent/FTRL \cite{Rakhlin2013online,steinhardt2014adaptivity,joulani2017modular} & $\epsilon^{-1}$ & $\epsilon^{-1}$ \\
   \citet{karmarkar2025solvingzerosumgames} & $\epsilon^{-8/9}$ & $\epsilon^{-7/9}$ \\
   \rowcolor[HTML]{EFEFEF}
   This paper & $\epsilon^{-2/3}$ & $\epsilon^{-2/3}$  \\ \midrule
%
   Lower bound \cite{kornowski2024oracle} & $\epsilon^{-2/5}$ & $\epsilon^{-2/3}$ \\ \bottomrule
   \end{tabular}
   \caption{\label{table:complexities}
       Asymptotic matvec complexities for $\ellOneEllOne$ and $\ellTwoEllOne$ games. Constants and polylogarithmic factors in $n$, $m$, and $\epsilon^{-1}$ are omitted for brevity. We note that \cite{kornowski2024oracle} improved upon \cite{hadiji2024towards}, which achieved a $\Omega(\log(1 / (n \epsilon)))$ lower bound for $\ellOneEllOne$ games when $m = n$ for sufficiently small $\epsilon = \mathrm{poly}(1 / n)$.
    %
       }
\end{table}


%

%

%

\vspace{-0.6cm} %

\paragraph{Techniques.} 
Our techniques build directly upon the algorithmic framework of \cite{karmarkar2025solvingzerosumgames}, which consists of an \emph{outer loop}, \emph{bisection search procedure}, and \emph{inner loop}. Their outer loop is based on the 
%
\emph{prox(imal) point method} \cite{rockafellar1976monotone,martinet1970regularisation}, which reduces solving the original matrix game \eqref{eq:intro-general-matrix-game} to solving a sequence of \emph{regularized} matrix game subproblems. By dynamically searching (via their bisection search procedure) for a particular level of regularization at each iteration of the outer loop, they ensure that each of the regularized matrix game subproblems is \emph{stable} (see \cref{sec:overview-of-approach} for further explanation). This in turn enables their \emph{inner loop} subproblem solver to compute a high accuracy solution for the regularized matrix game subproblem with $\Otilde(\epsilon^c)$ matvecs for a suitable constant $c$. The subproblem solver consists of a \emph{smooth-until-proven-guilty} procedure which leverages the fact that matvecs which do not directly contribute to progress in solving the subproblem must contribute to progress in \emph{learning} the matrix $A$, and therefore can be bounded with careful algorithmic modifications.



Their framework ultimately yields a $\Otilde(\epsilon^{-8/9})$ query complexity for $\ellOneEllOne$ and for $\ellTwoEllOne$ games. Additionally, \cite{karmarkar2025solvingzerosumgames} obtain an improved $\Otilde(\epsilon^{-7/9})$ query complexity for $\ellTwoEllOne$ games via an amortized analysis which involves maintaining an approximation of the matrix $A$ between regularized matrix game subproblems (so that progress made in learning $A$ is not lost between subproblems). 

At a high level, our framework follows a similar approach to their algorithm for $\ellTwoEllOne$ games. In particular, we also use an amortized analysis and have an outer loop, bisection search procedure, and inner loop. However, our outer loop and amortized analysis differ substantially from \cite{karmarkar2025solvingzerosumgames}. Regarding the former, we develop what we term a \emph{prox multi-point method}, which generalizes the standard 
%
prox point method. We show that by carefully applying this new general method, we can achieve tighter control of the total change in the regularized matrix game subproblems that the inner loop solves. Beyond yielding an improved query complexity, the prox multi-point method primitive enables a simpler and perhaps more flexible amortization argument than the $\Otilde(\epsilon^{-7/9})$ algorithm of \cite{karmarkar2025solvingzerosumgames} for $\ellTwoEllOne$ games. Via this new framework and improved analysis, our framework also extends the amortized argument directly to $\ellOneEllOne$ games.


Excitingly, following our approach hits a natural algorithmic limit suggested by \cite{karmarkar2025solvingzerosumgames}. In particular, their outer loop iteration complexity (namely, the number of regularized matrix game subproblems solved) is precisely $\Otilde(\epsilon^{-2/3})$. Furthermore, the iteration complexity of our \emph{prox multi-point method} is also $\Otilde(\epsilon^{-2/3})$. In other words, this paper stretches the amortization argument---when it comes to the average cost of the regularized matrix game subproblems---to a natural limit: the subproblems can be solved with $\Otilde(1)$ matvecs on average. Additionally, our framework arguably simplifies aspects of \cite{karmarkar2025solvingzerosumgames}, as discussed in \Cref{sec:overview-of-approach}, albeit at the expense of a more complicated outer loop. That said, we believe the prox multi-point method outer loop may be of independent interest and we hope this work provides valuable technical tools for improving the complexity of solving broader classes of structured optimization problems beyond matrix games.



%


%


%


%





%



%

%

%

%



\paragraph{Paper organization.} We define notation and cover preliminaries in Section~\ref{sec:prelims}. Leveraging this notation, we provide a detailed technical overview in Section~\ref{sec:overview-of-approach} which reviews the framework of \cite{karmarkar2025solvingzerosumgames} in greater depth and motivates our approach. The remainder of the paper gives our outer loop (\Cref{sec:combined-outer-loop-section}), bisection search (\Cref{sec:MDMP-implementation}), and inner loop (\Cref{sec:sug-solver}), which we ultimately put together \Cref{sec:putting-together} to obtain our results. Standard technical details are deferred to the appendix. 




%



%



%


%



 
%
\section{Preliminaries}\label{sec:prelims}

%

%



%

\paragraph{General notation.} For a vector $z \in \R^d$, we write $[z]_i$ for its $i$-th entry, $\innorm{z}_p$ for its $\ell_p$-norm, and $\diag(z) \in \R^{d \times d}$ for the diagonal matrix where the $(i, i)$-entry is $[z]_i$. If $z \in \zset \subseteq \R^d$ where $\zset = \xset \times \yset$ is a product space for $\xset \subseteq \R^n$ and $\yset \subseteq \R^m$, we write $z\x \in \xset$ and $z\y \in \yset$ for the first $n$ and last $m$ components of $z$, respectively. We refer to vectors in the $\ell_2$-unit ball, denoted in $d$-dimensions by $\ball^d$, as \emph{unit vectors}, and define $\normalize(z) \defeq z / \innorm{z}_2$ for vectors $z \ne 0$ and $\normalize(0) \defeq 0$. 
For $k \in \Z_{> 0}$ and $\ell \in \Z_{\ge 0}$, we use the notation $[k] = \inbraces{1, 2, \dots, k}$ and $[\ell]_0 = \inbraces{0, 1, \dots, \ell}$. We let $[0] \defeq \emptyset$ and use the convention that a summation over an empty index set is zero (e.g., $\sum_{t \in [0]} 1 = 0$). For sequences (of numbers, vectors, etc.) $\smash{u_1, u_2, \dots, u_T}$ or $\smash{u_1, w_1, u_2, w_2, \dots, u_T, w_T}$ we may use the notation $\{u_k\}_{t \in [T]} = \{u_k\}_{t = 1}^T$ and $\{u_k, w_k\}_{t \in [T]} = \{u_k, w_k\}_{t = 1}^T$ respectively. If $\uset$ is a multiset and $\zset$ is a set, we write $\uset \subseteq \zset$ to denote that $u \in \uset$ implies $u \in \zset$.

%

%




%



We write, e.g., $0_n$ and $0_{n \times m}$ for the zero vector in $\R^n$ and zero matrix in $\R^{n \times m}$ respectively. For any vectors $x, x' \in \R^d$ and $c > 1$, we use the shorthand $x \approx_c x'$ to denote that for every $i \in [d]$, $[x']_i/c \leq [x]_i \leq c [x']_i$. For a matrix $B$, we denote its $i$-th row and $i$-th column by $\smash{B_{i,:}}$ and $\smash{B_{:,i}}$ respectively. We further let $\innorm{B}_F \defeq \sqrt{\sum_{i, j} B_{i j}^2}$ denote its Frobenius norm, $\inmaxnorm{A} \defeq \max_{i, j} |B_{ij}|$ denote its max norm, and $\norm{B}_{p \to q} \defeq \max_{\innorm{x}_p \le 1} \innorm{Bx}_q$ denote the $p \to q$ induced norm. We let $f_B(x, y) \defeq y^\top B x$ denote the bilinear form in $B$. For symmetric matrices $A, B \in \R^{d \times d}$, we use $B \preceq A$ to denote that $(A-B)$ is positive semi-definite. 

\paragraph{Simplices, entropy, and KL divergence.} We let $\simplex^d$ denote the $d$-dimensional probability simplex, and further define, for $\nu > 0$, the sets $\Delta_\nu^d \defeq \{p \in \Delta^d: [p]_i > \nu, ~\forall i \in [d]\}$ and $\Delta_{>0}^d \defeq \{p \in \Delta^d : [p]_i > 0, ~\forall i \in [d]\}$. For any $d > 0$, we let $e: \R^d_{\geq0} \to \R$ denote the negative entropy function, i.e., $e(x) = \sum_{i \in [d]} [x]_i\log([x]_i)$ with $e(0) \defeq 0$. We denote the KL divergence by $\KL(x || x') \defeq \sum_{i \in [d]} [x]_i \log([x]_i/[x']_i)$ for $x\in \Delta^d$ and $x' \in \Delta^d_{>0}$, where we let $\smash{0 \log 0 \defeq 0}$.

%



%


%

%

%

\paragraph{Problem setups.} The following definition is adapted from \cite[Definition 1.6]{karmarkar2025solvingzerosumgames} and modified to assume the distance-generating function $r$ is twice differentiable. This assumption, while nonstandard in general, is typical when working with local norms as we frequently do throughout (see \Cref{def:product-dgf-setups} below).
%
%
 We note that the results of \Cref{subsec:prox-multi-general-monotone-ops} in particular only require $r$ to be differentiable.



%

%

\begin{definition}[dgf setup]\label{def:dgf-setup}
We say $\dgfsetup = (\zset, r)$ is a \emph{dgf setup} if: (i) $\zset \subset \R^d$ is compact and convex; and (ii) $r : \zset \to \R$, referred to as the \emph{distance-generating function (dgf)}, is twice differentiable and 1-strongly convex over $\zset$ with respect to some norm $\normInline{\cdot} : \R^d \to \R$. For any $z, z' \in \zset$, $\breg{z}{z'} \defeq r(z') - r(z) - \inangle{ \grad r(z),  z' - z }$ denotes the \emph{Bregman divergence} induced by the dgf $r$.
%
\end{definition}



%

%
%

For reasons similar to those in \citep{karmarkar2025solvingzerosumgames}, in our analysis, it is helpful to leverage a notion of \emph{local norms}, which has also been leveraged extensively in prior work on optimization theory and matrix games \citep{karmarkar2025solvingzerosumgames, carmon2019variance, clarkson2012sublinear, anuran2015studyoflocalapproximationsininfotheory, shwartz2012onlinelearning}. In general, a local norm is a function $\norm{\cdot}_{z}^{\mathsf{loc}}: \cZ \to \R_{\geq 0}$ which, for every $z \in \cZ$, is a norm. In order to introduce the specific local norms we consider in this paper, for convenience, we capture general dgf setups arising from product spaces in the following definition. Recall from above that we use $z\x \in \xset$ and $z\y \in \yset$ to denote the components of a given $z \in \zset$.

\begin{definition}[Product dgf setup and local-norm notation]
    \label{def:product-dgf-setups}
    For dgf setups $\dgfsetup\x = (\xset \subset \R^n, \rx)$ and $\dgfsetup\y = (\yset \subset \R^m, \ry)$, we say $\dgfsetup = (\zset \subset \R^d, r)$ is the \emph{product dgf setup induced by $\dgfsetup\x$ and $\dgfsetup\y$,} denoted $\dgfsetup = \prodsetup(\dgfsetup\x, \dgfsetup\y)$, if $\zset = \xset \times \yset$ as well as $r(z) = \rx(z\x) + \ry(z\y)$ for all $z \in \zset$. We associate the following local-norm notation with product dgf setups. For any $z, z' \in \zset$, we define the \emph{local norm} $\norm{z}_{z'}^2 \defeq \inangle*{z, \hess r(z') z}.$ Moreover, we define
\begin{align*}
    (z)_{z'} \defeq ((\hess r(z'\x))^{1/2} z\x , (\hess r(z'\y))^{1/2} z\y ) \in \R^d. 
\end{align*}
and for any $B \in \R^{m \times n}$ and $z' \in \zset$, we define
\begin{align*}
    (B)_{z'} \defeq (\hess r(z'\y))^{-1/2} B (\hess r(z'\x))^{-1/2} \in \R^{m \times n} ~\text{and}~ (B)_{z', *}  \defeq (\hess r(z'\y))^{1/2} B (\hess r(z'\x))^{1/2} \in \R^{m \times n}.
\end{align*}
\end{definition}

Note that in Definition~\ref{def:product-dgf-setups}, the transformation $\norm{\cdot}_{z'}$ performs the appropriate \emph{change of basis} such that $\normInline{z}_{z'}^2 = \normInline{(z)_{z'}}_2^2$. Similarly, the mapping $\ground{A}{z'}$ performs the corresponding change of basis to $A$ to maintain the invariant that $\inangle{z\y, A z\x} = \inangle{{\ground{z}{z'}}\y, \ground{A}{z'} {\ground{z}{z'}}\x}$. In turn, $\unground{A}{z'}$ inverts this change of basis. This is formalized in the following staight-forward fact. 

%
%
%



%
%
%
%
%
%
%
%
%
%
%


\begin{fact}\label{lemma:ungrounding} Letting $\dgfsetup\x, \dgfsetup\y, \dgfsetup$ be as in Definition~\ref{def:product-dgf-setups}, for any $z, z' \in \cZ$, we have $\inangle{z\y, A z\x} = \inangle{{\ground{z}{z'\y}}, \ground{A}{z'} {\ground{z}{z'\x}}}$ and $\ground{\unground{A}{z'}}{z'} = \unground{\ground{A}{z'}}{z'} = A$. Moreover, $\normInline{\ground{z}{z'}}_2^2 = \norm{z}_{z'}^2$. 
\end{fact}


%

In the rest of this section, we introduce further notations and definitions associated with dgf setups which will be used in the remainder of the paper.

%
%
%
%
%
%
%
%
%
%
%
%
%
%
%
%
%

%
%
%

%
%
%

%


%

%
%
%
%



%
%
%
%
%
%
%
%
%
%
%
%
%
%
%
%
%
%
%
%
%
%

%

%

%
%
%

%
%
%
%
%
%
%

%
%

%

%

\paragraph{Monotone operators and proximal mappings.} First, we review notation related to monotone operators and proximal mappings. Given a dgf setup $(\zset, r)$ (as in Definition~\ref{def:dgf-setup}), an operator $g: \zset \to \R^d$ is said to be \emph{$\alpha$-strongly monotone} (with respect to $r$) if for any $z, z' \in \cZ$, we have $\inangle*{g(z') - g(z), z' - z} \geq \alpha \breg{z'}{z}$. If $g$ is $0$-strongly monotone, we may simply say it is \emph{monotone}. In particular, in Sections~\ref{sec:MDMP-implementation} through~\ref{sec:putting-together} we use the following definition and associated notation extensively. 


%




    \begin{definition}[Proximal mappings, Definition 2.2 of \citep{karmarkar2025solvingzerosumgames}, restated]
        \label{def:proximal-mappings}
        For a given dgf setup $(\zset, r)$, monotone operator $g : \zset \to \R^d$, points $z, w \in \zset$, regularization levels $\lambda > 0, \mu \ge 0$, and $\zset' \subseteq \zset$, we let $\prox_{z, w}^{\lambda, \mu}(g; \zset')$ denote the unique $z' \in \zset'$ such that
        \begin{align*}
            \inangle{g(z'), z' - u}
            %
             \le \lambda \insquare{\breg{z}{u} - \breg{z'}{u} - \breg{z}{z'}} + \mu \insquare{\breg{w}{u} - \breg{z'}{u} - \breg{w}{z'}} ~~\text{for all $u \in \zset'$},
        \end{align*}
        and similarly let $\prox_{z}^{\lambda}(g; \zset')$ denote $\prox_{z,z}^{\lambda,0}(g; \zset')$, i.e., the unique $z' \in \zset'$ such that
        \begin{align}
            \label{eq:prox-single}
            \inangle{g(z'), z' - u}
            %
            \le \lambda \insquare{\breg{z}{u} - \breg{z'}{u} - \breg{z}{z'}} ~~\text{for all $u \in \zset'$}.
        \end{align}
        We drop $\zset'$ (e.g., writing $\prox_{z, w}^{\lambda, \mu}(g)$) when $\zset' = \zset$ for brevity. 
        
        Furthermore, in the context of the input to a proximal mapping, we may write a vector $v \in \R^d$ as a stand-in for the associated constant operator $z \mapsto v$.
        As an example, supposing $g : \zset \to \R^d$ is a monotone operator and $v \in \R^d$, then $\prox_{z}^{\lambda}(v + g; Z')$ denotes the unique $z' \in \zset'$ such that
        \begin{align*}
            \inangle{v + g(z'), z' - u}
            %
            \le \lambda \insquare{\breg{z}{u} - \breg{z'}{u} - \breg{z}{z'}} ~~\text{for all $u \in \zset'$}.
        \end{align*}
\end{definition}

\paragraph{Notation for finite multisets $\cU \subset \cZ$.} As mentioned in Section~\ref{sec:intro}, one crucial aspect of our algorithm and its analysis, which enables our improvement over \citep{karmarkar2025solvingzerosumgames}, is that we develop a new method which we call the \emph{prox multi-point method} (discussed further in Sections~\ref{sec:overview-of-approach} and~\ref{sec:combined-outer-loop-section}). At a high level, our prox multi-point method generalizes the proximal point method (see e.g., \citep{rockafellar1976monotone} and more specifically, Algorithm 6.2 of \citep{karmarkar2025solvingzerosumgames}) in that it regularizes with respect to a finite nonempty \emph{multiset} of points $\cU \subset \cZ$ (as opposed to only a single point $ u \in \cZ$). Correspondingly, we frequently work with \emph{sums} of Bregman divergences over a set $\cU$. Consequently, for a dgf setup $\dgfsetup = (\zset \subset \R^d, r)$, a finite nonempty multiset $\uset \subseteq \zset$, and $z \in \zset$, we  use the notation $\breg{\uset}{z'} \defeq \sum_{w \in \uset} \breg{w}{z'}$ for brevity. Similarly, given a multiset $\cU \subseteq \cZ$, we use $\cU\x \defeq \{u\x: u \in \cU\}$, and $\cU\y \defeq \{u\y : u \in \cU\}$ where $\cU\x, \cU\y$ are defined as multisets with multiplicity so that $|\cU|=|\cU\x|=|\cU\y|$. (Note that $\cU\x, \cU\y$ may be multisets even if $\uset$ is not a multiset. For example, if there exist $u, v \in \cU$ such that $u\x = v\x$, then $\cU\x$ will contain both $u\x$ and $v\x$.) Additionally, departing from \citep{karmarkar2025solvingzerosumgames}, for notational convenience when describing and analyzing our aforementioned prox multi-point method, we use the following additional notation. For a finite nonempty multiset $\cU \subseteq \cZ$, we write $\prox_{\cU}^\lambda(g; \cZ')$ to denote 
the unique $z' \in \cZ'$ such that 
\begin{align}\label{eq:set-prox-mapping} 
    \inangle{g(z'), z' - u}
%
\leq \lambda \sum_{z \in \cU} \insquare{\breg{z}{u} - \breg{z'}{u} - \breg{z}{z'}} \text{ for all $u \in \cZ'$}.
\end{align}
%

%
%
%
%
%
%
%
%
%
%
%
%
%
%
%
%
%
%
%
%
%
%
%

Finally, we remark that Bregman divergences satisfy the following (e.g., \cite[Sec. 3.1]{carmon2019variance}),
\begin{align}
    \label{eq:Bregman-three-point-equality}
  \inangle{ - \grad \breg{z}{z'},  z' - u } = \breg{z}{u} - \breg{z'}{u} - \breg{z}{z'},\text{ for all }
  z, z', u \in \cZ
\end{align}
%
%
%
%
where in general $\grad \breg{z}{z'} = \grad r(z') - \grad r(z)$ denotes the gradient of $u \mapsto \breg{z}{u}$ evaluated at $z'$.

%

%

\paragraph{Convex-concave functions, gradient mappings, and regret.} Let $\dgfsetup\x = (\xset \subseteq \R^n, \rx)$ and $\dgfsetup\y = (\yset \subseteq \R^m, \ry)$ be dgf setups with $\dgfsetup = (\zset \subset \R^d, r) \defeq \prodsetup(\dgfsetup\x, \dgfsetup\y)$ (recall Definition~\ref{def:product-dgf-setups}). We say $f : \xset \times \yset \to \R$ is a \emph{convex-concave} function if the restrictions of $f$ to the first $n$ and last $m$ inputs are convex and concave functions respectively. We recall the following general solution concept for solving such games:

\begin{definition}[$\epsilon$-solution and gap function, Definition 1.1 of \cite{karmarkar2025solvingzerosumgames}, restated]\label{def:epsilon-solution} Let $\epsilon \geq 0$ and $f: \cX \times \cY \to \R$ be a convex-concave function. We say $z = (x, y) \in \cX \times \cY$ is an \emph{$\epsilon$-solution} of $\min_{x \in \cX} \max_{y \in \cY} f(x,y)$ if it is an \emph{$\epsilon$-saddle point}, i.e.,
    \begin{align*}
        \gap(z) \defeq \max_{y' \in \cY} f(x, y') - \min_{x' \in \cX} f(x', y) \leq \epsilon.
    \end{align*}
We say $z$ is an \emph{exact solution} if it is a $0$-solution. 
\end{definition}

For notational convenience, throughout the paper, when $f$ is differentiable, we define the \emph{gradient mapping} $\gm f$ of $f$ via $\gm f(z) \defeq (\nabla\x f(z), - \nabla\y f(z))$, where $\nabla\x f(z)$ and $\nabla\y f(z)$ denote the partial gradients of $f$ with respect to the first $n$ and last $m$ coordinates. The gradient mapping $\gm f$ is a monotone operator when $f$ is convex-concave. Additionally, a useful fact (e.g., \citep{karmarkar2025solvingzerosumgames}) which we leverage, for example, in Section~\ref{sec:putting-together}, is that for $\alpha > 0$ and $z \in \zset$, $\prox_z^\alpha(\gm f)$ is the exact solution of 
\begin{align*}
    \min_{x \in \xset} \max_{y \in \yset} f(x, y) + \alpha \xbreg{z\x}{x} - \alpha \ybreg{z\y}{y}.
\end{align*}




%
%
%
%
%
%
%
%
%
%
%
%
%
%
%

%

%


As is standard in the literature, our algorithms obtain $\epsilon$-solutions by achieving low regret with respect to the operator $\gm f$. First, we define regret for a general operator $g$ in the following definition, and then we give a standard result which reduces obtaining an $\epsilon$-solution to regret minimization with respect to $\gm f$. Note that in both cases we allow nonuniform weights $\lambda^{(t)} / \Lambda$; this will be important for our algorithms in \Cref{sec:combined-outer-loop-section}.

\begin{definition}[Regret]
    \label{def:regret}
With $\zset' \subseteq \zset \subset \R^d$, let $g : \zset \to \R^d$; $z^{(1)}, \dots, z^{(T)} \in \zset$; and $\lambda^{(1)}, \dots, \lambda^{(T)} > 0$. With $\Lambda \defeq \sum_{t \in [T]} \lambda^{(t)}$, we define
\begin{align*}
    \regret_g(\inbraces{z^{(t)}, \lambda^{(t)}}_{t \in [T]}; \zset') \defeq \sup_{u \in \zset'} \inbraces*{
        \frac{1}{\Lambda} \sum_{t \in [T]} \lambda^{(t)} \inangle{g(z^{(t)}), z^{(t)} - u},
    }
\end{align*}
where $\regret_g(\inbraces{z^{(t)}, \lambda^{(t)}}_{t \in [T]}; \zset')$ is called the \emph{regret of the sequence $z^{(1)}, \dots, z^{(T)}$ (with respect to the operator $g$, weights $\inbraces{\lambda^{(t)}}_{t \in [T]}$, and set $\zset'$).} We may drop $\zset'$, writing $\regret_g(\inbraces{z^{(t)}, \lambda^{(t)}}_{t \in [T]})$, when $\zset' = \zset$ for brevity.
\end{definition}

\begin{lemma}[Lemma 2.3 of \citep{karmarkar2025solvingzerosumgames}, restated]
    \label{lem:regret-bounds-the-gap}
    Let $f : \xset \times \yset \to \R$ be a convex-concave function over compact, convex sets $\xset \subset \R^n$ and $\yset \subset \R^m$, with $\zset \defeq \xset \times \yset$. Then for any $z^{(1)}, \dots, z^{(T)} \in \zset $ and $\lambda^{(1)}, \dots, \lambda^{(T)} > 0$, letting $\Lambda \defeq \sum_{t \in [T]} \lambda^{(t)}$ and $\zbar \defeq \frac{1}{\Lambda} \sum_{t \in [T]} \lambda^{(t)} z^{(t)}$, we have $\gap(\zbar) \le \regret_{\gm f}(\inbraces{z^{(t)}, \lambda^{(t)}}_{t \in [T]}; \zset)$.
%
%
%
%
%
\end{lemma}
 
%

\section{Technical overview}
\label{sec:overview-of-approach}

In this section, we motivate and provide an overview of our algorithmic framework for proving Theorems \ref{thm:final-result-l1-l1-aka-zero-sum} and \ref{thm:final-result-l2-l1-aka-SVM}. We start by providing an overview of the approach we build upon, namely, the algorithm due to \cite{karmarkar2025solvingzerosumgames} which obtains a $\Otilde(\epsilon^{-7/9})$ matvec complexity for $\ellTwoEllOne$ games. Since we build upon their algorithm for $\ellTwoEllOne$ games, we use $\ellTwoEllOne$ games as a representative example throughout this section, including when we explain our own framework. However, we emphasize that unlike the $\Otilde(\epsilon^{-7/9})$ algorithm for $\ellTwoEllOne$ games given by \cite{karmarkar2025solvingzerosumgames}, our framework extends immediately to $\ellOneEllOne$ games, as discussed further below.


\paragraph{Assumptions.} Throughout this section (Section~\ref{sec:overview-of-approach}), we fix the dgf setups $\dgfsetup\x \defeq (\xset \defeq \ball^n, \rx)$ and $\dgfsetup\y \defeq (\yset \defeq \simplex^m, \ry)$ with $\dgfsetup = (\zset, r) \defeq \prodsetup(\dgfsetup\x, \dgfsetup\y)$ (Definition~\ref{def:product-dgf-setups}).\footnote{For technical reasons, \cite{karmarkar2025solvingzerosumgames} start by reducing the original $\ellTwoEllOne$ game to the same game except the probability-simplex strategy space of the $y$-player is appropriately truncated. We truncate simplex domains in our paper for similar reasons. However, since this truncation introduces additional notation and does not alter the core intuition, we omit details related to this point in our technical overview for brevity.} We further define $\rx(x) \defeq \frac{1}{2} \innorm{x}_2^2$ and $\ry(y) \defeq e(y)$, where $e$ denotes the negative entropy function per Section~\ref{sec:prelims}. With these choices, the $\ellTwoEllOne$ game is given by \eqref{eq:intro-general-matrix-game}.


\paragraph{The approach of \cite{karmarkar2025solvingzerosumgames} for $\ellTwoEllOne$ games.} Recall from the discussion in \Cref{sec:intro} that the algorithmic framework of \cite{karmarkar2025solvingzerosumgames} consists of an \emph{outer loop}, \emph{bisection search procedure}, and \emph{inner loop}. Their outer loop is based on the \emph{prox(imal) point method} \cite{rockafellar1976monotone,martinet1970regularisation}, which reduces solving matrix games to approximately solving\footnote{As in Section~\ref{sec:intro}, by ``solving'' we mean solving the pair of minimax and maximin problems simultaneously, but we omit the latter throughout this section for brevity.} a sequence of regularized matrix game subproblems of the form 
%
\begin{align}
    \label{eq:prox-point-subproblem}
    \min_{x \in \xset} \max_{y \in \yset} f_A(x, y) + \alpha^{(t)} \xbreg{z^{(t - 1)}\x}{x} - \alpha^{(t)} \ybreg{z^{(t - 1)}\y}{y}
\end{align}
for $t = 1, 2, \dots, T$, where $\alpha^{(t)} > 0$ and $z^{(t - 1)} \in \zset$. (Recall that $f_A(x, y) \defeq x^\top A y$.) In particular, the next iterate $z^{(t)}$ is set to an approximate solution of \eqref{eq:prox-point-subproblem}, and the regularization level $\alpha^{(t)}$ is set dynamically via their \emph{bisection search procedure} (discussed further below). Their \emph{inner loop} subproblem solver is designed to solve a constrained version of \eqref{eq:prox-point-subproblem} (discussed next) to high accuracy.

%

Letting $\zopt_{\alpha^{(t)}}$ denote the exact solution of \eqref{eq:prox-point-subproblem} viewed as a function of $\alpha^{(t)}$, formally $\zopt_{\alpha^{(t)}} \defeq \prox_{z^{(t - 1)}}^{\alpha^{(t)}}(\gm f_A)$, the key starting observation of \cite{karmarkar2025solvingzerosumgames} is that if $\breg{z^{(t - 1)}}{\zopt_{\alpha^{(t)}}} = O((\alpha^{(t)})^2)$, then the subproblem \eqref{eq:prox-point-subproblem} is \emph{stable} in the following sense. Using only $O(1)$ matvecs, it is possible to obtain a point $\zground^{(t)}$ such that $\zground^{(t)}$ and $\zopt_{\alpha^{(t)}}$ are \emph{multiplicatively close} in the simplex-constrained coordinates, namely $\zgroundy^{(t)} \approx_c \zopt_{\alpha^{(t)} \ysub}$ with $c = O(1)$.\footnote{See Sections 1.4 and 6.2 in \cite{karmarkar2025solvingzerosumgames} for further details regarding this stability result. Regarding terminology, we note that \cite{karmarkar2025solvingzerosumgames} refers to the points $\zground$ as \emph{local norm points} (and often uses the corresponding notation $z_{\mathsf{n}}$) due to the fact that they determine the change of basis/local norm in which the matrix $A$ is learned, as discussed later in this section. We instead refer to them as \emph{center points} to emphasize that they are the centers of the stable regions.}
%
%
Thus, under the assumption $\breg{z^{(t - 1)}}{\zopt_{\alpha^{(t)}}} = O((\alpha^{(t)})^2)$, we can restrict the domain $\yset$ in \eqref{eq:prox-point-subproblem} to a \emph{stable region} of the form $\ysetstable \defeq \inbraces{y \in \yset : y \approx_{c'} \zgroundy^{(t)}}$ without loss of generality, where $c' = O(1)$.

With this restriction of $\yset$ to $\ysetstable$, the \emph{inner loop} subproblem solver of \cite{karmarkar2025solvingzerosumgames} is able to solve the resulting problem
\begin{align}
    \label{eq:prox-point-subproblem-stable}
    \min_{x \in \xset} \max_{y \in \ysetstable} f_A(x, y) + \alpha^{(t)} \xbreg{z^{(t - 1)}\x}{x} - \alpha^{(t)} \ybreg{z^{(t - 1)}\y}{y}
\end{align}
to high accuracy efficiently.
%
The latter is achieved via a modified variant of the standard \emph{mirror prox} algorithm for strongly-monotone variational inequalities \cite{nem04}, which is termed \emph{smooth-until-proven-guilty composite strongly monotone  mirror prox} or $\SUGStronglyMonotoneMirrorProx$ for short \cite[Alg. 5.2]{karmarkar2025solvingzerosumgames}.
%
%
In particular, $\SUGStronglyMonotoneMirrorProx$ maintains a decomposition of the matrix $A$ of the form $A = (A - M) + M$, where $M \in \R^{m \times n}$ is an explicitly known \emph{model} of the matrix $A$, and $A - M$ is the \emph{residual} (which is not explicitly known and must be accessed via matvecs).
%
 Letting $\Min$ denote the initial model given as input to $\SUGStronglyMonotoneMirrorProx$ and $\Mout$ denote the final model returned by $\SUGStronglyMonotoneMirrorProx$, $\SUGStronglyMonotoneMirrorProx$ is guaranteed to return a high accuracy solution\footnote{We leave defining ``high accuracy'' and the exact solution concept obtained by the subproblem solver to the technical sections.} to \eqref{eq:prox-point-subproblem-stable} with
\begin{align}
    \label{eq:SUG-SM-MP-bound}
    \Otilde\inparen*{      \frac{\innorm{(A - \Min)_{\zground^{(t)}}}_F^2 -   \innorm{(A - \Mout)_{\zground^{(t)}}}_F^2 }{\tau^2}     + \frac{\tau}{ \alpha^{(t)}} }
\end{align}
matvecs, where $\tau > 0$ is a hyperparameter \cite[Theorem 5.9]{karmarkar2025solvingzerosumgames}. (Here, recall the local-norm/change-of-basis notation $(B)_{z'}$ from Definition~\ref{def:product-dgf-setups}.) Following the terminology of \cite{karmarkar2025solvingzerosumgames}, the first term in \eqref{eq:SUG-SM-MP-bound} bounds the number of \emph{model-update iterations}, which are iterations within $\SUGStronglyMonotoneMirrorProx$ where the algorithm makes progress in learning $(A)_{\zground^{(t)}}$. (We give further details when discussing our own approach below.) The second term bounds the number of \emph{progress iterations} in which progress is made in converging to an approximate solution of \eqref{eq:prox-point-subproblem-stable}. 

The next key step of \cite{karmarkar2025solvingzerosumgames} is to choose $\alpha^{(t)}$ optimally from the perspective of the prox point \emph{outer loop,} while still ensuring that the $\breg{z^{(t - 1)}}{\zopt_{\alpha^{(t)}}} = O((\alpha^{(t)})^2)$ implementability requirement of the \emph{inner loop} subproblem solver is met. In particular, the prox point method converges faster when the regularizers $\alpha^{(t)}$ are small (less regularization leads to more progress in a single step). Thus, \cite{karmarkar2025solvingzerosumgames} gives a \emph{bisection search procedure} to obtain $\alpha^{(t)}$ such that $\breg{z^{(t - 1)}}{\zopt_{\alpha^{(t)}}} = \Theta((\alpha^{(t)})^2)$ (or else $\alpha^{(t)}$ is set to a minimum level of regularization $\beta = \epsilon^{1/3}$; we leave the details to \cite[Sec. 6.3]{karmarkar2025solvingzerosumgames} and our modified bisection search procedure in Section~\ref{sec:MDMP-implementation}).
%
This bisection search procedure calls $\SUGStronglyMonotoneMirrorProx$ at most a polylogarithmic number of times within each iteration $t$ of the prox point outer loop. Furthermore, it is performed so that the total cost of the procedure is dominated by the final call to $\SUGStronglyMonotoneMirrorProx$ to obtain $z^{(t)}$ (up to multiplicative polylog factors). For simplicity (since our corresponding bisection search procedure in Section~\ref{sec:MDMP-implementation} also contributes only polylogarithmic overhead), we omit such details in this overview and assume that $\SUGStronglyMonotoneMirrorProx$ is called only once in each iteration of the outer loop (implying a single value of $\zground^{(t)}$ per outer-loop iteration) so as to obtain $z^{(t)}$. This simplification can equivalently be viewed as only tallying the total cost of the final $\SUGStronglyMonotoneMirrorProx$ calls over all individual bisection search procedures (since the total cost of all other $\SUGStronglyMonotoneMirrorProx$ calls is dominated up to polylog factors).

%

%

Then, by ensuring $\alpha^{(t)}$ satisfies $\breg{z^{(t - 1)}}{\zopt_{\alpha^{(t)}}} = \Theta((\alpha^{(t)})^2)$, \cite{karmarkar2025solvingzerosumgames} obtains an $\Otilde(\epsilon^{-2/3})$ iteration guarantee for their outer loop \cite[Corollary 6.18]{karmarkar2025solvingzerosumgames} (namely, this bounds the number of subproblems \eqref{eq:prox-point-subproblem} required to solve \eqref{eq:intro-general-matrix-game}). With this piece, we review the analysis of the final matvec complexity achieved by their algorithm. Their algorithm sets the initial model for $A$ to $0_{m \times n}$. In particular, this is $\Min$ for the first time the subproblem solver $\SUGStronglyMonotoneMirrorProx$ is called. Inputs $\Min$ to subsequent calls to $\SUGStronglyMonotoneMirrorProx$ are set to the value of $\Mout$ from the previous call. As a final step before stating the complexity, they guarantee $\alpha^{(t)} \ge \epsilon^{1/3}$ for all iterations $t$ by ensuring a minimum level of regularization $\beta = \epsilon^{1/3}$ in their outer loop and bisection search procedure (as mentioned above, we leave the details to \cite[Sec. 6.3]{karmarkar2025solvingzerosumgames} and Section~\ref{sec:MDMP-implementation}). Then, combining the $\Otilde(\epsilon^{-2/3})$ iteration bound of the outer loop with the cost \eqref{eq:SUG-SM-MP-bound} of each iteration (though we ignore it in this section as discussed above, the bisection search procedure to find $\alpha^{(t)}$ contributes only logarithmic overhead), they obtain a final matvec bound for solving \eqref{eq:intro-general-matrix-game} of 
\begin{align}
    \label{eq:prev-paper-final-complexity}
    \Otilde\inparen*{   \frac{1}{\tau^2} \sum_{t \ge 1} \insquare*{\innorm{(A - M^{(t - 1)})_{\zground^{(t)}}}_F^2 -   \innorm{(A - M^{(t)})_{\zground^{(t)}}}_F^2 }    + \tau \epsilon^{-1} } \, ,
\end{align}
where $M^{(t - 1)}$ denotes the model passed to $\SUGStronglyMonotoneMirrorProx$ when it is called for the $t$-th time. They then further bound 
\begin{align}
    \label{eq:prev-paper-model-update-bound}
    \sum_{t \ge 1} \insquare*{\innorm{(A - M^{(t - 1)})_{\zground^{(t)}}}_F^2 -   \innorm{(A - M^{(t)})_{\zground^{(t)}}}_F^2 }  \le O(1) \cdot \sum_{t \ge 1} \innorm{\zground^{(t + 1)} - \zground^{(t)}} \le \Otilde( \epsilon^{-1/3} )
\end{align}
in Lemma 6.31. Substituting \eqref{eq:prev-paper-model-update-bound} into \eqref{eq:prev-paper-final-complexity} and choosing $\tau \gets \epsilon^{2/9}$ yields their final $\Otilde(\epsilon^{-7/9})$ complexity. 

%
%
%





\paragraph{A key challenge faced by the prior work.} The core bottleneck of the approach of \cite{karmarkar2025solvingzerosumgames} is the argument \eqref{eq:prev-paper-model-update-bound}.
%
As mentioned above, within iteration $t$ of their outer loop, model-update steps made by $\SUGStronglyMonotoneMirrorProx$ make progress in learning $(A)_{\zground^{(t)}}$. Thus, the basis in which progress is made in learning $A$ changes each iteration, and therefore \cite{karmarkar2025solvingzerosumgames} perform a technical and somewhat ad hoc analysis to argue that not too much progress is lost when the basis $\zground^{(t)}$ updates. In particular, if the right side of \eqref{eq:prev-paper-model-update-bound} was $\Otilde(1)$, then a $\Otilde(\epsilon^{-2/3})$ matvec complexity would be achievable by choosing $\tau \gets \epsilon^{1/3}$ in \eqref{eq:prev-paper-final-complexity}, but it is not clear how to tighten \eqref{eq:prev-paper-model-update-bound}. Additionally, their argument seems tailored to the specific form of $\ellTwoEllOne$ games. For example, to obtain \eqref{eq:prev-paper-model-update-bound}, they use a \emph{one-sided} projection argument (see Algorithm 6.3 in their paper) within each model-update iteration of $\SUGStronglyMonotoneMirrorProx$ which does not seem to directly extend to $\ellOneEllOne$ games. Our approach, described next, fixes both of these issues (obtaining a $\Otilde(\epsilon^{-2/3})$ complexity and extending to $\ellOneEllOne$ games) simultaneously.

%

%


\paragraph{An attempt at a fix.} Towards overcoming the challenges in the prior work identified above, the first key design choice we make is to change the \emph{representation} of the matrix $(A)_{\zground^{(t)}}$.
%
%
To ease our explanation of this idea,
%
 we temporarily make the simplifying assumption that $z^{(t)} = \zground^{(t)}$ for all iterations $t \ge 1$. (We discuss how to remove it at the end of this overview.)
%
Then, note that we can equivalently represent $(A)_{\zground^{(t)}} = (A)_{z^{(t)}}$ via a \emph{telescoping sum} using the previous iterates $z^{(0)}, z^{(1)}, \dots, z^{(t - 1)}\in \zset$ obtained from the outer loop (recall \eqref{eq:prox-point-subproblem} and the surrounding text) via 
%
\begin{align}
    \label{eq:first-try-telescoping-representation}
    (A)_{z^{(t)}} &= \sum_{j = 0}^{t} \Delta_{j - 1, j}, ~~\text{where}~~ \Delta_{j - 1, j} \defeq (A)_{z^{(j)}} - (A)_{z^{(j - 1)}} ~\text{for}~  0 \le j \le t .
\end{align}
%
Here, we overload notation and define $(A)_{z^{(-1)}} \defeq 0_{m \times n}$ for brevity.
%


%

Next, suppose we maintain a separate model $M_{j - 1, j}^{(t - 1)}$ for each $\Delta_{j - 1, j}$ matrix ($0 \le j \le t$). Analogously to in the recap of the previous work given above, the superscript $t - 1$ in $M_{j - 1, j}^{(t - 1)}$ is used to denote the state of the model at the beginning of the $t$-th iteration of the outer loop, before $\SUGStronglyMonotoneMirrorProx$ has been called to obtain $z^{(t)}$. 
%
(Of course, we technically already have $z^{(t)}$ before calling $\SUGStronglyMonotoneMirrorProx$ by the simplifying assumption $z^{(t)} = \zground^{(t)}$, but our goal is to convey the intuition for where our improved complexities come from.) 
Note then that we can obtain a model $M^{(t - 1)}$ for $(A)_{z^{(t)}}$ at the start of the $t$-th iteration via the representation \eqref{eq:first-try-telescoping-representation}; we simply set $M^{(t - 1)} \defeq \sum_{j = 0}^t M_{j - 1, j}^{(t - 1)}$, in which case the corresponding residual is $(A)_{z^{(t)}} - M^{(t - 1)}$.\footnote{Note that in our recap of the method of \cite{karmarkar2025solvingzerosumgames} above, we let $M^{(t - 1)}$ denote a model for $A$, which we then performed a change of basis to, i.e., $(M^{(t - 1)})_{\zground^{(t)}}$, to obtain a model for $(A)_{\zground^{(t)}}$. To maintain consistency with the notation of Section~\ref{sec:combined-outer-loop-section}, we let $M^{(t - 1)}$ denote a model for $(A)_{\zground^{(t)}}$ when describing our framework.} Thus, we can apply $\SUGStronglyMonotoneMirrorProx$ as before, with the difference that in every model-update iteration we update \emph{all} of the models $M_{j - 1, j}^{(t - 1)}$ for $0 \le j \le t$. In anticipation of its formal statement in Algorithm~\ref{alg:subsolver} in Section~\ref{sec:sug-solver}, we refer to this modified version of $\SUGStronglyMonotoneMirrorProx$ as the \emph{smooth-until-proven-guilty solver} or $\SUG$ (Algorithm~\ref{alg:subsolver}) for short.



%
 
%

%

In particular, a model-update iteration of $\SUG$ occurs if, using $O(1)$ matvecs, we discover unit vectors $u \in \ball^m$ and $v \in \ball^n$ which exhibit large bilinear alignment with the current residual, formally:
\begin{align}
    \label{eq:guilty-broken-apart}
    u^\top \inparen{ (A)_{z^{(t)}} - M} v = \sum_{j = 0}^t u^\top (\Delta_{j - 1, j} -  M_{j - 1, j}) v \ge \tau \, .
\end{align}
Here, recall that the models $M_{j - 1, j}^{(t - 1)}$ (and corresponding induced model $M^{(t - 1)}$) have been given as input to $\SUG$, and we drop the superscripts in \eqref{eq:guilty-broken-apart} to denote the fact that each $M_{j - 1, j}$ is some (potentially) intermediate model iterate (namely, some updates to the input $M_{j - 1, j}^{(t - 1)}$ matrices may have already been performed). Similarly, $M \defeq \sum_{j = 0}^t M_{j - 1, j}$ is the corresponding intermediate model iterate for $(A)_{z^{(t)}}$.

%


We now make two assumptions which enable us to obtain an $\Otilde(\epsilon^{-2/3})$ matvec bound for solving \eqref{eq:intro-general-matrix-game}. While both of these assumptions do not hold under the description of the method so far, we will subsequently show how to modify the method to work around them. (In particular, the resulting bounds can be sufficiently approximated when the number of terms in the representation \eqref{eq:first-try-telescoping-representation} is $\Otilde(1)$, which we show how to guarantee later.) First, suppose that the individual contributions in the summation in \eqref{eq:guilty-broken-apart} are concentrated in the sense that there exists some $0 \le \jspecial \le t$ such that $u^\top (\Delta_{\jspecial - 1, \jspecial} -  M_{\jspecial - 1, \jspecial}) v \ge \Omega(\tau)$. Second, suppose it is possible to discover this index $\jspecial$ using at most $O(1)$ matvecs.
%
%
%
%
%
%
Under these assumptions, it is straightforward to choose the subsequent models $M_{j - 1, j}'$ using only $O(1)$ matvecs so that
\begin{align}
    \label{eq:hypothetical-guilty-bound}
   \sum_{j  =  0 }^t \innorm{\Delta_{j - 1, j} -  M_{j - 1, j}}_F^2 -  \sum_{j  =  0 }^t \innorm{\Delta_{j - 1, j} -  M'_{j - 1, j}}_F^2 \ge \Omega( \tau^2 )\,.
\end{align}
%
%
%
In particular, we can update $M'_{\jspecial - 1, \jspecial} \gets M_{\jspecial - 1, \jspecial} + u^\top (\Delta_{\jspecial - 1, \jspecial} -  M_{\jspecial - 1, \jspecial}) v \cdot u v^\top$ and otherwise set $M'_{j - 1, j} \gets M_{j - 1, j}$ for $j \ne \jspecial$.

Thus, under these assumptions, $\SUG$ returns a high accuracy solution with
\begin{align}
    \label{eq:hypothetical-subproblem-bound}
    \Otilde\inparen*{    \tau^{-2} \insquare*{\sum_{j  =  0 }^t \innorm{\Delta_{j - 1, j} -  M^{(t - 1)}_{j - 1, j}}_F^2 -  \sum_{j  =  0 }^t \innorm{\Delta_{j - 1, j} -  M^{(t)}_{j - 1, j}}_F^2}  +  \tau / \alpha^{(t)} } 
\end{align}
matvecs in iteration $t$ of the outer loop, where $M^{(t)}_{j - 1, j}$ is set to the final value of the internal iterate $M_{j - 1, j}$ maintained by $\SUG$. Indeed, if $J$ model-update iterations happen with the $\SUG$ call, we have $\sum_{j  =  0 }^t \innorm{\Delta_{j - 1, j} -  M^{(t - 1)}_{j - 1, j}}_F^2 -  \sum_{j  =  0 }^t \innorm{\Delta_{j - 1, j} -  M^{(t)}_{j - 1, j}}_F^2 \ge \Omega(J \tau^2)$, yielding the bound on $J$ given by the first term in \eqref{eq:hypothetical-subproblem-bound}. The second term in \eqref{eq:hypothetical-subproblem-bound} again bounds the number of progress iterations, which remains unchanged from \eqref{eq:SUG-SM-MP-bound}. Then, using the $\Otilde(\epsilon^{-2/3})$ outer-loop iteration bound of \cite{karmarkar2025solvingzerosumgames} and the fact that it is possible to guarantee $\alpha^{(t)} \ge \epsilon^{1/3}$ for all iterations $t$ as before, we can obtain a final matvec bound for solving \eqref{eq:intro-general-matrix-game} of
\begin{align}
    & \Otilde\inparen*{    \tau^{-2} \sum_{t \ge 1} \insquare*{\sum_{j  =  0 }^t \innorm{\Delta_{j - 1, j} -  M^{(t - 1)}_{j - 1, j}}_F^2 -  \sum_{j  =  0 }^t \innorm{\Delta_{j - 1, j} -  M^{(t)}_{j - 1, j}}_F^2}  +  \tau \epsilon^{-1} } \nonumber \\
    \le &
    \Otilde\inparen*{    \tau^{-2} \sum_{j \ge 0} \innorm{\Delta_{j - 1, j}}_F^2 +  \tau \epsilon^{-1} } . \label{eq:hypothetical-final-matvec-bound}
\end{align}
Here, the inequality is obtained by initializing $M_{j - 1, j}^{(j - 1)} \gets 0_{m \times n}$ and using the fact that the summations in the first expression telescope.\footnote{$M_{j - 1, j}^{(j - 1)}$ is the model for $\Delta_{j - 1, j}$ at the beginning of iteration $j$. Due to the simplifying assumption $\zground^{(j)} = z^{(j)}$, the point $z^{(j)}$ is well-defined (and indeed, this is when it is set).} Excitingly, we show (see Lemma~\ref{lemma:compatibility}) it is possible to obtain the following bound
\begin{align}
    \label{eq:using-div-bound-overview}
    \sum_{j \ge 0} \innorm{\Delta_{j - 1, j}}_F^2 = \sum_{j \ge 0} \innorm{(A)_{z^{(j)}} - (A)_{z^{(j - 1)}} }_F^2 \le O \inparen*{\innorm{(A)_{z^{(0)}}}_F^2 + \sum_{j \ge 1} \breg{z^{(j - 1)}}{z^{(j)}}} \le \Otilde(1),
\end{align}
where the last inequality follows from a movement bound achieved by the prox point method (e.g., \cite[Lemma 4.2]{karmarkar2025solvingzerosumgames}). Plugging this into \eqref{eq:hypothetical-final-matvec-bound}, we obtain an $\Otilde(\tau^{-2} + \tau \epsilon^{-1})$ bound, yielding the target $\Otilde(\epsilon^{-2/3})$ complexity by setting $\tau \gets \epsilon^{1/3}$!

%
%
%

\paragraph{The binary tree representation.} Taking stock, the key advantage of the representation \eqref{eq:first-try-telescoping-representation} is it eliminated the necessity of the argument \eqref{eq:prev-paper-model-update-bound} of \cite{karmarkar2025solvingzerosumgames}. Recall that the bound \eqref{eq:prev-paper-model-update-bound} was needed for their analysis because their framework changes the basis (determined by $\zground^{(t)}$) in which it learns the matrix $A$ in every iteration. On the other hand, unlike $(A)_{\zground^{(t)}}$, the matrices $\Delta_{{j - 1}, j}$ are \emph{fixed} as soon as they are defined, and therefore we are always able to learn them in the same basis! This results in the tighter $\Otilde(1)$ bound in \eqref{eq:using-div-bound-overview} compared to the $\Otilde(\epsilon^{-1/3})$ bound in \eqref{eq:prev-paper-model-update-bound}, thereby enabling a $\Otilde(\epsilon^{-2/3})$ complexity. Moreover, unlike the argument \eqref{eq:prev-paper-model-update-bound} of \cite{karmarkar2025solvingzerosumgames}, 
%
this approach extends immediately to $\ellOneEllOne$ games as well.

That said, the above argument made two assumptions that break the $\Otilde(\epsilon^{-2/3})$ bound as soon as they are lifted.
%
%
%
%
%
%
%
%
%
To forgo these assumptions and still achieve the target complexity, we instead use a \emph{binary decomposition} to reduce the number of terms in the representation \eqref{eq:first-try-telescoping-representation} to $O(\log t)$.\footnote{We note that techniques involving binary decompositions are widespread throughout data structure/algorithm design (e.g., binary index trees) and optimization in particular \cite{bachoc2022nearoptimalalgorithmunivariatezerothorder,Axelrod2019NearoptimalAD}. Its use in the paper \cite{carmon2024whole} for a matrix-vector maintenance data structure perhaps most closely resembles our application, albeit the context is still quite different.} Recall that in the method given above, we maintain a sequence of models $M_{j - 1, j}$ where $0 \le j \le t$ at iteration $t$ of the outer loop. In particular, $M_{j - 1, j}$ is a model for the term $\Delta_{j - 1, j} \defeq (A)_{z^{(j)}} - (A)_{z^{(j - 1)}}$ in the representation \eqref{eq:first-try-telescoping-representation}. In our method described next, we instead maintain
%
 models $M_{j, j'}$ for all pairs $(j, j')$ in the index set 
%
\begin{align*}
         \biset^{(t)} \defeq \inbraces{(-1, 0)} \cup \inbraces{(2^k m, 2^k (m + 1) ) : k, m \in \Z_{\ge 0} \text{ s.t. } 2^k (m + 1) \le t}
\end{align*}
at iteration $t$ of the outer loop, where $M_{j, j'}$ is a model for $\Delta_{j, j'} \defeq (A)_{z^{(j')}} - (A)_{z^{(j)}}$. (Here, we overload notation and define $(A)_{z^{(-1)}} \defeq 0_{m \times n}$ for brevity.)
%

Note that $\biset^{(t)}$ is precisely the set of all pairs given by consecutive multiples of the same power of two, with the restriction that they are at most $t$, along with the added pair $(-1, 0)$. Thus, using the binary (or powers of two) decomposition of $t$, there exists $\bindecompset^{(t)} \subseteq \biset^{(t)}$ such that $(-1, 0) \in \bindecompset^{(t)}$, $|\bindecompset^{(t)}| = O(\log t)$, and we have the representation
\begin{align}
    \label{eq:bin-decomp-rep}
    (A)_{z^{(t)}} = \sum_{(j, j') \in \bindecompset^{(t)}} \Delta_{j, j'} = \sum_{(j, j') \in \bindecompset^{(t)}} \insquare{ (A)_{z^{(j')}} - (A)_{z^{(j)}} } \, .
\end{align}
For example, if $t = 13$, then $\bindecompset^{(t)} = \inbraces{(-1, 0), (0, 8), (8, 12), (12, 13)}$. Note that the pair $(-1, 0)$ corresponds to the first term in the telescoping sum $(A)_{z^{(0)}} - (A)_{z^{(-1)}} = (A)_{z^{(0)}}$.
%
%

%

Thus, we can proceed analogously to before, defining $M_{j, j'}^{(t - 1)}$ to be the state of the model $M_{j, j'}$ at the beginning of iteration $t$ of the outer loop for all $(j, j') \in \biset^{(t)}$. We refer to the representation and associated models $\inbraces{\Delta_{j, j'}, M_{j, j'}^{(t - 1)}}_{(j, j') \in \bindecompset^{(t)}}$ which we pass to $\SUG$ in this iteration as a \emph{matrix-approximation path to $z^{(t)}$} (Definition~\ref{def:matrix-approx-path}). Since $|\bindecompset^{(t)}| = O(\log t)$, the models $M_{j, j'}$ can be updated to $M_{j, j'}'$ for all $(j, j') \in \bindecompset^{(t)}$ via $M'_{j - 1, j} \gets M_{j - 1, j} + u^\top (\Delta_{j - 1, j} -  M_{j - 1, j}) v \cdot u v^\top$ using only $O(\log t)$ matvecs. (Models corresponding to index pairs not within $\bindecompset^{(t)}$, i.e., $(j, j') \in \biset^{(t)} \setminus \bindecompset^{(t)}$, retain their previous value.) In which case, a Cauchy-Schwarz argument (again using the fact that $|\bindecompset^{(t)}| = O(\log t)$) shows
\begin{align*}
   \sum_{(j, j') \in \bindecompset^{(t)}} \innorm{\Delta_{j, j'} -  M_{j, j'}}_F^2 -  \sum_{(j, j') \in \bindecompset^{(t)}} \innorm{\Delta_{j, j'} -  M'_{j, j'}}_F^2 \ge \Omega ( \tau^2 / \log t ),
\end{align*}
thereby only losing a log factor compared to \eqref{eq:hypothetical-guilty-bound}. 

Then, an analogous telescoping argument and choice of initialization for the models (namely, initializing models to $0_{m \times n}$ as before) gives a final matvec bound for solving \eqref{eq:intro-general-matrix-game} of 
\begin{align*}
    \Otilde\inparen*{    \tau^{-2}  \sum_{(j, j') \in \biset^{(T)}} \innorm{\Delta_{j, j'}}_F^2 +  \tau \epsilon^{-1} } ,
\end{align*}
where $T = \Otilde(\epsilon^{-2/3})$ is the final iteration count of the outer loop. As in \eqref{eq:using-div-bound-overview}, we can bound 
\begin{align}
    \sum_{(j, j') \in \biset^{(T)}} \innorm{\Delta_{j, j'}}_F^2 &= \sum_{(j, j') \in \biset^{(T)}} \innorm{
        (A)_{z^{(j')}} - (A)_{z^{(j)}}
    }_F^2  \nonumber \\
    &\le O \inparen*{\innorm{(A)_{z^{(0)}}}_F^2 + \sum_{(j, j') \in \biset^{(T)} \setminus \inbraces{-1, 0}} \breg{z^{(j)}}{z^{(j')}}}. \label{eq:final-bound-with-movement-term}
\end{align}
However, we now run into an obstacle; it is not clear how to show $\sum_{(j, j') \in \biset^{(T)} \setminus \inbraces{-1, 0}} \breg{z^{(j)}}{z^{(j')}} \le \Otilde(1)$ via the standard prox point method! To overcome this obstacle, we develop a new general primitive described next, termed the \emph{prox multi-point method}, which may be of independent interest.

\paragraph{Prox multi-point method.} Toward motivating and describing our new \emph{prox multi-point method} primitive, let us first briefly recap the standard movement bound between consecutive iterates achieved by the regular prox point method. In particular, the latter was used by \cite{karmarkar2025solvingzerosumgames} to prove the second inequality in \eqref{eq:prev-paper-model-update-bound}. 

Recall from Section~\ref{sec:prelims} that the problem of obtaining an $\epsilon$-solution of a matrix game is encompassed by the more general problem of achieving $\epsilon$-regret with respect to a monotone operator. We ultimately present our prox multi-point method in the latter, more general setting in Section~\ref{subsec:prox-multi-general-monotone-ops}, and thus we operate in the same general setting here. Formally, fix a dgf setup $\dgfsetup = (\zset, r)$ per Definition~\ref{def:dgf-setup} with $\Gamma_\dgfsetup \ge \max_{z, z' \in \zset} r(z) - r(z')$ and a monotone operator $g : \zset \to \R^d$. 

The standard prox point method starts with an initial point $z^{(0)} \in \zset$ and iterates $z^{(t)} \gets \prox_{z^{(t - 1)}}^{\alpha^{(t)}} (g)$ for $t = 1, 2, \dots, T$, where $\alpha^{(1)}, \alpha^{(2)}, \dots$ is a sequence of (positive) regularization parameters which can be chosen dynamically. By Definition~\ref{def:proximal-mappings}, this update is equivalent to $z^{(t)}$ satisfying
\begin{align*}
    \inangle{g(z^{(t)}), g(z^{(t)}) - u} \le \alpha^{(t)} \insquare{\breg{z^{(t - 1)}}{u} - \breg{z^{(t)}}{u} - \breg{z^{(t - 1)}}{z^{(t)}}} \text{ for all $u \in \zset$}.
\end{align*}
Multiplying both sides by $(\alpha^{(t)})^{-1} / S$ for $S \defeq \sum_{t \in [T]} (\alpha^{(t)})^{-1}$ and summing gives the standard regret guarantee
\begin{align*}
   \frac{1}{S} \sum_{t \in [T]} (\alpha^{(t)})^{-1} \inangle{g(z^{(t)}), g(z^{(t)}) - u} \le \frac{\breg{z^{(0)}}{u} - \sum_{t \in [T]} \breg{z^{(t - 1)}}{z^{(t)}}}{S} \text{ for all $u \in \zset$}.
\end{align*}
In $\ellOneEllOne$ and $\ellTwoEllOne$ matrix games, it is straightforward to pick $z^{(0)}$ such that $\sup_{u \in \zset} \breg{z^{(0)}}{u} = \Otilde(1)$, in which case we obtain the movement bound $\sum_{t \in [T]} \breg{z^{(t - 1)}}{z^{(t)}} \le \Otilde(1)$ since regret with respect to a monotone operator is nonnegative (e.g., \cite[Prop. A.1]{karmarkar2025solvingzerosumgames}).

Per \eqref{eq:final-bound-with-movement-term}, our goal is to extend this movement bound between consecutive iterates to a movement bound over all pairs of iterates $z^{(j)}$ and $z^{(j')}$ for $(j, j') \in \biset^{T} \setminus \inbraces{-1, 0}$. We achieve this goal by carefully adding additional regularization to iterations $t$ of the standard prox point method about iterates \emph{before} the previous iterate $z^{(t - 1)}$. Conceptually, this serves to increase control over gapped pairs of iterates.

Formally, our \emph{prox multi-point method} (Algorithm~\ref{alg:multiprox-method-monotone-op}) maintains $K$ sequences of regularization center points, where the $k$-th sequence for $k \in [K]$ is denoted $w_k^{(0)}, w_k^{(1)}, \dots, w_k^{(T)}$, as well as a sequence of iterates $z^{(0)}, z^{(1)}, \dots, z^{(T)}$. We also maintain a sequence of \emph{active center index sets} $\idset^{(1)}, \idset^{(2)}, \dots, \idset^{(T)}$ where each $\idset^{(t)} \subseteq [K]$. These encode which regularization centers are active at each iteration $t$, and can be chosen dynamically (though this is not strictly necessary for our application to matrix games). We initialize $w_k^{(0)} \gets z^{(0)}$ for all $k \in [K]$. 

Then, at every iteration $t = 1, 2, \dots, T$ of the prox multi-point method, we let $\uset^{(t)} \defeq \inbraces{w_k^{(t - 1)} : k \in \idset^{(t)}}$ denote the multiset containing the centers which are active at that step, and update $z^{(t)} \gets \prox_{\uset^{(t)}}^{\alpha^{(t)}}(g ; \zset)$,\footnote{In Section~\ref{subsec:prox-multi-general-monotone-ops} we allow for an approximate solution, but we use exact solutions here for simplicity.} namely,
\begin{align}
    \label{eq:overview-prox-multi-point-update}
   (\alpha^{(t)})^{-1} \inangle{g(z^{(t)}), g(z^{(t)}) - u} \le  \sum_{k \in \idset^{(t)}} \insquare{\breg{w_k^{(t - 1)}}{u} - \breg{z^{(t)}}{u} - \breg{w_k^{(t - 1)}}{z^{(t)}}} \text{ for all $u \in \zset$} .
\end{align}
We then update the regularization centers via
\begin{align}
    \label{eq:w-update-in-overview}
    w^{(t)}_k \gets \begin{cases} 
      z^{(t)},  & \text{for all } k \in \idset^{(t)}, \\
      w^{(t - 1)}_k,  & \text{for all } k \in [K] \setminus \idset^{(t)}.
   \end{cases}
\end{align}
Summing \eqref{eq:overview-prox-multi-point-update} and dividing by $S$, we obtain for all $u \in \zset$:
\begin{align*}
    \frac{1}{S} \sum_{t \in [T]} (\alpha^{(t)})^{-1} \inangle{g(z^{(t)}), g(z^{(t)}) - u} &\le \frac{1}{S} \sum_{t \in [T]} \, \,\sum_{k \in \idset^{(t)}}  \insquare{\breg{w_k^{(t - 1)}}{u} - \breg{z^{(t)}}{u} - \breg{w_k^{(t - 1)}}{z^{(t)}}} \\
    &= \frac{1}{S} \sum_{t \in [T]} \sum_{k \in [K]}  \insquare{\breg{w_k^{(t - 1)}}{u} - \breg{w^{(t)}}{u} - \breg{w_k^{(t - 1)}}{w^{(t)}}} \\
    &= \frac{1}{S}  \sum_{k \in [K]} \sum_{t \in [T]} \insquare{\breg{w_k^{(t - 1)}}{u} - \breg{w^{(t)}}{u} - \breg{w_k^{(t - 1)}}{w^{(t)}}} \\
    &\le \frac{1}{S}  \insquare*{ \sum_{k \in [K]} \breg{w_k^{(0)}}{u}  -   \sum_{k \in [K]} \sum_{t \in [T]}  \breg{w_k^{(t - 1)}}{w^{(t)}}           }.
\end{align*}
Therefore, in the matrix games application where we can ensure $\sup_{u \in \zset} \breg{w_k^{(0)}}{u} \le \Otilde(1)$ for all $k \in [K]$ simultaneously, we obtain the movement bound $\sum_{k \in [K]} \sum_{t \in [T]}  \breg{w_k^{(t - 1)}}{w^{(t)}} \le \Otilde(K)$. Thus, the prox multi-point method can be viewed as enabling control over $K$ subsequences of iterates at the cost of a $K$ factor in the movement bound. 

In our application to matrix games where we aim to control $\sum_{(j, j') \in \biset^{(T)} \setminus \inbraces{-1, 0}} \breg{z^{(j)}}{z^{(j')}}$, choosing $\idset^{(t)} \gets \inbraces{k \in [K] : \text{ $t$ is divisible by $2^{k - 1}$ }}$ for all $t \in [T]$ and $K = \Theta (\log T)$ enables us to obtain a sufficiently tight bound
\begin{align}
    \label{eq:overview-multiprox-movement-bound}
    \sum_{(j, j') \in \biset^{(T)} \setminus \inbraces{-1, 0}} \breg{z^{(j)}}{z^{(j')}} =  \sum_{k \in [K]} \sum_{t \in [T]}  \breg{w_k^{(t - 1)}}{w^{(t)}}  \le \Otilde(\log T).
\end{align}
Indeed, note that for any $k \in [K]$, the update \eqref{eq:w-update-in-overview} along with the choice of $\idset^{(t)}$ implies $w_{k}^{(t)} = z^{(b_{k, t})}$ where $b_{k, t} \defeq t - (t \bmod 2^{k - 1})$; namely, $b_{k, t}$ is the largest multiple of $2^{k - 1}$ which is at most $t$. Thus, 
\begin{align*}
    \sum_{t \in [T]}  \breg{w_k^{(t - 1)}}{w^{(t)}} = \sum_{m \ge 0} \breg{z^{(m \cdot 2^{k - 1})}}{z^{((m + 1) \cdot 2^{k - 1})}} \, . 
\end{align*}
Namely, the $k$-th inner summation in the second term in \eqref{eq:overview-multiprox-movement-bound} bounds the divergences between iterates corresponding to consecutive powers of $2^{k - 1}$.


\paragraph{The dyadic prox method.} In our formal instantiation of the above in Section~\ref{sec:combined-outer-loop-section}, we do not actually maintain models $M_{j, j'}$ for all pairs $(j, j')$ in $\biset^{(t)}$ at the $t$-th iteration of the outer loop. Instead, we only maintain models for pairs $(j, j')$ which appear in the binary decomposition of $t$.\footnote{As discussed momentarily, we actually maintain models for pairs $(j, j')$ which appear in the binary decomposition of $t - 1$ as opposed to $t$ due to subtleties related to the handling of $\zground$ (namely, undoing the assumption $z^{(t)} = \zground^{(t)}$ made for simplicity in this overview).} For example, when $t = 8$ and we use the representation $(A)_{z^{(t)}} = \Delta_{-1, 0} + \Delta_{0, 8}$ (recall $\Delta_{j, j'} \defeq (A)_{z^{(j')}} - (A)_{z^{(j)}}$ where we overload notation and define $(A)_{z^{(-1)}} \defeq 0_{m \times n}$ for brevity), note that, e.g., $(6, 8)$ and $(7, 8)$ are both in $\biset^{(8)}$. However, it is not necessary to maintain models $M_{6, 8}$ or $M_{7, 8}$ for $\Delta_{6, 8}$ and $\Delta_{7, 8}$ respectively since the latter do not appear in the binary decomposition of 8. In fact, $\Delta_{6, 8}$ and $\Delta_{7, 8}$ do not appear in the binary decomposition of any natural number. Ultimately, as discussed further in Section~\ref{subsec:prox-multi-matrix-games}, it is possible to maintain only $K$ models at each step of the outer loop. We also ultimately choose $\idset^{(t)} \gets \inbraces{k \in [K] : \text{ $t$ is divisible by $2^{K - k}$ }}$ as opposed to the choice $\idset^{(t)} \gets \inbraces{k \in [K] : \text{ $t$ is divisible by $2^{k - 1}$ }}$ made above for aesthetic reasons discussed further in Section~\ref{subsec:prox-multi-matrix-games}. Note that this corresponds to simply relabeling the sequences $w_k^{(0)}, w_k^{(1)}, \dots, w_k^{(T)}$ (e.g., what was previously $w_1^{(0)}, \dots, w_1^{(T)}$ is now $w_K^{(0)}, \dots, w_K^{(T)}$). We call the prox multi-point method the \emph{dyadic prox method} (Definition~\ref{def:dyadic-prox-multi-point-method}) with the specific choice $\idset^{(t)} \gets \inbraces{k \in [K] : \text{ $t$ is divisible by $2^{K - k}$ }}$.



%


%


%

%




%




%


%


\paragraph{Lifting the $z^{(t)} = \zground^{(t)}$ assumption.} As a final point before concluding, recall that we made a simplifying assumption $z^{(t)} = \zground^{(t)}$ in the above overview. In actuality, when our current center point is $\zground$ at some point during the $t$-th iteration of the outer loop, we must build a representation for $(A)_{\zground}$ (since this is the matrix which model-update steps find a large component of). This is achieved by instead representing $(A)_{z^{(t - 1)}}$ (as opposed to $(A)_{z^{(t)}}$) using a binary decomposition of the form \eqref{eq:bin-decomp-rep}, and then adding on a final ``head'' term to the telescoping sum given by $(A)_{\zground} - A_{z^{(t - 1)}}$. This may seem problematic at first glance, as unlike other terms in \eqref{eq:bin-decomp-rep} of the form $(A)_{z^{(j')}} - (A)_{z^{(j)}}$ for $j, j' \le t - 1$, the head term $(A)_{\zground} - A_{z^{(t - 1)}}$ is not fixed; it will change when we update $\zground$ (e.g., at the next step of the bisection search). However, we show we can afford the additional matvecs performed within model-update iterations of $\SUG$ due to the model for $(A)_{\zground} - A_{z^{(t - 1)}}$ being reset to $0_{m \times n}$ whenever $\zground$ changes. This is achieved via an amortized analysis since the sum of the terms $\innorm{(A)_{\zground} - A_{z^{(t - 1)}}}_F^2$ can be bounded using the movement guarantee of the prox multi-point method; see Section~\ref{sec:putting-together} for details. 


\paragraph{Conclusion.} To conclude the overview, our algorithmic framework for proving Theorems \ref{thm:final-result-l1-l1-aka-zero-sum} and \ref{thm:final-result-l2-l1-aka-SVM} consist of an outer loop given by an instantiation of our prox multi-point method (Section~\ref{sec:combined-outer-loop-section}), a bisection search procedure to determine an appropriate amount of regularization $\alpha^{(t)}$ at each step (Section~\ref{sec:MDMP-implementation}), and the subproblem solver $\SUG$ (Section~\ref{sec:sug-solver}), which can be viewed as an appropriately modified version of $\SUGStronglyMonotoneMirrorProx$ \cite[Alg. 5.2]{karmarkar2025solvingzerosumgames} to handle updating all models in the matrix-approximation path. We emphasize that both our bisection search procedure and subproblem solver $\SUG$ differ compared to the corresponding procedures in \cite{karmarkar2025solvingzerosumgames} due to the specific requirements of our prox multi-point method outer loop; albeit, these changes are perhaps more straightforward. We hope that the technical tools we introduce, particularly our multi-point/dyadic prox methods and accompanying tools for controlling movement over sequences of variational inequalities, may find broader use.

%




%

%

%


%


%




%


%


%


%



%
%
%
%
%




%

%





%





%




%

%














%

%

%

%


%



%


%






%


%


%



%


%



%




%


%


%







%











%



%


%


%


%
%





%

%
%
%
%
%
%

%
%
%
%
%
%
%
%

%
%
%
%
%



%
%

%

%

%

%
%





%
%
%
%
%
%
%

%



%

%
%
%
%
%
%

%
%
%
%
%
%






















%

%

 
%


\section{Prox multi-point method outer loop}
\label{sec:combined-outer-loop-section}

In this section we provide and analyze our prox multi-point method. Section~\ref{subsec:prox-multi-general-monotone-ops} provides our general prox multi-point method for general monotone operators. In the context of matrix games, \Cref{subsec:prox-multi-general-monotone-ops} can be viewed as proving correctness along with iteration bound and iterate movement bounds for our outer loop. \Cref{subsec:prox-multi-matrix-games} then specializes the prox multi-point method to matrix games, thereby forming the outer loop of our ultimate algorithm for proving Theorems \ref{thm:final-result-l1-l1-aka-zero-sum} and \ref{thm:final-result-l2-l1-aka-SVM}. In particular, Section~\ref{subsec:prox-multi-matrix-games} handles model creation and clearing, as well as passing models to our bisection search procedure (discussed in further detail in Section~\ref{sec:MDMP-implementation}) and through it to our inner loop subproblem solver (discussed in further detail in Section~\ref{sec:sug-solver}). By carefully applying the movement bounds between iterates given in Section~\ref{subsec:prox-multi-general-monotone-ops}, Section~\ref{subsec:prox-multi-matrix-games} forms the backbone of the amortized analysis discussed in Section~\ref{sec:overview-of-approach} (and formally computed in Section~\ref{sec:putting-together})
%
 to bound the total number of matvecs made over all inner loop model-update steps.

%



%

\subsection{Prox multi-point method for general monotone operators}
\label{subsec:prox-multi-general-monotone-ops}

%

In this section, we provide our new \emph{prox multi-point method} (Algorithm~\ref{alg:multiprox-method-monotone-op}) as well as a specialization termed the \emph{dyadic prox method} (Definition~\ref{def:dyadic-prox-multi-point-method}). For $\epsilon > 0$, the prox multi-point method obtains $\epsilon$-regret with respect to a general monotone operator $g : \zset \to \R^d$; namely, it obtain sequences $z^{(1)}, \dots, z^{(T)} \in \zset$ and $\lambda^{(1)}, \dots, \lambda^{(T)} > 0$ such that $\regret_g(\inbraces{z^{(t)}, \lambda^{(t)}}_{t \in [T]}) \le \epsilon$ (recall Definition~\ref{def:regret}). In the next section (Section~\ref{subsec:prox-multi-matrix-games}), we  use this regret bound to bound the gap in our matrix games applications via Lemma~\ref{lem:regret-bounds-the-gap}.

%

%



\begin{assumptions}
In this section (Section~\ref{subsec:prox-multi-general-monotone-ops}), we fix a dgf setup $\dgfsetup = (\zset, r)$ per Definition~\ref{def:dgf-setup} with $\Gamma_\dgfsetup \ge \max_{z, z' \in \zset} r(z) - r(z')$ and a monotone operator $g : \zset \to \R^d$.
\end{assumptions}


First, in the following Definition~\ref{def:DMP} we define a key oracle to which our algorithm will assume access. A $\DMP$ oracle approximately solves a strongly monotone variational inequality \eqref{eq:DMP-def-variational-property} with respect to the operator $g + \alpha \grad \breg{\uset}{\cdot}$; in particular, it approximates $\prox_{\cU}^\alpha(g ; \zset)$. (Note that $\grad \breg{\uset}{z'} = \sum_{w \in \uset} \grad \breg{w}{z'}$ for all $z' \in \zset$ by linearity of the gradient, recalling the notation of Section~\ref{sec:prelims}.)
%
Additionally, we say a $\DMP$ oracle is \emph{kinetic} if it either uses a default level of regularization $\beta$, or else certifies progress by lower bounding the movement of the output. Definition~\ref{def:DMP} can be viewed as the natural extension of \cite[Def. 4.1]{karmarkar2025solvingzerosumgames} to regularization about multiple points. 

%

%





\begin{definition}[$\epsilon'$-$\DMP$] \label{def:DMP}
    For $\epsprim > 0$, we call $\ODMP(\cdot)$ an \emph{$\epsprim$-dynamic multiprox oracle} or $\epsprim$-$\DMP$ (with respect to the operator $g$ and setup $\dgfsetup$) if given a finite, nonempty multiset $\uset \subset \zset$ as input, it returns $(z' \in \zset, \alpha > 0)$
    %
    such that
\begin{align}
    \label{eq:DMP-def-variational-property}
    \inangle{g(z') + \alpha \grad \breg{\uset}{z'}, z' - u} \le \epsilon' \text{~for all $u \in \zset$}.
\end{align}
For $\beta, \gamma, \rho > 0$, we say an $\epsilon'$-$\DMP$ oracle is \emph{$(\beta, \gamma, \rho)$-kinetic} if additionally the output always satisfies at least one of 
\begin{enumerate*}[series = tobecont, itemjoin =, label=(\alph*)]
    \item $\alpha = \beta$ 
    %
    or \label{item:DMP-beta-cond}
    \item $\breg{\cU}{z'} \ge \gamma \alpha^\rho$. \label{item:DMP-movement-cond}
\end{enumerate*}
\end{definition}


%



Next, we present the prox multi-point method in Algorithm~\ref{alg:multiprox-method-monotone-op}. Algorithm~\ref{alg:multiprox-method-monotone-op} maintains a sequence of iterates $z^{(1)}, z^{(2)}, \dots$ outputted by the $\DMP$ oracle, as well as $K$ sequences of regularization center points, where the $k$-th sequence for $k \in [K]$ is denoted $w_k^{(0)}, w_k^{(1)}, \dots$. At each iteration $t$, a set of \emph{active centers} is dynamically chosen via the index set $\idset^{(t)}$ in Line~\ref{line:multiprox-choose-active-centers}. The corresponding centers $\inbraces{w^{(t - 1)}_k : k \in \idset^{(t)}}$ are passed to the $\DMP$ oracle in Line~\ref{line:multiprox-DMP} to obtain $z^{(t)}$ and $\alpha^{(t)}$. Then, the regularization centers for the next iteration $w_k^{(t)}$ are set in Line~\ref{line:multiprox-w^{(t)}_k-unified-def}. Centers which were active in the current iteration are updated to $z^{(t)}$; otherwise they retain their previous value. This ensures appropriate telescoping occurs in the analysis. 
%
Finally, note that the conditional in Line~\ref{line:multiprox-while-loop} evaluates to $\true$ when $t = 0$ due to the convention from Section~\ref{sec:prelims} that a summation over an empty index set is 0.


\RestyleAlgo{ruled}
\DontPrintSemicolon
\SetKwComment{Comment}{/* }{ */}
\begin{algorithm2e}[h!]
\caption{Prox multi-point method}
\label{alg:multiprox-method-monotone-op}
\KwInput{Precision $\epsilon > 0$, max centers per step $K \in \Z_{> 0}$, 
%
 $\epsilon$-$\DMP$ oracle $\ODMP$
}

$z^{(0)} \gets \argmin_{z \in \zset} r(z)$ ~and~ $t \gets 0$\;

$w^{(0)}_k \gets z^{(0)}$ for all $k \in [K]$ \;

\While(\tcp*[f]{Recall $\Range_\dgfsetup \ge \max_{z, z' \in \zset} r(z) - r(z')$}){
    $\sum_{j \in [t]} (\alpha^{(j)})^{-1} < K \Range_\dgfsetup \epsilon^{-1}$ 
}{ \label{line:multiprox-while-loop}

    $t \gets t + 1$ \;

    Choose $\idset^{(t)} \subseteq [K]$ \tcp*{Dynamically select the active center indices at the $t$-th step} \label{line:multiprox-choose-active-centers}

    $(z^{(t)}, \alpha^{(t)}) \gets \ODMP(\inbraces{w^{(t - 1)}_k : k \in \idset^{(t)}})$ \label{line:multiprox-DMP}

    %

    %

    $w^{(t)}_k \gets \begin{cases} 
      z^{(t)},  & \text{for all } k \in \idset^{(t)} \\
      w^{(t - 1)}_k,  & \text{for all } k \in [K] \setminus \idset^{(t)}
   \end{cases}$ \label{line:multiprox-w^{(t)}_k-unified-def}

}

\Return{$\inbraces{z^{(j)}, \alpha^{(j)}}_{j \in [T]}$ where $T \defeq t$ \tcp*{$T$ is used in the analysis to refer to the final iteration count}
}\label{line:multiprox-return}

\end{algorithm2e}

Next, we give our correctness guarantee as well as a movement bound over the sequences of regularization centers in Lemma~\ref{lem:multiprox-correctness}. Recall from the discussion in Section~\ref{sec:overview-of-approach} that this movement bound can be viewed as enabling control over $K$ subsequences of the iterates $z^{(0)}, z^{(1)}, \dots$ as opposed to the movement bound over the single sequence $z^{(0)}, z^{(1)}, \dots$ obtained by the standard prox point method, albeit at the cost of an additional $K$ factor in the bound.
%
In Lemma~\ref{lem:multiprox-correctness} and throughout the remainder of this section (Section~\ref{subsec:prox-multi-general-monotone-ops}), we use the notation $\sum_{t \ge 1}$ to simultaneously capture the cases where Algorithm~\ref{alg:multiprox-method-monotone-op} does and doesn't terminate. $\sum_{t \ge 1}$ is equivalent to $\sum_{t \in [T]}$ in the former case and $\sum_{t = 1}^\infty$ in the latter case.



%

\begin{lemma}
    \label{lem:multiprox-correctness}
    The iterates of Algorithm~\ref{alg:multiprox-method-monotone-op} satisfy $\sum_{t \ge 1} \sum_{k \in [K]} \breg{w^{(t - 1)}_k}{w^{(t)}_k} \le K \Range_\dgfsetup$, and if Algorithm~\ref{alg:multiprox-method-monotone-op} terminates, then $\regret_g(\inbraces{z^{(t)}, (\alpha^{(t)})^{-1}}_{t \in [T]})  \le 2 \epsilon$.
\end{lemma}

\begin{proof}
The definition of $z^{(t)}$ in \Cref{line:multiprox-DMP}, along with \Cref{def:DMP} and \eqref{eq:Bregman-three-point-equality}, imply that for all $t \ge 1$ and $u \in \zset$:
\begin{align*}
    (\alpha^{(t)})^{-1} \inangle{g(z^{(t)}), z^{(t)} - u} &\le \sum_{k \in \idset^{(t)}} [\breg{w_k^{(t - 1)}}{u} - \breg{z^{(t)}}{u} - \breg{w_k^{(t - 1)}}{z^{(t)}}] + (\alpha^{(t)})^{-1} \epsilon \\
    &= \sum_{k \in [K]} [\breg{w_k^{(t - 1)}}{u} - \breg{w_k^{(t)}}{u} - \breg{w_k^{(t - 1)}}{w_k^{(t)}}] + ( \alpha^{(t)})^{-1} \epsilon,
\end{align*}
where the equality followed from the definition of $w_k^{(t)}$ in Line~\ref{line:multiprox-w^{(t)}_k-unified-def}; note in particular 
%
%
%
%
that the expression within the final summation is zero for $k \in [K] \setminus \idset^{(t)}$. Letting $t' \ge 1$ denote some value the counter $t$ takes during Algorithm~\ref{alg:multiprox-method-monotone-op}, summing both sides over $t \in [t']$, dividing by $S_{t'} \defeq \sum_{t \in [t']} (\alpha^{(t)})^{-1}$ (defined as a function of $t'$), and using the nonnegativity of Bregman divergences yields
\begin{align*}
    \frac{ K \Gamma_\dgfsetup - \sum_{t \in [t']} \sum_{k \in [K]}  \breg{w_k^{(t - 1)}}{w_k^{(t)}} }{S_{t'}} + \epsilon \ge
        \sup_{u \in \zset} \inbraces*{\frac{1}{S_{t'}} \sum_{t \in [t']} (\alpha^{(t)})^{-1} \inangle{ g(z^{(t)}), z^{(t)} - u }} \overge{(i)} 0,
\end{align*}
where $(i)$ follows since regret with respect to a monotone operator is nonnegative (e.g., \cite[Proposition A.1]{karmarkar2025solvingzerosumgames}). Then the first claim is immediate as $t'$ was arbitrary. The second claim follows by instantiating $t' \gets T$ in the above display and noting $S_T \ge K \Gamma_\dgfsetup \epsilon^{-1}$ due to the termination condition in Line~\ref{line:multiprox-while-loop}.
\end{proof}

Next, when the $\DMP$ is kinetic, we bound the number of iterations $T$ as well as the sum of the regularization levels $\alpha^{(t)}$ raised to the $\rho$ power. The latter will ultimately be used to handle the point discussed at the end of Section~\ref{sec:overview-of-approach}---needing to bound the sum of the $\innorm{(A)_{\zground} - A_{z^{(t - 1)}}}_F^2$ terms (see Section~\ref{sec:putting-together}).

\begin{lemma}
    \label{lem:multiprox-iteration-bound}
If the $\epsilon$-$\DMP$ oracle given as input to Algorithm~\ref{alg:multiprox-method-monotone-op} is $(\beta, \gamma, \rho)$-kinetic, then the algorithm terminates with
\begin{align*}
   T \leq K \Gamma_\dgfsetup (\beta \epsilon^{-1} + \gamma^{- \frac{1}{\rho+1}} \epsilon^{- \frac{\rho}{\rho + 1}}) + 2 ~~\text{and}~~  \sum_{t \in [T]} (\alpha^{(t)})^\rho \le K \Gamma_{\dgfsetup} \gamma^{-1} + T \beta^\rho.
\end{align*}
%
\end{lemma}



\begin{proof}
Let $J_a \defeq \inbraces{t \ge 1 : \alpha^{(t)} = \beta}$ and $J_b \defeq \inbraces{t \ge 1 : \alpha^{(t)} \ne \beta}$. (In defining these sets, we restrict to $t$ such that $\alpha^{(t)}$ is well-defined; namely, $t \in [T]$ if Algorithm~\ref{alg:multiprox-method-monotone-op} terminates and $t \in \Z_{> 0}$ otherwise. In particular, note that the $\DMP$ oracle call in Line~\ref{line:multiprox-DMP} during an iteration $t$ such that $t \in J_b$ must satisfy condition \ref{item:DMP-movement-cond} in Definition~\ref{def:DMP}.) Then $|J_a| \le \beta K \Gamma_\dgfsetup \epsilon^{-1} + 1$ due to the termination condition in Line~\ref{line:multiprox-while-loop}.
%
%
%
%
As for $J_b$, note
%
\begin{align}
    \label{eq:bound-rho-alpha-c-sum}
    \begin{split}
    \sum_{t \in J_b} \gamma (\alpha^{(t)})^\rho \le \sum_{t \in J_b}  \sum_{k \in \idset^{(t)}} \breg{w_k^{(t - 1)}}{z^{(t)}}
    \overeq{(i)} \sum_{t \in J_b'}  \sum_{k \in [K]} \breg{w_k^{(t - 1)}}{w_k^{(t)}} &\overle{(ii)} \sum_{t \ge 1}  \sum_{k \in [K]} \breg{w_k^{(t - 1)}}{w^{(t)}} \\
    &\overle{(iii)} K \Gamma_\dgfsetup,
    \end{split}
\end{align}
where we used $(i)$ the definition of $w_k^{(t)}$ in Line~\ref{line:multiprox-w^{(t)}_k-unified-def},
%
$(ii)$ the nonnegativity of Bregman divergences, and $(iii)$ Lemma~\ref{lem:multiprox-correctness}. Thus, $|J_b|$ is finite and Algorithm~\ref{alg:multiprox-method-monotone-op} terminates, as otherwise \eqref{eq:bound-rho-alpha-c-sum} implies $\lim_{t \to \infty, \, t \in J_b} \alpha^{(t)} = 0$, a contradiction due to the termination condition in Line~\ref{line:multiprox-while-loop}. Next, toward bounding $|J_b|$, let $J_b' \defeq J_b \setminus \inbraces{T}$ and note
%
\begin{align*}
    |J_b'| = \sum_{t \in J_b'} (\alpha^{(t)})^{\frac{\rho}{\rho + 1}} (\alpha^{(t)})^{- \frac{\rho}{\rho + 1}} 
    &\overle{(i)} 
    \inparen*{
        \sum_{t \in J_b'} (\alpha^{(t)})^\rho
    }^{\frac{1}{\rho + 1}}
     \inparen*{
        \sum_{t \in J_b'} (\alpha^{(t)})^{-1} 
     }^{\frac{\rho}{\rho + 1}} \\
     &=   \gamma^{- \frac{1}{\rho+1}}  \inparen*{
        \sum_{t \in J_b'} \gamma (\alpha^{(t)})^\rho
    }^{\frac{1}{\rho + 1}}
     \inparen*{
        \sum_{t \in J_b'} (\alpha^{(t)})^{-1} 
     }^{\frac{\rho}{\rho + 1}}
     \\
     &\overle{(ii)} \gamma^{- \frac{1}{\rho+1}} (K \Gamma_\dgfsetup)^{\frac{1}{\rho + 1}} (K \Gamma_\dgfsetup \epsilon^{-1})^{\frac{\rho}{\rho + 1}} \\
     &= \gamma^{- \frac{1}{\rho+1}} K \Gamma_\dgfsetup  \epsilon^{- \frac{\rho}{\rho + 1}},
\end{align*}
by $(i)$ \Holder's inequality and $(ii)$ \eqref{eq:bound-rho-alpha-c-sum} as well as the fact that $\sum_{t \in J_b'} (\alpha^{(t)})^{-1} \le \sum_{t \in [T - 1]} (\alpha^{(t)})^{-1} < K \Gamma_\dgfsetup \epsilon^{-1}$ by the termination condition in Line~\ref{line:multiprox-while-loop}. To obtain the 
%
desired upper bound on $T$, note
\begin{align*}
    T = |J_a| + |J_b| \le |J_a| + |J_b'| + 1 \le K \Gamma_\dgfsetup (\beta \epsilon^{-1} + \gamma^{- \frac{1}{\rho+1}} \epsilon^{- \frac{\rho}{\rho + 1}}) + 2.
\end{align*}
As for the bound on the sum of $(\alpha^{(t)})^\rho$, note
\begin{align*}
    \sum_{t \in [T]} (\alpha^{(t)})^\rho = \sum_{t \in J_a} (\alpha^{(t)})^\rho + \sum_{t \in J_b} (\alpha^{(t)})^\rho \le K \Gamma_{\dgfsetup} \gamma^{-1} + T \beta^\rho
\end{align*}
by the definition of $J_a$ as well as \eqref{eq:bound-rho-alpha-c-sum}.
%
%
%
%
%
%
%
%
%
%
%
%
%
%
%
%
%
%
%
%
%
%
%
%
%
%
%
%
%
%
%
%
\end{proof}



%

Next, we formally define the \emph{dyadic prox method} mentioned in Section~\ref{sec:overview-of-approach}. The dyadic prox method fixes the choice $\idset^{(t)} \gets \inbraces{k \in [K] : \text{$t$ is divisible by $2^{K - k}$}}$, which results in the movement bound in Lemma~\ref{lem:multiprox-correctness} controlling pairs of iterates $z^{(j)}$ and $z^{(j')}$ where $j' - j$ is a power of two; recall \eqref{eq:overview-multiprox-movement-bound} from Section~\ref{sec:overview-of-approach} and the surrounding discussion. We note that whenever we use Definition~\ref{def:dyadic-prox-multi-point-method} in this paper, we also ensure (either by assumption or by making an explicit choice of $K$) that $K \ge \log_2 T + 5$ (recall $T$ is the final iteration count per Line~\ref{line:multiprox-return}); this ensure that we are indeed controlling every such pair up to $T$. 

%


\begin{definition}[Dyadic prox method]
    \label{def:dyadic-prox-multi-point-method}
We refer to Algorithm~\ref{alg:multiprox-method-monotone-op} with the choice $\idset^{(t)} \gets \inbraces{k \in [K] : \text{$t$ is divisible by $2^{K - k}$}}$ in Line~\ref{line:multiprox-choose-active-centers} for all $t \ge 1$ as the \emph{dyadic prox method}.
\end{definition}

%
%
%
%

In the following lemma, we show that for the specific case of the dyadic prox method, we can obtain a certain alternate movement bound to that given in Lemma~\ref{lem:multiprox-correctness}.
%
%
In particular, recall from above that the middle term in \eqref{eq:dyadic-prox-multi-point-method-movement-bound} bounds the movement between all pairs of iterates $z^{(j)}$ and $z^{(j')}$ where $j$ and $j'$ are gapped by a power of two (as long as $K$ is sufficiently large). However, as discussed at the end of Section~\ref{sec:overview-of-approach}, it is not necessary to create models corresponding to all such pairs, but rather only those which actually appear in the binary decompositions of the natural numbers up to $T$. This is precisely what the leftmost term in \eqref{eq:dyadic-prox-multi-point-method-movement-bound} corresponds to (and it is why \eqref{eq:dyadic-prox-multi-point-method-movement-bound} is an inequality since the pairs $j, j'$ which appear in binary decompositions are a strict subset of all pairs gapped by a power of two). We use Lemma~\ref{lem:dyadic-prox-movement-bound} in the next section (Section~\ref{subsec:prox-multi-matrix-games}) and defer to it for further discussion.

%

%


%

%


%

\begin{lemma}
    \label{lem:dyadic-prox-movement-bound}
    Supposing $K \ge \log_2 T + 5$ (where $T$ is the final iteration count per Line~\ref{line:multiprox-return}), the iterates of the dyadic prox method (Definition~\ref{def:dyadic-prox-multi-point-method}) satisfy
    %
\begin{align}
    \label{eq:dyadic-prox-multi-point-method-movement-bound}
  \sum_{t \in [T]} \sum_{k \in \idset^{(t)}} \breg{w^{(t)}_{k - 1}}{w^{(t)}_{k}} \le  \sum_{t \in [T]} \sum_{k \in [K]} \breg{w^{(t - 1)}_k}{w^{(t)}_k} \le K \Range_\dgfsetup .
\end{align}
Furthermore, if $(w_{k - 1}^{(t - 1)}, w_{k}^{(t - 1)}) \ne (w_{k - 1}^{(t)}, w_{k}^{(t)})$ for some $t \ge 1$ and $2 \le k \le K$, then $k \in \idset^{(t)}$.
%
%
%
%
%
\end{lemma}





\begin{proof}
Note that for any $t \in [T]_0$ and $k \in [K]$, we have that $w_{k}^{(t)} = z^{(a_{k, t})}$ where $a_{k, t} \defeq t - (t \bmod 2^{K - k})$; namely, $a_{k, t}$ is the largest multiple of $2^{K - k}$ which is at most $t$. Indeed, this follows from the fact that by the choice of $\idset^{(t)}$ in Definition~\ref{def:dyadic-prox-multi-point-method}, the subsequence of $\idset^{(1)}, \idset^{(2)}, \idset^{(3)}, \dots$ such that each set in the subsequence contains $k$ is precisely $\idset^{2^{K - k}}, \idset^{2(2^{K - k})}, \idset^{3(2^{K - k})}, \dots$. Hence, by Line~\ref{line:multiprox-w^{(t)}_k-unified-def} of Algorithm~\ref{alg:multiprox-method-monotone-op}, $w_{k}^{(t)}$ is updated to $z^{(t)}$ in iterations $t$ such that $t$ is a multiple of $2^{K - k}$, and otherwise retains its previous value $w_{k}^{(t - 1)}$.

Let us now examine the leftmost summation in \eqref{eq:dyadic-prox-multi-point-method-movement-bound}; in particular, fix an arbitrary $t \in [T]$ and consider $\sum_{k \in \idset^{(t)}} \breg{w^{(t)}_{k - 1}}{w^{(t)}_{k}}$. Note that the latter is well-defined since $K \ge \log_2 T + 5$ implies $1 \notin \idset^{(t)}$. (In other words, the fact that $w_0^{(t)}$ is not defined does not pose an issue.) Then, letting $\ktstar \defeq \min_{k \in \idset^{(t)}} k$ (note $\idset^{(t)}$ is nonempty since $K \in \idset^{(t)}$), we claim
\begin{align}
    \label{eq:evaluating-the-idset-sum}
    \sum_{k \in \idset^{(t)}} \breg{w^{(t)}_{k - 1}}{w^{(t)}_{k}} = \breg{w^{(t)}_{\ktstar - 1}}{w^{(t)}_{\ktstar}}.
        %
\end{align}
This follows since for any $k \in [K - 1]$, we have that $k \in \idset^{(t)}$ implies $k + 1 \in \idset^{(t)}$ by the choice of $\idset^{(t)}$ (if $t$ is divisible by $2^{K - k}$ then it is also divisible by $2^{K - k - 1}$). Thus, $w_{k - 1}^{(t)} = w_{k}^{(t)} = z^{(t)}$ for all $\ktstar + 1 \le k \le K$ by Line~\ref{line:multiprox-w^{(t)}_k-unified-def} of Algorithm~\ref{alg:final-algo-outer-loop}. 
%

Then to prove \eqref{eq:dyadic-prox-multi-point-method-movement-bound}, it suffices to show $w^{(t)}_{\ktstar - 1} = w^{(t - 1)}_{\ktstar}$, as combining the latter with \eqref{eq:evaluating-the-idset-sum} yields
\begin{align*}
    \sum_{k \in \idset^{(t)}} \breg{w^{(t)}_{k - 1}}{w^{(t)}_{k}} = \breg{w^{(t)}_{\ktstar - 1}}{w^{(t)}_{\ktstar}} = \breg{w^{(t - 1)}_{\ktstar}}{w^{(t)}_{\ktstar}} \le \sum_{k \in [K]} \breg{w^{(t - 1)}_k}{w^{(t)}_k}
\end{align*}
and recall $t \in [T]$ was set arbitrarily. (The second inequality in \eqref{eq:dyadic-prox-multi-point-method-movement-bound} is immediate from Lemma~\ref{lem:multiprox-correctness}.) As for proving $w^{(t)}_{\ktstar - 1} = w^{(t - 1)}_{\ktstar}$, it suffices to show $a_{\ktstar - 1, t} = a_{\ktstar, t - 1}$ by the above general characterization of $w_k^{(t)}$. In other words, we need to show that the largest multiple of $2^{K - \ktstar + 1}$ which is at most $t$ is equal to the largest multiple of $2^{K - \ktstar}$ which is at most $t - 1$. This follows because $t$ is a multiple of $2^{K - \ktstar}$ due to the fact that $\ktstar \in \idset^{(t)}$, but $t$ is not a multiple of $2^{K - \ktstar + 1}$ by the definition of $\ktstar$ (in particular $\ktstar - 1 \notin \idset^{(t)}$). Therefore, the multiple of $2^{K - \ktstar}$ before $t$ must coincide with the multiple of $2^{K - \ktstar + 1}$ before $t$.

Finally, to prove that $(w_{k - 1}^{(t - 1)}, w_{k}^{(t - 1)}) \ne (w_{k - 1}^{(t)}, w_{k}^{(t)})$ for some $t \ge 1$ and $2 \le k \le K$ implies $k \in \idset^{(t)}$, we will prove the contrapositive; namely, $k \in [K] \setminus \idset^{(t)}$ implies $(w_{k - 1}^{(t - 1)}, w_{k}^{(t - 1)}) = (w_{k - 1}^{(t)}, w_{k}^{(t)})$. Note that for any $2 \le k \le K$, we have $k \in [K] \setminus \idset^{(t)}$ implies $k - 1 \in [K] \setminus \idset^{(t)}$ due to the choice of $\idset^{(t)}$ (if $t$ isn't divisible by $2^{K - k}$, it also isn't divisible by $2^{K - k + 1}$). Thus, $(w_{k - 1}^{(t - 1)}, w_{k}^{(t - 1)}) = (w_{k - 1}^{(t)}, w_{k}^{(t)})$ for all $k$ such that $k \in [K] \setminus \idset^{(t)}$ and $k \ge 2$ by Line~\ref{line:final-algo-wtk-unified-def}.
\end{proof}





 
%


\subsection{Prox multi-point method outer loop for matrix games}
\label{subsec:prox-multi-matrix-games}



%



%
%
%
%
%

In this section, we provide and analyze the outer loop of our ultimate algorithm for obtaining Theorems \ref{thm:final-result-l1-l1-aka-zero-sum} and \ref{thm:final-result-l2-l1-aka-SVM}. In particular, the main guarantee of this section, Theorem~\ref{thm:matrix-games-outer-loop-guarantee}, provides key bounds for our amortized analysis of the total number of matvecs made over all model-update steps within inner loop calls (Section~\ref{sec:sug-solver}).


%



\begin{assumptions}
In this section (Section~\ref{subsec:prox-multi-matrix-games}), we fix arbitrary dgf setups $\dgfsetup\x = (\xset \subset \R^n, \rx)$ and $\dgfsetup\y = (\yset \subset \R^m, \ry)$ with $\dgfsetup = (\zset \subset \R^d, r) \defeq \prodsetup(\dgfsetup\x, \dgfsetup\y)$ (recall Definition~\ref{def:product-dgf-setups}) and $\Gamma_\dgfsetup \ge \max_{z, z' \in \zset} r(z) - r(z')$.
%
For a given $A \in \R^{m \times n}$, our goal in this section is to obtain an $\epsilon$-solution of the general matrix game \eqref{eq:intro-general-matrix-game}.
%
%
%
%
%
Moreover, we assume throughout that $\dgfsetup$ is $\zeta$-compatible with respect to $A$ in the sense of Definition~\ref{def:zeta-compatible-mapping} given below. 
\end{assumptions}

Definition~\ref{def:zeta-compatible-mapping} abstracts the key property we use to bound the total number of matvecs made within all inner loop model-update steps by the divergences between iterates (recall, e.g., \eqref{eq:using-div-bound-overview} and \eqref{eq:final-bound-with-movement-term} from Section~\ref{sec:overview-of-approach}). We later show that Definition~\ref{def:zeta-compatible-mapping} is satisfied in the context of Theorems \ref{thm:final-result-l1-l1-aka-zero-sum} and \ref{thm:final-result-l2-l1-aka-SVM} with $\zeta = 2$ in Lemma~\ref{lemma:compatibility}.




\begin{definition}[$\zeta$-compatible]
    \label{def:zeta-compatible-mapping}
%
For $\zeta > 0$, we say the dgf setup $\dgfsetup$ is \emph{$\zeta$-compatible with respect to a matrix $B \in \R^{m \times n}$} if $\innorm{(B)_{z'} - (B)_{z}}_F^2 \le \zeta \breg{z}{z'}$ for all $z, z' \in \zset$.
\end{definition}












%
 

%

%

%
%
%
%
%








%

%


%


We now formally define \emph{matrix-approximation paths.} Recall from Section~\ref{sec:overview-of-approach} that matrix-approximation paths constitute the representation of $(A)_{z^{(t)}}$ for an iterate $z^{(t)}$, namely the $\Delta_\ell$ terms in the telescoping sum and the corresponding models $M_\ell$, which gets passed to our bisection search procedure (Section~\ref{sec:MDMP-implementation}) and through it our inner loop subproblem solver $\SUG$ (Section~\ref{sec:sug-solver}). The requirements of Defintion \ref{def:matrix-approx-path} abstract the key properties of this representation (implicitly used, e.g., in the sketches in Section~\ref{sec:overview-of-approach}) which are necessary for obtaining our matvec bounds. The \emph{size} of a matrix-approximation path \eqref{eq:def-of-size-for-path} quantifies how good the models $M_\ell$ are for the corresponding differences $\Delta_\ell$, and will be useful in our amortized analysis to bound the number of model-update steps made in all calls to $\SUG$ (although the latter is abstracted away in this section through an oracle introduced next).




\begin{definition}[Matrix-approximation path]
    \label{def:matrix-approx-path}
For $z \in \zset$ and $L \in \Z_{>0}$, we call $\pathd = \inbraces{\Delta_\ell \in \R^{m \times n}, M_\ell \in \R^{m \times n}}_{\ell \in [L]}$ a \emph{matrix-approximation path to $z$} if: (i) $\sum_{\ell \in [L]} \Delta_\ell = (A)_z$, (ii) a matvec to any $\Delta_\ell$ can be computed in $O(1)$ matvecs to $A$, and (iii) the matrices $M_\ell$ are known explicitly. We refer to $L$ as the \emph{length} of $\pathd$, and additionally define
\begin{align}
    \label{eq:def-of-size-for-path}
    \size(\pathd) \defeq \sum_{\ell \in [L]} \innorm{\Delta_\ell - M_\ell}_F^2.
\end{align}
\end{definition}



We now extend Definition~\ref{def:DMP} from Section~\ref{subsec:prox-multi-general-monotone-ops} in Definition~\ref{def:MDMP} below. Note in particular that an $\epsilon'$-$\MDMP$ is an $\epsprim$-$\DMP$ with respect to the dgf setup $\dgfsetup$ and monotone operator $\gm f_A$. The only difference (or rather extension) is that an $\epsilon'$-$\MDMP$ also takes in a matrix-approximation path $\pathd = \inbraces{\Delta_\ell, M_\ell}_{\ell \in [L]}$ to some $z \in \uset$ and outputs another matrix-approximation path $\pathd' = \inbraces{\Delta_\ell, M_\ell' \in \R^{m \times n}}_{\ell \in [L]}$ to $z$ where only the models may have changed. 
%
As will be discussed further below, we use an $\epsilon'$-$\MDMP$ to abstract our bisection search procedure (Section~\ref{sec:MDMP-implementation}) and inner loop subproblem solver (Section~\ref{sec:sug-solver}). We also discuss the reason for the requirement $z \in \uset$ below when discussing the main guarantee of this section (Theorem~\ref{thm:matrix-games-outer-loop-guarantee}).

%


 


\begin{definition}[$\epsilon'$-$\MDMP$] \label{def:MDMP}
    For $\epsprim > 0$, we call $\OMDMP(\cdot, \cdot)$ an \emph{$\epsprim$-matrix-games dynamic multiprox oracle} ($\epsprim$-$\MDMP$) if on input $(\uset \subset \zset, \pathd = \inbraces{\Delta_\ell, M_\ell}_{\ell \in [L]})$ where $\uset$ is a finite nonempty multiset and $\pathd$ is a matrix-approximation path to some $z \in \uset$, it returns $(z' \in \zset, \alpha > 0, \pathd' = \inbraces{\Delta_\ell, M_\ell' \in \R^{m \times n}}_{\ell \in [L]})$ such that: (i) the outputs $z', \alpha$ satisfy the property of Definition~\ref{def:DMP} with respect to $\gm f_A$ and $\dgfsetup$, i.e., \eqref{eq:DMP-def-variational-property} holds with $g \gets \gm f_A$, and (ii) $\pathd'$ is also a matrix-approximation path to $z$. For $\beta, \gamma, \rho > 0$, we say an $\epsilon'$-$\MDMP$ is \emph{$(\beta, \gamma, \rho)$-kinetic} if additionally the output always satisfies at least one of the conditions \ref{item:DMP-beta-cond} or \ref{item:DMP-movement-cond} from Definition~\ref{def:DMP}.
\end{definition}

%

%
%
%
%
%
%
%
%
%
%
%
%
%





Next, Algorithm~\ref{alg:final-algo-outer-loop} gives the outer loop of our ultimate algorithm for obtaining Theorems \ref{thm:final-result-l1-l1-aka-zero-sum} and \ref{thm:final-result-l2-l1-aka-SVM}. In particular, Algorithm~\ref{alg:final-algo-outer-loop} can be viewed as an instantiation of Algorithm~\ref{alg:multiprox-method-monotone-op}
%
for the dgf setup $\dgfsetup$ and monotone operator $\gm f_A$, with extensions to handle model creation, clearing, and passing. Indeed, note that the initialization and updates to the iterates $w_k^{(t)}$ and $z^{(t)}$ in Algorithm~\ref{alg:final-algo-outer-loop} are identical to those in Algorithm~\ref{alg:multiprox-method-monotone-op}. As for differences between the two algorithms, we specifically instantiate $\idset^{(t)}$ in Line~\ref{line:final-algo-idset} of Algorithm~\ref{alg:final-algo-outer-loop} to the choice made in Definition~\ref{def:dyadic-prox-multi-point-method}, thereby making Algorithm~\ref{alg:final-algo-outer-loop} a dyadic prox method.
%

The other key difference is of course the model iterates: Algorithm~\ref{alg:final-algo-outer-loop} maintains $K$ sequences of said iterates, where the $k$-th sequence for $k \in [K]$ is denoted $M_k^{(0)}, M_k^{(1)}, \dots$. In particular, $M_{k}^{(t - 1)}$ (for $t \in [T + 1]$) is a model for the difference $\Delta^{(t - 1)}_{k} \defeq (A)_{w^{(t - 1)}_{k}}  -  (A)_{w^{(t - 1)}_{k - 1}}$, where we overload notation and define $(A)_{w^{(t - 1)}_{0}} \defeq 0_{m \times n}$ for brevity (note that $\Delta^{(t - 1)}_{k}$ is defined the same way in Line~\ref{line:final-algo-set-P^(t)}).
%
%
The key observation which enables us to only store and update $K$ models (as opposed to models corresponding to all pairs $(j, j')$ gapped by a power of two as initially suggested in Section~\ref{sec:overview-of-approach}) is that Algorithm~\ref{alg:final-algo-outer-loop} maintains the invariant
\begin{align}
    \label{eq:expl-sec4_2-for-path}
    (A)_{z^{(t - 1)}} = (A)_{w_K^{(t - 1)}} = \sum_{k \in [K]} \Delta_k^{(t - 1)} 
\end{align}
for all $t \in [T + 1]$. In other words, $\pathd^{(t)}$ (defined in Line~\ref{line:final-algo-set-P^(t)}) is a matrix-approximation path to $z^{(t - 1)}$, as in the sketch given in Section~\ref{sec:overview-of-approach}. The first equality in \eqref{eq:expl-sec4_2-for-path} follows because in fact $w_K^{(t)} = z^{(t)}$ for all $t \ge 0$ due to the fact that $K \in \idset^{(t)}$ for all $t \ge 1$ and the update rule of Line~\ref{line:final-algo-wtk-unified-def}. 

More broadly, we have $w_{k}^{(t)} = z^{(a_{k, t})}$ for all $t \ge 0$ and $k \in [K]$ where $a_{k, t} \defeq t - (t \bmod 2^{K - k})$; namely, $a_{k, t}$ is the largest multiple of $2^{K - k}$ which is at most $t$.
%
 Again, this is due to the choice of $\idset^{(t)}$ in Line~\ref{line:final-algo-idset} and the update rule of Line~\ref{line:final-algo-wtk-unified-def}, and it implies the terms in the rightmost summation in \eqref{eq:expl-sec4_2-for-path} are in fact tracking the differences in the binary decomposition of $t - 1$ (as long as $K$ is sufficiently large, e.g., $K \ge \log_2 T + 5$ as in Lemma~\ref{lem:dyadic-prox-movement-bound}). For example, if $K = 20$ and we are on iteration $t = 10$, then one can verify $w_{20}^{(t - 1)} = z^{(9)}$, $w_{19}^{(t - 1)} = z^{(8)}$, $w_{18}^{(t - 1)} = z^{(8)}$, $w_{17}^{(t - 1)} = z^{(8)}$, and $w_j^{(t - 1)} = z^{(0)}$ for all $1 \le j \le 16$. In other words (assuming the iterates $z^{(j)}$ are unique for simplicity), if $w_{k}^{(t - 1)} = z^{(j)}$ and $w_{k - 1}^{(t - 1)} = z^{(j')}$ for some $j \ne j'$, then the jump from $j'$ to $j$ appears in the binary decomposition of $t - 1$ (which is 9 in the above example).\footnote{We note that this is the reason mentioned in Section~\ref{sec:overview-of-approach} for why we set $\idset^{(t)} \gets \inbraces{k \in [K] : \text{$t$ is divisible by $2^{K - k}$}}$ in Definition~\ref{def:dyadic-prox-multi-point-method} instead of $\idset^{(t)} \gets \inbraces{k \in [K] : \text{$t$ is divisible by $2^{k - 1}$}}$. In the latter case, $w_1^{(t)}$ would be the head of the path instead of $w_K^{(t)}$, i.e., the directions of the paths $\pathd^{(t)}$ in Line~\ref{line:final-algo-set-P^(t)} and $\pathd'^{(t)}$ in Line~\ref{line:final-algo-MDMP} would need to be reversed, resulting in perhaps less concise indexing.}

Let us now discuss the logic of the updates to the models $M_k^{(t)}$ in Algorithm~\ref{alg:final-algo-outer-loop}. All models are initialized to $0_{m \times n}$ in Line~\ref{line:final-algo-w-M-init}. As mentioned above, the $\MDMP$ oracle call in Line~\ref{line:final-algo-MDMP} abstracts our bisection search procedure (Section~\ref{sec:MDMP-implementation}) and inner loop subproblem solver $\SUG$ (Section~\ref{sec:sug-solver}). As discussed in Section~\ref{sec:overview-of-approach}, $\SUG$ will perform updates to the models $M_k^{(t - 1)}$ within the path $\pathd^{(t)}$ in model-update steps, and thus $M_k'^{(t)}$ are the results of all these updates (potentially over many calls to $\SUG$ within the bisection search procedure). In Line~\ref{line:final-algo-wtk-unified-def}, we set the new model iterates $M_k^{(t)}$. If $k \in \idset^{(t)}$, then potentially $w_k^{(t)} \ne w_k^{(t - 1)}$ (again due to the update logic for $w_k^{(t)}$ in Line~\ref{line:final-algo-wtk-unified-def}), and therefore potentially $\Delta_k^{(t - 1)} \ne \Delta_k^{(t)}$. Since $M_k^{(t - 1)}$ is a model for $\Delta_k^{(t - 1)}$ and $M_k^{(t)}$ is a model for $\Delta_k^{(t)}$, we therefore reset $M_k^{(t)} \gets 0_{m \times n}$ in Line~\ref{line:final-algo-wtk-unified-def}. If $k \in [K] \setminus \idset^{(t)}$, then we are guaranteed $\Delta_k^{(t - 1)} = \Delta_k^{(t)}$ due to the contrapositive of the final statement in Lemma~\ref{lem:dyadic-prox-movement-bound}.
%
Thus, the term $M_k$ is modeling has not changed, and we update $M_k^{(t)}$ in Line~\ref{line:final-algo-wtk-unified-def} to the corresponding output of the $\MDMP$ call in Line~\ref{line:final-algo-MDMP}. 



%





\RestyleAlgo{ruled}
\DontPrintSemicolon
\SetKwComment{Comment}{/* }{ */}
\begin{algorithm2e}[h!]
\caption{Prox multi-point method for matrix games}
\label{alg:final-algo-outer-loop}
\KwInput{Precision $\epsilon > 0$, max centers per step $K$, $\epsilon$-$\MDMP$ oracle $\OMDMP$
%
%
%
}

%

%

$z^{(0)} \gets \argmin_{z \in \zset_\nu} r(z)$ ~and~ $t \gets 0$\;

$(w^{(0)}_k, M_{k}^{(0)} )  \gets (z^{(0)}, 0_{m \times n} )$ for all $k \in [K]$ \label{line:final-algo-w-M-init}



%

%

%

%

\While(\tcp*[f]{Recall $\Range_\dgfsetup \ge \max_{z, z' \in \zset} r(z) - r(z')$}){
    $\sum_{j \in [t]} (\alpha^{(j)})^{-1} < K \Range_\dgfsetup \epsilon^{-1}$ 
}{ \label{line:final-algo-while-loop}

    $t \gets t + 1$ \;

    %

    %

    %

    %

    %

    \tcp{Here, we overload notation and let $(A)_{w_{0}^{(t - 1)}} \defeq 0_{m \times n}$}

    %

    %

    $\pathd^{(t)} \gets \inbraces{\Delta^{(t - 1)}_{k}  ,  M^{(t - 1)}_{k}    }_{k \in [K]}$ ,  where $\Delta^{(t - 1)}_{k} \defeq (A)_{w^{(t - 1)}_{k}}  -  (A)_{w^{(t - 1)}_{k - 1}}$ for $k \in [K]$ \label{line:final-algo-set-P^(t)}

    $\idset^{(t)} \gets \inbraces{k \in [K] : \text{$t$ is divisible by $2^{K - k}$}}$ \label{line:final-algo-idset}
    
    %

    %

    %



    $\inparen{z^{(t)}, \alpha^{(t)}, \pathd'^{(t)} \defeq \inbraces{\Delta^{(t - 1)}_{k}  ,  M'^{(t)}_{k}    }_{k \in [K]  }} \gets \OMDMP(\inbraces{w^{(t - 1)}_k : k \in \idset^{(t)}}, \pathd^{(t)})$           \label{line:final-algo-MDMP} 
    
    %

%
%
%
%
%
%

    $(w^{(t)}_k, M_{k}^{(t)}) \gets \begin{cases} 
     ( z^{(t)},  0_{m \times n} ) \, ,  & \text{for all } k \in \idset^{(t)} \\
     ( w^{(t - 1)}_k, M'^{(t)}_{k} ) \, ,  & \text{for all } k \in [K] \setminus \idset^{(t)} 
   \end{cases}$ \label{line:final-algo-wtk-unified-def}

   


%

%




%

%

%

%

%

%
%
%
%
%
}

\tcp{$T$ is used in the analysis to refer to the final iteration count}

\Return{$\zbar \defeq \frac{1}{S} \sum_{j \in [T]} (\alpha^{(j)})^{-1} z^{(j)}$, where $T \defeq t$ and $S \defeq \sum_{j \in [T]} (\alpha^{(j)})^{-1}$} \label{line:final-algo-return}

\end{algorithm2e}



%
%
%
%
%
%
%
%
%
%
%
%

%




We give our guarantee for Algorithm~\ref{alg:final-algo-outer-loop} in Theorem~\ref{thm:matrix-games-outer-loop-guarantee}. The latter chooses $K$ so as to satisfy the lower bound requirement of Lemma~\ref{lem:dyadic-prox-movement-bound}, while also ensuring $K = \Otilde(1)$, which will be useful when we instantiate Theorem~\ref{thm:matrix-games-outer-loop-guarantee} in Section~\ref{sec:putting-together} in the context of Theorems \ref{thm:final-result-l1-l1-aka-zero-sum} and \ref{thm:final-result-l2-l1-aka-SVM}. Beside guaranteeing correctness, Theorem~\ref{thm:matrix-games-outer-loop-guarantee} provides several bounds which will enable our ultimate matvec bounds in Section~\ref{sec:putting-together}. The iteration bound on $T$ and the bound on $\sum_{t \in [T]} (\alpha^{(t)})^\rho$ are immediate from Lemma~\ref{lem:multiprox-iteration-bound} and repeated here for ease of reference in Section~\ref{sec:putting-together}. We note that the latter as well as the requirement $z \in \uset$ in Definition~\ref{def:MDMP} are ultimately used to handle the point discussed at the end of Section~\ref{sec:overview-of-approach}---needing to bound the sum of the $\innorm{(A)_{\zground} - A_{z^{(t - 1)}}}_F^2$ terms in Section~\ref{sec:putting-together}. 

The bound on $\sum_{t \in [T]} \insquare{\size(\pathd^{(t)}) - \size(\pathd'^{(t)})}$ in Theorem~\ref{thm:matrix-games-outer-loop-guarantee} is new; namely, it uses the additional machinery of this section as opposed to only that of Section~\ref{subsec:prox-multi-general-monotone-ops}.
%
It is used in Section~\ref{sec:putting-together} to bound the total number of matvecs made over all model-update steps within calls to the subproblem solver $\SUG$ (Section~\ref{sec:sug-solver}) within our implementation of the $\MDMP$ oracle. We note that the proof of this bound is where we use the ``alternate movement bound'' for the dyadic prox method given in Lemma~\ref{lem:dyadic-prox-movement-bound}. In particular, the terms $\breg{w_{k - 1}^{(t)}}{w_k^{(t)}}$ for $k \in \idset^{(t)}$ in the leftmost summation in \eqref{eq:dyadic-prox-multi-point-method-movement-bound} correspond to models $M_k^{(t)}$ for $k \in \idset^{(t)}$ which are reset to $0_{m \times n}$ in Line~\ref{line:final-algo-wtk-unified-def} of Algorithm~\ref{alg:final-algo-outer-loop}. Using the assumption that $\dgfsetup$ is $\zeta$-compatible with respect to $A$, we are able to bound $\innorm{\Delta_k^{(t)} - M_k^{(t)}}_F^2 = \innorm{\Delta_k^{(t)}}_F^2 \le \zeta \breg{w_{k - 1}^{(t)}}{w_k^{(t)}}$ for $k \in \idset^{(t)}$ in the proof.


%

%






\begin{theorem}[Algorithm~\ref{alg:final-algo-outer-loop} guarantee]
    \label{thm:matrix-games-outer-loop-guarantee}
    Suppose the $\MDMP$ oracle given as input to Algorithm~\ref{alg:final-algo-outer-loop} is $(\beta, \gamma, \rho)$-kinetic (Def.~\ref{def:MDMP}) and we choose $K \gets \inceil{5 \log_2 \inparen{\Gamma_\dgfsetup (\beta \epsilon^{-1} + \gamma^{- \frac{1}{\rho+1}} \epsilon^{- \frac{\rho}{\rho + 1}}) + 2} } + 5$.
    %
    %
    %
    %
    %
    %
    Then Algorithm~\ref{alg:final-algo-outer-loop} terminates with $T \le K \Gamma_\dgfsetup (\beta \epsilon^{-1} + \gamma^{- \frac{1}{\rho+1}} \epsilon^{- \frac{\rho}{\rho + 1}}) + 2$ and the output $\zbar$ is a $2 \epsilon$-solution of \eqref{eq:intro-general-matrix-game}. Furthermore, the length of $\pathd^{(t)}$ is $K$ for all $t \in [T]$, and
    \begin{align}
        \label{eq:bound-on-sum-alpha-and-path-diffs}
        \sum_{t \in [T]} (\alpha^{(t)})^\rho \le K \Gamma_{\dgfsetup} \gamma^{-1} + T \beta^\rho 
         ~~\text{and}~~ \sum_{t \in [T]} \insquare{\size(\pathd^{(t)}) - \size(\pathd'^{(t)})} \le \innorm{(A)_{z^{(0)}}}_F^2 + \zeta K \Gamma_{\dgfsetup} \, .
    \end{align}
    %
    %
    %
    %
    %
    %
    %
    %
    %
    %
    %
    %
    %
    %
    %
    %
    %
    %
    %
\end{theorem}

\begin{proof}
First, we verify that the input $\pathd^{(t)}$ to the $\MDMP$ oracle in Line~\ref{line:final-algo-MDMP} is indeed a matrix-approximation path to some $z \in \inbraces{w^{(t - 1)}_k : k \in \idset^{(t)}}$ (note that the latter is the multiset passed into the $\MDMP$ oracle in Line~\ref{line:final-algo-MDMP}), thereby satisfying the stipulations of Definition~\ref{def:MDMP}. Indeed, note
\begin{align*}
    \sum_{k \in [K]} \Delta_k^{(t - 1)} = (A)_{w_K^{(t - 1)}} - (A)_{w_{0}^{(t - 1)}} = (A)_{w_K^{(t - 1)}}
\end{align*}
since $(A)_{w_{0}^{(t - 1)}} = 0$ by definition. Note $w_K^{(t - 1)} \in \inbraces{w^{(t - 1)}_k : k \in \idset^{(t)}}$ as required since $K \in \idset^{(t)}$. Furthermore, the matrices $M_{k}^{(t - 1)}$ are known explicitly for all $t \in [T]$ since the matrices $M_k'^{(t)}$ are known explicitly for all $t \in [T]$ by Definition~\ref{def:MDMP}. Finally, it is clear that matvecs with any $\Delta_k^{(t - 1)}$ can be computed with $O(1)$ matvecs to $A$ by Definition~\ref{def:product-dgf-setups}.

Note then that Algorithm~\ref{alg:final-algo-outer-loop} is an instantiation of Algorithm~\ref{alg:multiprox-method-monotone-op} for the dgf setup $\dgfsetup$ and monotone operator $\gm f_A$. Then by Lemmas \ref{lem:regret-bounds-the-gap}, \ref{lem:multiprox-correctness}, and \ref{lem:multiprox-iteration-bound}, Algorithm~\ref{alg:final-algo-outer-loop} terminates with $T \le K \Gamma_\dgfsetup (\beta \epsilon^{-1} + \gamma^{- \frac{1}{\rho+1}} \epsilon^{- \frac{\rho}{\rho + 1}}) + 2$ and $\zbar$ is a $2 \epsilon$-solution for \eqref{eq:intro-general-matrix-game}. The first inequality in \eqref{eq:bound-on-sum-alpha-and-path-diffs} is immediate from Lemma~\ref{lem:multiprox-iteration-bound}, and thus we focus on the second for the remainder of the proof.

%
%
%
%
%
%
%
%

%

Toward this goal, define $\pathd^{(T + 1)} \gets \inbraces{\Delta^{(T)}_{k}  ,  M^{(T)}_{k}    }_{k \in [K]}$ where $\Delta^{(T)}_{k} \defeq (A)_{w^{(T)}_{k}}  -  (A)_{w^{(T)}_{k - 1}}$ for all $k \in [K]$ (namely, extending the definitions of Line~\ref{line:final-algo-set-P^(t)} to $t \gets T + 1$; here as in Line~\ref{line:final-algo-set-P^(t)}, we overload notation and let $(A)_{w_{0}^{(T)}} \defeq 0_{m \times n}$ for brevity). Then note that for any $t \in [T]$, we have
\begin{align*}
    \size(\pathd^{(t + 1)}) &= \sum_{k \in \idset^{(t)}} \innorm{\Delta_k^{(t)} - M_k^{(t)}}_F^2  + \sum_{k \in [K] \setminus \idset^{(t)}} \innorm{\Delta_k^{(t)} - M_k^{(t)}}_F^2 \\
    &\overeq{(i)} \sum_{k \in \idset^{(t)}} \innorm{\Delta_k^{(t)} }_F^2  + \sum_{k \in [K] \setminus \idset^{(t)}} \innorm{\Delta_k^{(t - 1)} - M_k'^{(t)}}_F^2 \\
    &\le \sum_{k \in \idset^{(t)}} \innorm{\Delta_k^{(t)} }_F^2  + \sum_{k \in [K]} \innorm{\Delta_k^{(t - 1)} - M_k'^{(t)}}_F^2 \\
    &= \sum_{k \in \idset^{(t)}} \innorm{\Delta_k^{(t)} }_F^2 + \size(\pathd'^{(t)}) \, .
\end{align*}
Here, $(i)$ follows because $M_k^{(t)} = 0$ for all $k \in \idset^{(t)}$ by Line~\ref{line:final-algo-wtk-unified-def}. Moreover, we claim $\Delta_k^{(t)} = \Delta_k^{(t - 1)}$ and $M_k^{(t)} = M_k'^{(t)}$ for all $k \in [K] \setminus \idset^{(t)}$. The latter is immediate from Line~\ref{line:final-algo-wtk-unified-def} of Algorithm~\ref{alg:final-algo-outer-loop}. As for the former, the case where $2 \le k \le K$ follows from the final claim of Lemma~\ref{lem:dyadic-prox-movement-bound} (it is the contrapositive). As for the case $\Delta_1^{(t)} = \Delta_1^{(t - 1)}$, recall $(A)_{w_{0}^{(t)}} = (A)_{w_{0}^{(t - 1)}} = 0$ by definition.
%
Then 
%
\begin{align}
    \label{eq:telescoping-size-sum}
    \begin{split}
    0 \le \size(\pathd^{(T + 1)}) &= \size(\pathd^{(1)}) + \sum_{t \in [T]} \insquare{\size(\pathd^{(t + 1)}) - \size(\pathd^{(t)})} \\
    &\le \size(\pathd^{(1)}) + \sum_{t \in [T]} \sum_{k \in \idset^{(t)}} \innorm{\Delta_k^{(t)}}_F^2
    + \sum_{t \in [T]} \insquare{\size(\pathd'^{(t)}) -  \size(\pathd^{(t)})} \, .
    \end{split}
\end{align}
Note that by the choice of $w_k^{(0)}$ and $M_k^{(0)}$ in Line~\ref{line:final-algo-w-M-init} as well as the fact that $(A)_{w_0^{(0)}} = 0$ by definition, we have $\size(\pathd^{(1)}) = \innorm{(A)_{z^{(0)}}}_F^2$.
%
%
%
Furthermore,
\begin{align*}
    \sum_{t \in [T]} \sum_{k \in \idset^{(t)}} \innorm{\Delta_k^{(t)}}_F^2 \le \zeta \sum_{t \in [T]} \sum_{k \in \idset^{(t)}}  \breg{w_{k - 1}^{(t)}}{w_k^{(t)}} \le \zeta K \Gamma_{\dgfsetup}
\end{align*}
by Definition~\ref{def:zeta-compatible-mapping} and Lemma~\ref{lem:dyadic-prox-movement-bound}, noting Algorithm~\ref{alg:final-algo-outer-loop} is indeed an instantiation of the dyadic prox method (Definition~\ref{def:dyadic-prox-multi-point-method}) by the choice of $\idset^{(t)}$ in Line~\ref{line:final-algo-idset}, and $K \ge \log_2 T + 5$ since $T \le K \Gamma_\dgfsetup (\beta \epsilon^{-1} + \gamma^{- \frac{1}{\rho+1}} \epsilon^{- \frac{\rho}{\rho + 1}}) + 2$. Then rearranging \eqref{eq:telescoping-size-sum} and using the subsequent bounds, we obtain the desired inequality:
\begin{align*}
    %
    \sum_{t \in [T]} \insquare{\size(\pathd^{(t)}) - \size(\pathd'^{(t)})} \le \innorm{(A)_{z^{(0)}}}_F^2 + \zeta K \Gamma_{\dgfsetup} \, .
\end{align*}
%
\end{proof}

%






%
%

%
%
%
%
%
%
%
%
%
%
%
%
%
%
%
%
%
%
%

%
%
%
%
%

%
%
%
%
%
%
%
%
%
%
%
%
%
    
    
%

%
%
%
%
%
%
%
%
%
%
%
%
%
%
%
%
%
%

%
%
%
%
%
%
%
%
%
%
%
%
%
%
%
%

%
%
%
%
%
%
%
%
%
%
%
%
%
%
%
%
%
%
%
%







%

%
%
%
%

%
%
%
%
%
%



%


%
%
%
%
%
%
%
    

%

%

%

%

%

%

%

%

%

%

%

%
   
%

%
%


%
%
%

%

%
%
%
%

%
%
%
%
%
%
%

%
%
%
%
%
%
%
%
%
%
%
%
%
%

%
%
%
%
%
%
%
%
%
%
%
%
%
%
  
%

%

\section{$\MDMP$ implementation for matrix games}
\label{sec:MDMP-implementation}

%
%
%
%
%
%
%
%
%
%
%
%
%
%
%
%
%
%
%
%
%
%
%
%
%
%
%
%
%
%
%
%
%

%

In this section, we provide and analyze our implementation of a dynamic $\epsilon$-$\MDMP$ oracle (Definition~\ref{def:MDMP}) for matrix-games using the bisection search procedure discussed in Section~\ref{sec:overview-of-approach}. The pseudocode of our implementation is described in Algorithm~\ref{alg:DMP-implementation-matrix-games} and we derive and analyze it in several steps. In particular, we reduce implementing an $\epsilon$-$\MDMP$ to solving a sequence of what we call \emph{constrained prox multi-point problems}. In Section~\ref{sec:wrapper}, we define these problems and related solution concepts. In Section~\ref{sec:binary_search_intro} we introduce a crucial bisection search subroutine (Algorithm~\ref{alg:cautious}) which enables our method, as discussed in Section~\ref{sec:overview-of-approach}. In Section~\ref{sec:mgdamo-sub} we show how to leverage these preliminaries to implement an $\epsilon$-$\MDMP$ oracle $\MDMPImp$ (Algorithm~\ref{alg:DMP-implementation-matrix-games}). Finally, in Section~\ref{sec:analysis-of-implementation} we analyze the implementation.

As in \citep{karmarkar2025solvingzerosumgames}, our algorithm leverages the notion of Hessian stability \citep{carmon2020acceleration, karimireddy2018globallinearconvergencenewtons} in order to implement the \emph{inner loop} discussed in Sections~\ref{sec:intro} and~\ref{sec:overview-of-approach}. In order to formalize this, in the remainder of the paper, we use the following notions of a $c$-\textit{stable ball} and \textit{stability} with respect to a fixed (but arbitrary) mapping. First, the following Definition~\ref{def:stable-region-hess} defines a notion of a stable ball, generalizing Definition 5.1 of \cite{karmarkar2025solvingzerosumgames} to general dgf setups. 
%

%
%

%
%
%
%
%
%
%
%
%

%
%
%
%
%
%
%

\begin{definition}[$c$-stable ball]\label{def:stable-region-hess} For a dgf setup $\cS = (\zset \subset \R^d, r)$, $z \in \cZ$ and $c > 1$, we define the \emph{$c$-stable ball about $z$} as $\cB_{c, z}^{\cS} \defeq \{z' \in \cZ : c^{-1} \cdot \hess r(z) \preceq \hess r(z') \preceq c \cdot \hess r(z)\}.$
\end{definition}

Next, we define a notion of stability with respect to a mapping. The following definition generalizes the notion of stability introduced in Section 6.2 of \cite{karmarkar2025solvingzerosumgames} to arbitrary dgf setups. 

%

%
%

\begin{definition}[Stability]\label{def:best-response-stability} We say that a dgf setup $\cS = (\zset \subset \R^d, r)$ is $(\iota, \rho)$-\emph{stable with respect to a mapping}\footnote{Here, as usual, we allow $\cU$ to be a multiset of $\cZ$ when we write $\cU \subseteq \cZ$.}  $(\alpha > 0, \cU \subseteq \cZ) \mapsto \bestresponse(\alpha, \cU) \in \R^d$ if $\iota: \R_{>1} \to \R_{>1}$ is a strictly increasing, $\rho > 0$, and $z_\alpha^\star \in \cB^{\cS}_{\iota(c), \bestresponse(\alpha, \cU)}$ for any $\alpha > 0$ and $c > 1$ with $z_\alpha^\star \defeq \prox_\cU^\alpha(\nabla_\pm f_A; \cZ)$ and $\breg{\cU}{z_\alpha^\star} \leq c\alpha^\rho$.
\end{definition}

\paragraph{Assumptions.} %
In the remainder of this section, we fix an arbitrary dgf setup, $\cS = (\zset \subset \R^d, r)$, which is $(\iota, \rho)$-stable with respect to a fixed but arbitrary mapping $(\alpha, \cU) \mapsto \bestresponse(\alpha, \cU)$ (Definition~\ref{def:best-response-stability}) for some strictly increasing function $\iota: \R_{>1} \to \R_{>1}$ and $\rho > 0$. In addition, we assume that for any $z \in \cZ$ and $c > 1$, $\cB_{c, z}^\cS$ (Definition~\ref{def:stable-region-hess}) is closed and convex. In Section~\ref{sec:applications} we verify this assumption, specify $\cS, \rho,$ and $\bestresponse$, and bound $\iota$ for our particular applications in Theorems~\ref{thm:final-result-l1-l1-aka-zero-sum} and \ref{thm:final-result-l2-l1-aka-SVM}. 

\subsection{Constrained prox multi-point problems}\label{sec:wrapper}

 %

Here we introduce what we call a constrained prox multi-point problem, corresponding to implementing a constrained variant of the proximal step $\prox_{\cU}^\alpha(\nabla_\pm f_{A}; \cZ)$ (recall Definition~\ref{def:proximal-mappings}). We will implement our $\epsilon$-$\MDMP$ ($\MDMPImp$, Algorithm~\ref{alg:DMP-implementation-matrix-games}) by carefully iteratively solving constrained prox multi-point problems and processing their solutions.

\begin{definition}[Constrained prox multi-point problem]\label{def:subproblem} In the $(\cU, c, \alpha, z, \cS)$-\emph{constrained prox multi-point problem}, we are given a finite, non-empty multiset $\cU \subseteq \cZ$, $c > 1$, $\alpha >0$, and $z \in \cZ$ and define the \emph{solution to the problem} as $z^\star \defeq \prox_{\cU}^\alpha(\nabla_\pm f_A; \cB^{\cS}_{c, z})$ (recall the notation in \Cref{def:proximal-mappings}).
\end{definition}

More precisely we reduce implementing an $\epsilon$-DMDP to computing a type of approximate solution to constrained prox multi-point problems. The following definition introduces this notion of an \emph{$(\epsilon, \delta, \rho)$-approximate solution} to a constrained prox multi-point problem. 

\begin{definition}[Approximate solution]\label{def:approx-solution} 

For $\epsilon, \delta \geq 0$ and $\rho > 0$, letting $z^\star$ be the solution to the $(\cU, c, \alpha, z, \cS)$-constrained prox multi-point problem (Definition~\ref{def:subproblem}), we say that a point $z' \in \cB^{\cS}_{c, z}$ is an $(\epsilon, \delta, \rho)$-\emph{approximate solution} to the problem if, 
\begin{itemize}
    \item $\abs{\breg{\cU}{z'} - \breg{\cU}{z^\star}} < \alpha^\rho/10$, 
    \item $z' \in \cB^{\cS}_{1+\delta, z^\star}$, and
    \item if $\prox_\cU^\alpha(\nabla_\pm f_A; \cZ) \in \cB_{c, z}^{\cS}$ then $\inangle*{\nabla_{\pm}f_A(z') + \alpha \grad \breg{\cU}{z'}, z' -u} \leq \epsilon, \text{ for all } u \in \cZ$. 
\end{itemize}
\end{definition}

Correspondingly, we define an oracle for this approximate solution concept as follows. 

\begin{definition}\label{def:oracle} For any $\rho > 0$, a $\rho$-\emph{approximate solution oracle} $\OAPPROX$
(for $\cS$) takes in a finite non-empty multiset $\cU \subseteq \cZ$, $c > 1$, $\alpha >0$, $z \in \cZ$, a matrix approximation path $\cP = \{\Delta_\ell, M_\ell\}_{\ell \in [L]}$ to $z$ (Definition~\ref{def:matrix-approx-path}), and $\epsilon, \delta \geq 0$ and returns $(z', \cP' = \{\Delta_\ell, M'_\ell\}_{\ell \in [L]})$, where $z'$ is an $(\epsilon, \delta, \rho)$-approximate solution to the $(\cU, c, \alpha, z, \cS)$-constrained prox multi-point problem (Definition~\ref{def:approx-solution}) and $\cP'$ is a matrix approximation path to $z$. 
\end{definition}

%
%
%
%
%
%
%

%

%
%

In the next sections, we show how, for any $\beta > 0$, we can leverage a $\rho$-approximate solution oracle for $\cS$ to implement an $\epsilon$-$\MDMP$ oracle which is $(\beta, \Theta(1), \rho)$-kinetic (Definition~\ref{def:MDMP}). 

%

%
%


%

%
%
%
%
%
%
%
%
%
%
%

%
%
%
%
%
%
%
%
%
%
%
%

%
%
%
%
%
%
%
%
%
%
%
%
%
%
%

%

%

%

\subsection{Cautious Bisection Search}\label{sec:binary_search_intro}

In this section, we introduce a general routine $\cautiousSearch$ (Algorithm~\ref{alg:cautious}) which is our main bisection search procedure to reduce implementing an $\epsilon$-MDMP to implementing a $\rho$-approximate solution oracle. $\cautiousSearch$ is a key subroutine of our $\MDMP$ oracle implementation (the pseudocode of which is in Algorithm~\ref{alg:DMP-implementation-matrix-games}). $\cautiousSearch$ takes a tolerance $\epsilon \geq 0$, a range $[\theta_\ell, \theta_r] \subset \R_{\geq 0}$, and an oracle $\oracle(\cdot)$ that when queried at any $\alpha \in [\theta_\ell,\theta_r]$ outputs a point $z \in \zset \cup \{0\}$ and either $\oracleSuccess$ or $\oracleFailure$. In our application, $\theta_\ell$ corresponds to the $\beta$ parameter of a $\MDMP$ oracle (Definition~\ref{def:MDMP}) and $\theta_r$ corresponds to a value of $\alpha$ for which we are \emph{guaranteed} that $\breg{\cU}{\prox_{\cU}^\alpha(\nabla_\pm f_A; \cZ)} \leq C \alpha^\rho$ for appropriate $C$ (motivated by Definition~\ref{def:best-response-stability}). Correspondingly, the procedure assumes that $\oracle$ outputs $\oracleSuccess$ at $\alpha = \theta_r$ and then, either the oracle outputs $\oracleSuccess$ at $\alpha = \theta_\ell$ or else finds a pair of query values that are $\epsilon$-close where for the larger the $\oracle$ outputs $\oracleSuccess$ and for the smaller the $\oracle$ outputs $\oracleFailure$. 

It is possible to compute the desired $\alpha$ with a logarithmic number of queries using bisection search. However, in our application, querying the oracle for larger $\alpha$ may require more matvecs. Consequently, $\cautiousSearch$ instead queries the oracle at geometrically increasing values starting from $\alpha = \theta_\ell$ searching for the oracle to either output $\oracleFailure$ or $\oracleSuccess$ at $\alpha$. In the case it finds an $\oracleSuccess$ at $\alpha = \theta_\ell$, the procedure returns $\theta_\ell$. Otherwise, the oracle finds $\oracleSuccess$ for some $\alpha \in (\theta_\ell, \theta_r]$. In this case, the procedure performs a bisection search for an $\alpha_u$ such that the oracle outputs $\oracleSuccess$ at $\alpha_u$ and $\oracleFailure$ at an $\alpha_l \geq \alpha_u - \epsilon$. Ultimately, this ensures that the $\cautiousSearch$ both finds the requisite value of $\alpha$ with only a logarithmic number of queries to the oracle and that the oracle does not query the oracle with a value of $\alpha$ much higher than the value of $\alpha$ it ultimately outputs. In our application, this helps ensure that the bisection search only induces polylogarithmic factors of overhead in terms of the number of matvecs made in our applications.

The main guarantees of $\cautiousSearch$ are given below in \Cref{lemma:cautious_search}. A similar procedure was used in the $\lambda$-bisection procedure in Algorithm 1 of \citep{carmon2021thinking} for similar reasons. Our algorithm uses the same general ideas as in that procedure.

%
 
 \begin{algorithm2e}[ht]
 	\caption{$\cautiousSearch(\epsilon, \theta_\ell, \theta_r, \oracle(\cdot))$}
 	\label{alg:cautious}
 	\KwInput{Range lower bound $\theta_\ell > 0$, range upper bound $\theta_r > \theta_\ell$, error threshold $\epsilon \in \R_{\geq 0}$, oracle $\oracle : (0, \theta_r] \rightarrow (\cZ, \{\oracleSuccess, \oracleFailure\})$.}
    \BlankLine
    \tcp{If $\oracleSuccess$ at $\alpha = \theta_\ell$, output $\theta_\ell$.}
    \lIf{$\flag = \oracleSuccess$ when $(z, \flag) \gets \oracle(\theta_\ell)$}{\Return{$(z, \theta_\ell)$}}

    \tcp{Repeatedly double $L^{(i)}$ looking for $\oracleSuccess$ at $L^{(i)} \geq \theta_\ell$.}
 	$i = 1$ and $L^{(1)} = \theta_\ell$\;
 	\While{$\flag = \oracleFailure$ when $(z, \flag) \gets \oracle(L^{(i)})$ \label{line:cautious:start}}{
        $L^{(i + 1)} \gets \min\{2L^{(i)}, \theta_r\}$ and then $i \gets i + 1$ \label{line:cautious:end}
    }
    \BlankLine
    \tcp{Bisection search between $\oracleSuccess$ at $L^{(i_*)}$ and $\oracleFailure$ at $L^{(i_*-1)}$ for close $\oracleSuccess$ and $\oracleFailure$ output}
    $\alpha_u^{(1)} \gets L^{(i_*)}$, and $\alpha_\ell^{(1)} \gets L^{(i_*-1)}$ where $i_* = i$\;
    \For{$j \in [j_*]$ where $j_* \defeq \max\{1,\ceil{\log_2((\alpha_u^{(1)} - \alpha_{\ell}^{(1)})/\epsilon)}\}$\label{line:for:start}}{
 			$\alpha_m^{(j)} \gets \frac{\alpha_u^{(j)} + \alpha_{\ell}^{(j)}}{2}$ and 
            $(z', \flag) \gets \oracle(\alpha_m^{(j)})$\; 
 			\lIf{$\flag = \oracleSuccess$}{
 				$(z, \alpha_{u}^{(j +1)}) \gets (z', \alpha_{m}^{(j)})$ and $\alpha_{\ell}^{(j +1)} \gets \alpha_{\ell}^{(j)}$
 				}
 			\lElse{
 					$\alpha_{u}^{(j +1)} \gets \alpha_{u}^{(j)}$ and $\alpha_{\ell}^{(j +1)} \gets \alpha_{m}^{(j)}$
                    \label{line:for:end}
 				}
	}
 	\Return{$(z, \alpha_{u}^{(j_*+1)})$}
 \end{algorithm2e}
  
 \begin{lemma}[\cautiousSearch~guarantee]\label{lemma:cautious_search} Let 
 \[(z_*, \alpha_*) = \cautiousSearch(\epsilon, \theta_\ell, \theta_r, \oracle(\cdot))\] (\Cref{alg:cautious}) where  $\epsilon, \theta_\ell, \theta_r \in \R_{> 0}$ with $0 < \theta_\ell < \theta_r$ and $\oracle : [\theta_\ell, \theta_r] \rightarrow \zset \cup \{0\} \times \{\oracleSuccess, \oracleFailure\}$ is a deterministic oracle satisfying $\oracle(\theta_r) = (\cdot,\oracleSuccess)$.\footnote{In \Cref{lemma:cautious_search} and  \Cref{alg:cautious}, $\zset$ can be any non-empty set.}\footnote{Here and throughout $(a,b) = (\cdot,c)$ denotes that $b = c$.} Then $\alpha_* \in [\theta_l, \theta_r]$, $\oracle(\alpha_*) = (z_*,\oracleSuccess)$ and either \[
 \alpha_* = \theta_\ell
 \text{ or } 
 \oracle(\varsigma) = (\cdot,\oracleFailure)\text{ for some }
 \varsigma \in [\max\{\alpha_* - \epsilon, \theta_\ell\}, \alpha_*)\,.
 \]
 Furthermore, $\cautiousSearch(\cdot)$ makes at most $O(\log(\theta_r/\min\{\epsilon,\theta_\ell\}))$ queries to $\oracle(\alpha)$ and in each query $\alpha \in [\theta_\ell, \min\{2\alpha_*, \theta_r\}]$.
 \end{lemma}

%

 \begin{proof}
Every iteration of the $\code{while}$ loop, Line~\ref{line:cautious:start} to Line~\ref{line:cautious:end} that does not terminate either increases $L^{(i)}$ by a factor of 2 or has $L^{(i)} = \theta_r$. In the latter case, the while loop terminates at the next iteration, due to the guarantee that $\oracle(\theta_r) = (\cdot, \oracleSuccess)$. Consequently, the $\code{while}$ loop terminates with $O(\log(\theta_r/\theta_\ell))$ queries and ends the loop with $i_* > 1$, $(\cdot, \oracleFailure) = \oracle(L^{(i_*-1)})$ and $(\cdot, \oracleSuccess) = \oracle(L^{(i_*)})$ for $\theta_\ell < L^{(i_*-1)} < L^{(i_*)} = \min\{2L^{(i_*-1)} \theta_r\}$. 

We now check that the algorithm has the desired properties. Note that the $\code{for}$ loop (Line~\ref{line:for:start} to Line~\ref{line:for:end}) simply performs a bisection search between $\alpha_u^{(1)} = L^{(i_*)}$ and $\alpha_\ell^{(1)} = L^{(i_* - 1)}$ maintaining the invariant that $(z,\oracleSuccess) = \oracle(\alpha_u^{(j)})$ and $(\cdot,\oracleFailure) = \oracle(\alpha_\ell^{(j)})$. In addition, note that the returned value of $z_*$ corresponds to the first argument of $\oracle(\alpha_*)$, as desired. 

Also note that $j_*$ is designed so that when the algorithm terminates $\alpha_u^{(j_*+1)} - \alpha_\ell^{(j_*+1)} \leq \epsilon$. Furthermore, all calls to $\oracle(\varsigma)$ made by the algorithm satisfy $\varsigma \in [\theta_\ell, L^{(i_*)}]$ and $\alpha^{(j_*+1)} \in [L^{(i_*-1)}, L^{(i_*)}] \subseteq [L^{(i_*)}/2, L^{(i_*)}]$. Thus, all calls to $\oracle(\varsigma)$ made by the algorithm satisfy $\varsigma \in [\theta_\ell, 2\alpha_*]$ as desired. Finally, the number of queries made is $O(\log(\theta_r/\theta_\ell) + \max\{1,O(\log((L^{(i_*)} - L^{(i_* - 1)}) / \epsilon))\}$. Since $L^{(i_*)} - L^{(i_* - 1)} \leq 2 L^{(i_*-1)}\leq 2\theta_r$, the overall query bound holds. 
 \end{proof}

\subsection{$\MDMP$ Implementation}\label{sec:mgdamo-sub}

Here, we introduce our $\epsilon$-$\MDMP$ implementation $\MDMPImp$ (Algorithm~\ref{alg:DMP-implementation-matrix-games}) which is parameterized by $\epsilon > 0$ (other parameters are discussed in the subsequent paragraph). Note that $\MDMPImp$ essentially reduces the implementation of a $\epsilon$-$\MDMP$ to computing approximate solutions of a sequence of constrained prox multi-point problems (recall Definition~\ref{def:subproblem}). 

%

$\MDMPImp$ essentially has two major components. The first component is a subroutine $\Validate$ which is designed to (approximately) identify whether an inputted value of $\alpha$ satisfies the movement guarantee $\breg{\cU}{\prox_\cU^\alpha(\nabla_\pm f_A; \cB^{\cS}_{\iota(5), \zground})} = \Theta(\alpha^\rho)$ (where $\zground$ is the point defined in Line~\ref{line:zground}) by invoking the $\rho$-approximate solution oracle $\OAPPROX$. $\Validate$ also accesses and updates the global matrix approximation path $\cP$ passed as input to $\MDMPImp$. 

The second component of $\MDMPImp$ is an invocation of $\cautiousSearch$ to compute the desired output for Definition~\ref{def:MDMP}. In particular, note that $\MDMPImp$ is parameterized by $\beta, \rho > 0$. Here, $\beta$ and $\rho$ control the \emph{kineticness} of the resulting $\MDMP$ is (recall Definition~\ref{def:MDMP}). Furthermore, $\MDMPImp$ instantiates $\cautiousSearch$, passing the subroutine $\Validate(\cdot)$ as the underlying $\oracle$ and range lower and upper bounds $\beta$ and $\theta_r$ respectively. 

In the next sections, we discuss and analyze the implementation of $\Validate$ and $\MDMPImp$ in further detail.

\RestyleAlgo{ruled}
\DontPrintSemicolon
\SetKwComment{Comment}{/* }{ */}
\begin{algorithm2e}[h!]
\caption{$\MDMP$ for matrix games implementation $\MDMPImp(\cU, \cP)$}
\label{alg:DMP-implementation-matrix-games}
\KwInput{Finite nonempty multiset $\uset \subseteq \zset$ and a matrix-approximation path $\pathd = \{\Delta_\ell, M_\ell\}_{\ell \in [L]}$ to $z \in \cU$ (Definition~\ref{def:matrix-approx-path})}
\KwParameter{ $\epsilon > 0$, $0 < \beta < \theta_r$, and a $\rho$-approximate solution (AS) oracle $\OAPPROX$ for $\cS$ (Definition~\ref{def:oracle})}
%
%
%
%
%
%
%
%
%
%
%
%
%
\BlankLine
Define $\epsilon'$ as in \eqref{eq:epsilonprime}\; 
$(z_*, \alpha_*) \gets \cautiousSearch(\epsilon', \beta, \theta_r, \Validate(\cdot)) \label{line:bsearch}$ 
\tcp*{Algorithm~\ref{alg:cautious}}
\Return{$(z_*, \alpha_*, \cP)$}
\BlankLine
\Function{$\Validate(\alpha)$}{
     %
     \tcp{Such a $\delta$ always exists because $\iota$ is strictly increasing over $\R_{>1}$}
     Set $\delta > 0$ so that $(1+\delta)^2 \cdot \iota(3) < (1+\delta) \cdot \iota(4) < \iota(5)$ and either $(1+\delta) \cdot \iota(3) = (\iota(3) + \iota(4))/2$ or else $(1+\delta) \cdot \iota(4) = (\iota(4) + \iota(5))/2$ \label{line:delta2}\; 
     Compute $\tilde{z} \gets \bestresponse(\alpha, \cU) \label{line:bestresponse}$,
     $\zground \gets \argmin_{z \in \cB^{\cS}_{\iota(5), \tilde{z}}} \breg{\cU}{z}$ \label{line:zground}\; 
     \lIf{$\breg{\cU}{\zground} > 3\alpha^\rho$}{\Return{$(0, \oracleFailure) \label{line:safety-check}$}}
     \tcp{Complete $\cP$ into a matrix-approximation path to $\zground$ by adding an additional term with a null model}
    $\Delta_{L+1} \gets \ground{A}{\zground} - \ground{A}{z}$ and $M_{L+1} \gets 0_{m \times n}$ \label{line:null-model}\;  
     $(z', \{\Delta_{\ell}, M'_{\ell}\}_{\ell \in [L+1]}) \gets \OAPPROX(\cU, \iota(5)^2, \alpha, \zground, \{\Delta_{\ell}, M_{\ell}\}_{\ell \in [L+1]}, \epsilon, \delta)$ \label{line:wrap}\; 
     $\cP \gets \{\Delta_\ell, M'_{\ell}\}_{\ell \in [L]}$ \label{line:inplace} \tcp*{Update $\cP$ in-place}
     \lIf{$z' \notin \cB^{\cS}_{\iota(4) \cdot \iota(5), \zground}$}{\Return{$(0, \oracleFailure)$ \label{line:too-big-2}}} 
     \lIf{$\breg{\cU}{z'} > 2.5\alpha^\rho$}{\Return{$(0, \oracleFailure)$ \label{line:too-big-3}}}
     \lElse{\Return{$(z', \oracleSuccess)$ \label{line:too-small}}}
     %
}
\end{algorithm2e}

\subsection{$\MDMP$ analysis}\label{sec:analysis-of-implementation}

Here, we analyze $\MDMPImp$ (Algorithm~\ref{alg:DMP-implementation-matrix-games}). First, we analyze the  $\Validate$ subroutine Algorithm~\ref{alg:DMP-implementation-matrix-games}. $\Validate$ accepts an $\alpha > 0$ (and, implicitly accesses the global variables $\cU, \cP$). As we will show, this subroutine returns $(\cdot, \oracleSuccess)$ only if $\breg{\cU}{\prox_\cU^\alpha(\nabla_\pm f_A; \cZ)} > 2.4\alpha^\rho$ and $(\cdot, \oracleFailure)$ only if $\breg{\cU}{\prox_\cU^\alpha(\nabla_\pm f_A; \cZ)} \leq 2.6\alpha^\rho$ (as we prove in Lemma~\ref{lemma:validate-correctness}.) Note that by Lemma~\ref{lemma:cautious_search}, this ensures that the $\alpha_*$ returned by $\MDMPImp$ satisfies $\breg{\cU}{\prox_\cU^{\alpha_*}(\nabla_\pm f_A; \cZ)} \leq 2.6 \alpha_*^2$, as described in Section~\ref{sec:overview-of-approach}.

In Line~\ref{line:bestresponse}, the algorithm first computes the mapping $\bestresponse(\alpha, \cU)$ and then selects a point $\zground \in \cB^{\cS}_{C, \tilde{z}}$ in Line~\ref{line:zground} which minimizes the sum of divergences from points in $\cU$. In the case that this sum of divergences is too large, then the algorithm returns $(0, \oracleFailure)$ in Line~\ref{line:safety-check}. This check is included for two reasons. First, if $\breg{\cU}{\zground} > 3\alpha^\rho$, then, as we prove in the following Lemma~\ref{lemma:validate-correctness}, this immediately implies $\breg{\cU}{\prox_\cU^{\alpha}(\nabla_\pm f_A; \cZ)} > 2.4 \alpha^\rho$ (consequently, $\Validate(\alpha)$ must return $(\cdot, \oracleFailure)$. Second, in our eventual application, invoking the approximate solution oracle $\OAPPROX$ on such a $\zground$ as in Line~\ref{line:wrap} might require many matvecs. To avoid needlessly exceeding the matvec budget in this case, Line~\ref{line:safety-check} returns ``early'' without ever invoking $\OAPPROX$, ensuring that the algorithm will never call $\OAPPROX$ on a point $\zground$ which is too far (in the sense of divergence) from $\cU$. 

On the other hand, if $\breg{\cU}{\zground} \leq 3\alpha^\rho$, the algorithm completes $\cP$ (which is a matrix approximation path to $z$) into a matrix-approximation path to the selected $\zground$ and computes an $(\epsilon, \delta, \rho)$-approximate solution of a constrained prox multi-point problem (Definition~\ref{def:subproblem}) in Line~\ref{line:wrap} using the oracle $\OAPPROX$. Next, the algorithm updates the matrix approximation path $\cP$ in Line~\ref{line:inplace} in-place. Finally, the algorithm returns depending on a variety of conditions on $z'$ in Lines~\ref{line:too-big-2},~\ref{line:too-big-3} or~\ref{line:too-small}. These return conditions are tailored to enable the following correctness guarantee. 

\begin{lemma}[$\Validate$~correctness guarantee]\label{lemma:validate-correctness}For any $\alpha, \epsilon > 0$, finite nonempty multiset $\cU \subseteq \cZ$, matrix-approximation path $\cP$ to $z \in \cU$, and $\beta > 0$, letting $z^\star_\alpha \defeq \prox_{\cU}^{\alpha}(\gm f_A; \cZ)$ and $(z', \flag) \gets \Validate(\alpha)$, we have that 
\begin{align*}
        \flag = \begin{cases}
        \oracleFailure, & \text{only if } \breg{\cU}{z^\star_\alpha} > 2.4\alpha^\rho, \\
        \oracleSuccess, & \text{only if } \breg{\cU}{z^\star_\alpha} \leq 2.6 \alpha^\rho. 
    \end{cases}
\end{align*}
Furthermore, if $\flag = \oracleSuccess$, then $z^\star_\alpha \in \cB^{\cS}_{\iota(5)^2, \zground}$, $z' \in \cZ$, and
\begin{align}\label{eq:dmp-closeness-guarante}
    \inangle*{ \nabla_\pm f_A(z') + \alpha \nabla \breg{\cU}{z'}, z'-u } \leq \epsilon, \text{ for all } u \in \cZ. 
\end{align}
%
\end{lemma}
\begin{proof} The proof considers all of the return conditions in $\Validate(\alpha)$ and reasoning about the containment of $z^\star_\alpha \in \cB_{\iota(5)^2, \zground}^\cS$ in the case that $\flag=\oracleSuccess$. 
\\\\
\emph{Line~\ref{line:safety-check}:} First, suppose that $\Validate(\alpha)$ returns in Line~\ref{line:safety-check}. Then, suppose, for the sake of contradiction, that $\breg{\cU}{z^\star_\alpha} \leq 2.4\alpha^\rho$. Then, by Definition~\ref{def:best-response-stability}, we have that $z^\star_\alpha \in \cB^{\cS}_{\iota(2.4), \tilde{z}} \subseteq \cB^{\cS}_{\iota(5), \tilde{z}}$ because $\iota(5) > \iota(2.4)$ (recall that $\iota$ is strictly increasing). Consequently, we have that $3\alpha^\rho < \breg{\cU}{\zground} = \min_{z \in \cB^{\cS}_{\iota(5), \tilde{z}}} \breg{\cU}{z} \leq \breg{\cU}{z^\star_\alpha} \leq 2.4 \alpha^\rho$, which is a contradiction. Thus, the lemma holds. 
\\\\
\emph{Line~\ref{line:too-big-2}:} Suppose that $\Validate(\alpha)$ returns in Line~\ref{line:too-big-2}. For notational convenience, let $z_\ell \defeq \prox_{\cU}(\nabla_\pm f_A; \cB^{\cS}_{C^2, \zground})$ (where we follow the convention of \citep{karmarkar2025solvingzerosumgames} and use $\ell$ to denote ``local''). By the properties of $\OAPPROX$ (Definition~\ref{def:oracle} and~\ref{def:approx-solution}), we have that $z' \in \cB^{\cS}_{1+\delta, z_\ell}$. Now, by the fact that $z' \notin \cB^{\cS}_{\iota(4)\cdot \iota(5), \zground}$ and Definition~\ref{def:stable-region-hess}, we must have that
\begin{align*}
    \hess r(z') \prec \frac{1}{\iota(4) \iota(5)} \hess r(\zground) \text{~~or~~}  \hess r(z') \succ \iota(4)\iota(5) \hess r(\zground).
\end{align*}
However, since $z' \in \cB^{\cS}_{1 + \delta, z_\ell}$, we must also have that 
\begin{align*}
    \frac{1}{(1 + \delta)} \hess r(z_\ell) \preceq \hess r(z') \preceq (1 + \delta) \hess r(z_\ell). 
\end{align*}
Consequently, we must have that 
\begin{align*}
    \frac{1}{(1 + \delta)} \hess r(z_\ell) \prec\frac{1}{\iota(4)\iota(5)}\hess r(\zground) \text{~~or~~} {\iota(4)\iota(5)} \hess r(\zground) \prec (1 + \delta) \hess r(z_\ell). 
\end{align*}
Rearranging the above display, we must have that
\begin{align*}
    \hess r(z_\ell) \prec\frac{(1+\delta)}{\iota(4)\iota(5)}\hess r(\zground) \text{~~or~~} \frac{\iota(4)\iota(5)}{(1 + \delta)} \hess r(\zground) \prec \hess r(z_\ell). 
\end{align*}
Now, by the choice of $\delta$ in Line~\ref{line:delta2}, we have that $\iota(3) < \frac{\iota(4)}{(1 + \delta)}.$ 
Thus, $z_\ell \notin \cB^{\cS}_{\iota(3)\iota(5), \zground}$. We claim that this implies $\breg{\cU}{z^\star_\alpha} > 3\alpha^\rho$. 

Indeed, suppose for the sake of contradiction that $\breg{\cU}{z^\star_\alpha} \leq 3\alpha^\rho$. Then, by Definition~\ref{def:best-response-stability}, we would have that $z^\star_\alpha \in \cB^{\cS}_{\iota(3), \tilde{z}}$. However, note that by construction (Line~\ref{line:zground}) we have that $\zground \in \cB^\cS_{\iota(5), \tilde{z}} = \cB^\cS_{\iota(5), \tilde{z}}$ which also implies that $\tilde{z} \in\cB^\cS_{\iota(5), \zground}$ by Definition~\ref{def:best-response-stability}. Thus, by Definition~\ref{def:best-response-stability}, we would have that 
\begin{align*}
    z^\star_\alpha \in \cB^{\cS}_{\iota(3), \tilde{z}} \in \cB^{\cS}_{\iota(3)\iota(5), \zground} \subseteq \cB^{\cS}_{\iota(5)^2, \zground}, 
\end{align*}
where the $\subseteq$ follows because $\iota$ is strictly increasing and hence $\iota(3) < \iota(5)$. But this would imply that $z^\star_\alpha = z_\ell$, which contradicts that $z_\ell \notin  \cB^{\cS}_{\iota(3)\iota(5), \zground}$. Consequently, we must have that $\breg{\cU}{z^\star_\alpha} > 3\alpha^\rho$ and hence the lemma holds. 
\\\\
\emph{Containment of $z^\star_\alpha \in \cB_{\iota(5)^2, \zground}^{\cS}$:} Next, we prove that if the algorithm reaches Line~\ref{line:too-big-3}, then $z^\star_\alpha \in \cB^{\cS}_{\iota(5)^2, \zground}$. Indeed, the algorithm does not return in Line~\ref{line:too-big-2}, then it must be the case that $z' \in \cB^{\cS}_{\iota(4)\cdot \iota(5), \zground}$. By Definition~\ref{def:stable-region-hess}, we must have that 
\begin{align*}
    \frac{1}{\iota(4)\iota(5)} \hess r(\zground) \preceq \hess r(z') \preceq \iota(4)\iota(5) \hess r(\zground). 
\end{align*}
Next, recall that, taking $z_\ell$ as defined above, we have $z' \in \cB^{\cS}_{1 + \delta, z_\ell}$ (due to the the properties of $\OAPPROX$ from Definition~\ref{def:oracle}). Thus,
\begin{align}\label{eq:notbinding}
    \frac{1}{\iota(4)\iota(5) (1+\delta)} \hess r(\zground) \preceq \hess r(z_\ell) \preceq \iota(4)\iota(5) \cdot (1+ \delta) \hess r(\zground). 
\end{align}
Now, by the choice of $\delta$ in Line~\ref{line:delta2}, we have that $\iota(4)(1+\delta) < (\iota(4) + \iota(5))/2$. Thus, by \eqref{eq:notbinding}, the constraint to $\cB^{\cS}_{C^2, \zground}$ in the definition of $z_\ell$ is not binding. Hence, $z^\star_\alpha = z^\ell \in \cB_{C^2, \zground}^{\cS}$.
\\\\
\emph{Lines~\ref{line:too-big-3}:} Next, suppose that the algorithm returns in Line~\ref{line:too-big-3}. Then, by the properties of $\OAPPROX$ (Definition~\ref{def:oracle}) and the fact that $z^\star_\alpha = z^\ell$, we have that $\breg{\cU}{z^\star_\alpha} > 2.5 \alpha^\rho - \alpha^\rho/10 = 2.4 \alpha^\rho$ as required, and the lemma holds.
\\\\
\emph{Lines~\ref{line:too-small}:} Suppose that $\Validate$ returns in Line~\ref{line:too-small}. Then, by the properties of $\OAPPROX$ (Definition~\ref{def:oracle}) and the fact that $z^\star_\alpha = z^\ell$, we must have that $\breg{\cU}{z^\star_\alpha} \leq 2.5\alpha^\rho + \alpha^\rho/10 = 2.6\alpha^\rho$ as required, and the lemma holds.
%
%
\\\\
The final claim now follows by the properties of $\OAPPROX$ (Definition~\ref{def:oracle}).
\end{proof}

%

%

%
%
%
%
%
%
%
%
%
%
%
%
%
%
%
%
%
%
%
%
%
%
%
%
%
%
%
%
%

Next, we combine the analysis of $\Validate$ and $\cautiousSearch$ to analyze Algorithm~\ref{alg:DMP-implementation-matrix-games} and prove that it meets the conditions of Definition~\ref{def:MDMP} under appropriate assumptions. The following theorem shows how to instantiate Algorithm~\ref{alg:DMP-implementation-matrix-games} to implement a kinetic $\epsilon$-$\MDMP$ (Definition~\ref{def:MDMP}).

\begin{theorem}\label{thm:dmp-basic} 
For any finite nonempty multiset $\cU \subseteq \cZ$ and $\alpha > 0$, let $z_\alpha^\star \defeq \prox_\cU^\alpha(\nabla_\pm f_A; \cZ)$ and $h(\alpha) \defeq \breg{\cU}{z^\star_\alpha}$. Suppose that $h$ is $M$-Lipchitz over $[\beta, \theta_r]$, and set 
\begin{align}\label{eq:epsilonprime}
     \epsilon' = \min\left\{\left(1 - \paren{\frac{14}{15}}^{{1}/{\rho}}\right)\beta,\frac{\beta^\rho}{15M}\right\}.  
\end{align}
Suppose further that $h(R)\leq 2.4 \cdot \theta_r^\rho$. Then, $\MDMPImp$ (Algorithm~\ref{alg:DMP-implementation-matrix-games}) is a $(\beta, 2, \rho)$-kinetic $\epsilon$-$\MDMP$ (Definition~\ref{def:MDMP}). Moreover, for any finite nonempty multiset $\cU \subseteq \cZ$ and matrix approximation path $\cP$, letting $(z_*, \alpha_*, \cP') \defeq \MDMPImp(\cU, \cP)$, the algorithm makes at most $O(\log(\theta_r/\min\{\epsilon,\beta\}))$ queries to $\Validate(\alpha)$ where in each query, $\alpha \in [\beta, \min\{2\alpha_*, \theta_r\}]$.
\end{theorem}

We prove this theorem using the following natural monotonicity property \Cref{lem:monotonicity}, which is a generalization of Lemma B.9 of \citep{karmarkar2025solvingzerosumgames}. The proof follows very similarly to the proof of Lemma B.9 of \citep{karmarkar2025solvingzerosumgames} (only mildly modified to handle sums over $\cU$ rather than divergence from a single point).

\begin{lemma}[Generalization of Lemma B.9 of \citep{karmarkar2025solvingzerosumgames}]\label{lem:monotonicity}
Let $(\zset, r)$ denote a dgf setup (Definition~\ref{def:dgf-setup}) with $g : \zset \to \R$ a monotone operator, some $\alpha > \beta > 0$, and let $\cU \subseteq \zset$ be a finite nonempty multiset. Then $\walpha \defeq \prox^\alpha_{\cU}(g)$ and $\wbeta \defeq \prox^{\beta}_{\cU}(g)$ satisfy $\breg{\cU}{\walpha} \le \breg{\cU}{\wbeta}$.
\end{lemma}

\begin{proof}
Applying Definition~\ref{def:proximal-mappings}, we have for all $u, u' \in \zset$:
\begin{align*}
    \inangle*{g(\walpha), (\walpha - u)} &\le \alpha \sum_{v \in \cU}\insquare{\breg{v}{u} - \breg{\walpha}{u} - \breg{v}{\walpha}}, \\
    \inangle*{g(\wbeta), (\wbeta - u')} &\le \beta \sum_{v \in \cU} \insquare{\breg{v}{u'} - \breg{\wbeta}{u'} - \breg{v}{\wbeta}}.
\end{align*}
Setting $u \gets \wbeta$, $u' \gets \walpha$, and using the monotonicity of $g$, yields that
\begin{align*}
    0 &\le \alpha \sum_{v \in \cU} \insquare{\breg{v}{\wbeta} - \breg{\walpha}{\wbeta} - \breg{v}{\walpha}} + \beta \sum_{q \in \cU} \insquare{\breg{v}{\walpha} - \breg{\wbeta}{\walpha} - \breg{v}{\wbeta}} \\
    &= \sum_{v \in \cU} (\alpha - \beta) \breg{v}{\wbeta} + (\beta - \alpha) \breg{v}{\walpha} - \alpha \breg{\walpha}{\wbeta} - \beta \breg{\wbeta}{\walpha}.
\end{align*}
Rearranging the above display,
\begin{align*}
    (\alpha - \beta) \sum_{v \in \cU} \breg{v}{\walpha} \le \sum_{v \in \cU} (\alpha - \beta) \breg{v}{\wbeta} - \alpha \breg{\walpha}{\wbeta} - \beta \breg{\wbeta}{\walpha} \le \sum_{v \in \cU}(\alpha - \beta) \breg{q}{\wbeta}.
\end{align*}
The result follows by dividing through by $(\alpha - \beta) > 0$. 
\end{proof}

With this lemma, we now prove the following theorem.

\begin{proof}[Proof of Theorem~\ref{thm:dmp-basic}] By Lemma~\ref{lemma:cautious_search} together with Lemma~\ref{lemma:validate-correctness}, we have that $\Validate(\alpha_*) = (z_*, \oracleSuccess)$ and $z_*$ satisfies \eqref{eq:dmp-closeness-guarante}. Next we prove that either $\alpha =\beta$ or else $\breg{\cU}{z'} > 2 \alpha^\rho.$

Without loss of generality, suppose that $\alpha > \beta$. Then, by Lemma~\ref{lemma:cautious_search}, $\Validate(\varsigma) = \oracleFailure$ for some $\varsigma \in [\max\{\alpha -\epsilon', \beta\}, \alpha)$. Consequently, by Lemma~\ref{lemma:validate-correctness}, $\breg{\cU}{z^\star_\varsigma} > 2.4 \varsigma^\rho.$ Because $h$ is $M$-Lipschitz, by Lemma~\ref{lem:monotonicity}, we have that
\begin{align*}
    h(\varsigma) - h(\alpha) \leq M (\alpha - \varsigma), 
\end{align*}
and consequently, 
\begin{align*}
    h(\alpha) &\geq h(\varsigma) - M(\alpha - \varsigma)  \geq 2.4 (\alpha - \epsilon')^\rho - M\epsilon' \\
    &\geq 2.4 \paren{\paren{\frac{14}{15}}^{1/\rho} \alpha}^\rho - M\epsilon' \\
    &\geq 2.4 \paren{\frac{14}{15} \alpha^\rho} - \frac{\alpha^\rho}{15} \geq 2.1 \alpha^\rho,
\end{align*}
where we used that that 
\begin{align*}
     \epsilon' = \min\left\{\left(1 - \paren{\frac{14}{15}}^{{1}/{\rho}}\right)\beta,\frac{\beta^\rho}{15M}\right\} \leq \min\left\{\left(1 - \paren{\frac{14}{15}}^{{1}/{\rho}}\right)\alpha,\frac{\alpha^\rho}{15M}\right\}. 
\end{align*}
Finally, by the properties of $\OAPPROX$ (Definition~\ref{def:oracle}), we have that 
\begin{align*}
    \breg{\cU}{z'} > 2.1\alpha^\rho - \alpha^\rho/10 = 2 \alpha^\rho. 
\end{align*}
Lastly, the query complexity bounds follow immediately from Lemma~\ref{lemma:cautious_search}. 
\end{proof}

%

%
%
%

%
%
%
%
%
%
%
%
%
%
%
%
%
%
%
%
%
%


%
%
%
%
%
%
%
%
%
%
%
%
%
%
%
%
%
%
%
%
%

%


%
%
%
%
%
%
%
%
%
%
%
%
%
%
%
%
%
%
%
%
%
%
%
%
%
%
%

%

%
%
%
%
%
%
%
%
%
%
%
%
%
%
%
%
%
%
%
%
%
%
%
%
%
%
%
%
%
%
%
%

%
%
%
%
%
%
 
%

\section{Smooth-until-proven-guilty solver}\label{sec:sug-solver}

Recall that the $\MDMPImp$ algorithm presented in Section~\ref{sec:MDMP-implementation} assumed access to a $\rho$-approximate solution oracle (Definition~\ref{def:oracle}) $\OAPPROX$. In this section, we adapt the framework from \citep{karmarkar2025solvingzerosumgames} towards implementing a $\rho$-approximate solution oracle for constrained prox multi-point problems (Definition~\ref{def:subproblem}) for our applications to $\ell_1$-$\ell_1$ and $\ell_2$-$\ell_2$ matrix games. 

In fact, in this section, we show a somewhat more general result. In all of our applications (due to geometric properties of our specific setups for $\ell_1$-$\ell_1$ and $\ell_2$-$\ell_1$ matrix games), in order to compute an approximate solution (Definition~\ref{def:approx-solution}) it suffices to compute a (possibly) weaker solution concept, which we term a \emph{divergence-bounded solution.}

\begin{definition}[Divergence-bounded solution]\label{def:divergence-bounded-solution}\label{def:divergence-bounded}
For $\epsilon \geq 0$, letting $z^\star$ be the solution to the $(\cU, c, \alpha, z, \cS)$-constrained prox multi-point problem (Definition~\ref{def:subproblem}), we say that a point $z' \in \cB^{\cS}_{c, z}$ is an \emph{$\epsilon$-divergence-bounded solution} to the problem if $\breg{z^\star}{z'} \leq \epsilon$. 
\end{definition}

Correspondingly, we define a divergence-bounded solution oracle. 

\begin{definition}\label{def:divergence-bounded-oracle} A \emph{divergence-bounded solution oracle} $\ODB$ (for a dgf setup $\cS = (\cZ \subset \R^d, r)$) takes in a finite non-empty multiset $\cU \subseteq \cZ$, $c > 1$, $\alpha >0$, $z \in \cZ$, a matrix approximation path $\cP = \{\Delta_\ell, M_\ell\}_{\ell \in [L]}$ to $z$ (Definition~\ref{def:matrix-approx-path}), and $\epsilon \geq 0$ and returns $(z', \cP' = \{\Delta_\ell, M'_\ell\}_{\ell \in [L]})$, where $z'$ is an $\epsilon$-divergence-bounded solution to the $(\cU, c, \alpha, z, \cS)$-constrained prox multi-point problem (Definition~\ref{def:divergence-bounded}) and $\cP'$ is a matrix approximation path to $z$.
\end{definition}

Under the following definition of \emph{robustness}, access to a divergence-bounded solution oracle is sufficient to implement an approximate-solution oracle (Definition~\ref{def:oracle}) and consequently is sufficient to instantiate an $\MDMP$ (Definition~\ref{def:MDMP}) as described in Section~\ref{sec:MDMP-implementation} (Algorithm~\ref{alg:DMP-implementation-matrix-games}).

%
%

\begin{definition}[Robustness]\label{def:robustness} For $\epsilon, \delta, \kappa \geq 0$ and $\rho > 0$, a dgf setup $\mathcal{S} = (\mathcal{Z} \subset \R^d, r)$ is $(\epsilon, \delta, \rho, \kappa)$-\emph{robust} if for every constrained prox multi-point problem $(\cU, c, \alpha, z, \cS)$ (Definition~\ref{def:subproblem}), every $\kappa$-divergence-bounded solution (Definition~\ref{def:divergence-bounded}) is also an $(\epsilon, \delta, \rho)$-approximate solution (Definition~\ref{def:approx-solution}). 
\end{definition}

In particular, the following condition is sufficient to ensure robustness. 
\begin{lemma}[Sufficient conditions for robustness]
    \label{lem:sufficient-cond-for-robustness}
    Suppose that for every $(\cU, c, \alpha, z, \cS = (\cZ \subset \R^d, r))$-constrained prox multi-point problem (Definition~\ref{def:subproblem}) letting $z^\star$ denote its solution, any $z' \in \cB_{c, z}^{\cS}$ with $\breg{z^\star}{z'} \leq \kappa$ satisfies 
\begin{itemize}
    \item $\abs{\breg{\cU}{z'} - \breg{\cU}{z^\star}} < \alpha^\rho/10$, 
    \item $z' \in \cB^{\cS}_{1+\delta, z^\star}$, and
    \item if $\prox_\cU^\alpha(\nabla_\pm f_A; \cZ) \in \cB_{c, z}^{\cS}$ then $\inangle*{\nabla_{\pm}f_A(z') + \alpha \grad \breg{\cU}{z'}, z' -u} \leq \epsilon, \text{ for all } u \in \cZ$. 
\end{itemize}
Then, $\cS$ is $(\epsilon, \delta, \rho, \kappa)$-robust.
\end{lemma}
\begin{proof} The proof is immediate from Definitions~\ref{def:approx-solution} and~\ref{def:divergence-bounded}. 
\end{proof}

In Section~\ref{sec:applications}, we show that for $\kappa$ scaling polynomially in $1/\epsilon, 1/\delta$ and the problem parameters (namely, $m, n$), the preconditions of Lemma~\ref{lem:sufficient-cond-for-robustness} and consequently the robustness condition (Definition~\ref{def:robustness}) is met in our applications. Thus, for our applications, the methods in this section suffice to implement an approximate solution oracle as required in Section~\ref{sec:MDMP-implementation}. Consequently, in this section, we discuss how to implement a divergence-bounded solution oracle (Definition~\ref{def:divergence-bounded-oracle}).

\paragraph{Assumptions.} %
In the remainder of this section, we fix arbitrary dgf setups $\dgfsetup\x = (\xset \subset \R^n, \rx)$ and $\dgfsetup\y = (\yset \subset \R^m, \ry)$ with $\dgfsetup = (\zset \subset \R^d, r) \defeq \prodsetup(\dgfsetup\x, \dgfsetup\y)$ (recall Definition~\ref{def:product-dgf-setups}). Moreover, we assume that $\cZ$ is $\pi$-locally bounded in the sense of the following definition. 


\begin{definition}[$\pi$-locally bounded]\label{def:consistency} We say the dgf setup $(\zset \subset \R^d, r)$ is \emph{$\pi$-locally bounded} for $\pi : \R_{>1} \to \R_{>0}$ if for any $z \in \cZ$, $z', z'' \in \cB_{c, z}^{\cS}$, and $c > 1$ we have that $\breg{z'}{z''} \geq \pi(c) \normInline{z' - z''}_{z}$.
\end{definition}

Additionally, we let $\Gamma_\cS$ denote an upper bound on the range of $r$ so that $\sup_{z \in \cZ} \breg{z'}{z} \leq \Gamma_\cS$ for $z' \defeq \argmin_{z\in\cZ} r(z)$. 

In the remainder of this Section~\ref{sec:SUG-solver-prelims}, we discuss a simple linear algebraic sub-routine, which we term a $\judge$ as in \citep{karmarkar2025solvingzerosumgames}. This $\judge$ subroutine shows how we update the matrix approximation path $\cP$ and enables complexity analysis as a function of $\size(\cP) -\size(\cP')$. In Section~\ref{sec:step} we generalize the smooth until proven guilty mirror prox steps from \citep{karmarkar2025solvingzerosumgames}, which enables our implementation of a divergence-bounded solution oracle (Definition~\ref{def:divergence-bounded-oracle}). Finally, in Section~\ref{sec:solver} we describe our approximate solution oracle. This section is largely motivated by Section 5 of \citep{karmarkar2025solvingzerosumgames} and leverages similar techniques to their prior work; however, to handle our general setups and use of matrix approximation paths, we require several slight modifications.


\subsection{The smooth-guilty judge}\label{sec:SUG-solver-prelims} 

Here we describe our notion of a $\judge$ subroutine, which is inspired by the $\judge$ subroutine in \citep{karmarkar2025solvingzerosumgames} but is appropriately adapted to our setting of working with matrix approximation paths. The input to $\judge$ is a center point $\zground \in \cZ$, a matrix-approximation path $\cP$ to $\zground$, a parameter $\tau > 0$ (which we call a \emph{smoothness threshold} as in \citep{karmarkar2025solvingzerosumgames}), and two vectors $z, z' \in \cZ$. The $\judge$ subroutine ``judges'' whether the vector $z$ or $z'$ reveals a $\tau$-large singular direction along the matrix-approximation path $\cP$. This is formalized in the following pseudocode (Algorithm~\ref{alg:judge}), where we use $\normalize(z): z \mapsto z/\normInline{z}_2$ to be the mapping which takes any vector $z \in \R^d$ to a unit vector in the direction of $z$.   

\RestyleAlgo{ruled}\label{alg:judge}
\SetKwComment{Comment}{/* }{ */}
\begin{algorithm2e}[ht]
\caption{$\judge(\cP, \tau, z, z')$}
\KwInput{Matrix-approximation path $\cP = \{\Delta_\ell, M_\ell\}_{\ell \in [L]}$, smoothness threshold $\tau > 0$, $z, z' \in \cZ$.}
\For{$\bar{z} \in \{z, z'\}$}{
    \tcp{If we find a $\tau$-large singular direction, return a $\guilty$ verdict and update $\cP$}
    \If{
    $\inangle*{\bar{z}\y, \sum_{t \in [L]}(\Delta_\ell - M_\ell) \bar{z}\x} > \tau \normInline{\bar{z}\y}_2\normInline{\bar{z}\x}_2$ \label{line:guilty}
    }{
    $v \gets \normalize(\bar{z}\y)$ and  $u \gets \normalize(\bar{z}\x)$ \label{line:normalize}\; 
    \lFor(\tcp*[f]{Update the $\ell$-th model in $\cP$}){$\ell \in [L]$}{   
       $M_\ell \gets M_\ell + \inangle*{v, (\Delta_\ell - M_\ell) u} \cdot vu^\top$\label{line:update}}
    \Return{$(\guilty, \cP)$}
    }
}
\Return{$(\smooth, \cP)$}
\end{algorithm2e}

To analyze, $\judge$ (Algorithm~\ref{alg:judge}), we use the following property of the Frobenius norm.

\begin{lemma}[Lemma C.1 of 
\citep{karmarkar2025solvingzerosumgames}%
]\label{lemma:frobenius}  $\normInline{A-B}_F^2 \leq \normInline{A}_F^2 - \inangle*{v, A u}^2$ for any $A \in \R^{m \times n}$, unit vectors $u \in \R^n$ and $v \in \R^m$ (i.e., $\normInline{u}_2 = \normInline{v}_2 = 1$), and $B =  \inangle*{v, A u} \cdot vu^\top$.
\end{lemma}
%
%
%
%
%
%

With \Cref{lemma:frobenius}, we can analyze the $\judge$ subroutine (Algorithm~\ref{alg:judge}). 

\begin{lemma}\label{lemma:judge-guarantee} Let $\cP = \{\Delta_\ell, M_\ell\}_{\ell \in [L]}$ be a matrix-approximation path, $\tau >0$ be a smoothness threshold, and $z, z' \in \cZ$. Then $(\verdict, \cP') \gets \judge(\cP, \tau, z, z')$ can be implemented with $O(L)$ matvecs to $A$ and satisfies the following: \begin{itemize}
    \item If $\verdict = \smooth$ then $\cP' = \cP$, 
\begin{align*}
    \inangle*{z\y, \sum_{\ell \in [L]} (\Delta_\ell - M_\ell) z\x} \leq \tau \normInline{z\x}_2\normInline{z\y}_2, ~~\text{ and }~~ \inangle*{z'\y, \sum_{\ell \in [L]} (\Delta_\ell - M_\ell) z'\x} \leq \tau \normInline{z'\x}_2 \normInline{z'\y}_2. 
\end{align*} 
\item If $\verdict = \guilty$ then $\cP' = \{(\Delta_\ell, M'_\ell)\}_{\ell \in [L]}$ is a matrix-approximation path to $\zground$ such that $\size(\cP') \leq \size(\cP) - \tau^2/L$. 
\end{itemize}
\end{lemma}

\begin{proof} From the pseudocode, it is easy to verify that if $\verdict = \smooth$, then the first bullet holds. Thus, it remains to prove the second bullet. 

First, observe that $\cP'$ is a matrix-approximation path to $\zground$ because each $M'_\ell$ is known explicitly after each update in Line~\ref{line:update}. Next, to analyze $\size(\cP')$, note that $\verdict = \guilty$ ensures that the if statement in Line~\ref{line:guilty} executes for a $\bar{z} \in \{z, z'\}$ and for this value of $\bar{z}$,
\begin{align}
\label{eq:iftrue}
    \inangle*{\bar{z}\y, \sum_{\ell \in [L]} (\Delta_\ell - M_\ell) \bar{z}\x} > \tau \normInline{\bar{z}\y}_2 \normInline{\bar{z}\x}_2\,.
\end{align}
Consequently, $\bar{z}\x \neq 0_n$ and $\bar{z}\y \neq 0_m$. Rescaling \eqref{eq:iftrue} and using the definition of $u$ and $v$ then yields
\begin{align*}
    \tau < 
    \sum_{\ell \in [L]}\inangle*{\frac{\bar{z}\y}{\normInline{\bar{z}\y}_2}, (\Delta_\ell - M_\ell) \frac{\bar{z}\x}{\normInline{\bar{z}\x}_2}} =  
    \sum_{\ell \in [L]}\inangle*{v, (\Delta_\ell - M_\ell) u}. 
\end{align*}
Applying the Cauchy-Schwarz inequality then yields that 
\begin{align}\label{eq:cauchy-schwarz}
    \frac{\tau^2}{L} <
    \left(\frac{1}{L} \sum_{\ell \in [L]}\inangle*{v, (\Delta_\ell - M_\ell) u}\right)^2
    \leq
    \sum_{\ell \in [L]}\inangle*{v, (\Delta_\ell - M_\ell) u}^2
\end{align}
Now, using Lemma~\ref{lemma:frobenius} to reason about the updates in Line~\ref{line:update}, we can conclude that for each $\ell \in [L]$, 
\begin{align*}
    \normInline{\Delta_\ell - M'_\ell}_F^2 \leq \normInline{\Delta_\ell - M_\ell}_F^2 - \inangle*{v, (\Delta_\ell - M_\ell) u}^2. 
\end{align*}
By \eqref{eq:cauchy-schwarz}, it follows that 
\begin{align*}
    \size(\cP') &= \sum_{\ell \in [L]} \normInline{\Delta_\ell - M'_\ell}_F^2 \leq \sum_{\ell \in[L]} \paren{\normInline{\Delta_\ell - M_\ell}_F^2 - \inangle*{v, (\Delta_\ell - M_\ell) u}^2} 
    \leq \size(\cP) - \frac{\tau^2}{L}. 
\end{align*}
Finally, to justify the query complexity, note that the if statement in Line~\ref{line:guilty} requires $O(L)$ matvecs to $A$ while each iteration of Line~\ref{line:update} requires $O(1)$ matvecs to $A$. 
\end{proof}

In some cases, there are alternative implementations of the $\judge$ routine which satisfy the guarantees of Lemma~\ref{lemma:judge-guarantee} (see Section 6.5.2 and Appendix C of \citep{karmarkar2025solvingzerosumgames}); however, we focus on this implementation, as it is particularly simple.

%
%
%
%
%
%
%
%

%
%

%
%
%
%
%
%
%
%

%
%
%
%
%
%
%
%

%
%
%
%


%

%
%
%
%
%
%
%
%
%
%
%
%

%

%
%
%
%
%
%
%
%
%

\subsection{Smooth until proven guilty composite mirror prox}\label{sec:step}

Here we adapt the smooth until proven guilty composite mirror prox algorithm of \citep{karmarkar2025solvingzerosumgames} to our framework with path approximations. This adaptation (Algorithm~\ref{alg:mirror-prox-step}) enables us to implement a divergence-bounded solution oracle (Definition~\ref{def:oracle}). The reader might also find it helpful to refer to Definition~\ref{def:product-dgf-setups} for  a reminder of the notation used in Lines~\ref{line:unground} and~\ref{line:unground2}. The following \Cref{lemma:step-guarantee} 
provides the main guarantee of $\Step$. 

\RestyleAlgo{ruled}\label{alg:mirror-prox-step}
\SetKwComment{Comment}{/* }{ */}
\begin{algorithm2e}[ht]
\caption{$\Step(\cU, \cP, c, \alpha, \zground, z)$}
\KwInput{$\cU, c, \alpha, \zground$ as in Definition~\ref{def:subproblem}, a matrix approximation path $\cP = \{\Delta_\ell, M_\ell\}_{\ell \in [L]}$ to $\zground$ and a $z \in \cB_{c, \zground}^{\cS}$.}
\KwParameter{ A smoothness threshold $\tau > 0$}
$B \gets \paren{\sum_{\ell \in [L]} (\Delta_\ell - M_\ell)}_{\zground, *}$ \label{line:unground} \tcp*{$B$ is the unknown portion of $A$ (the subtraction is done implicitly)}
$C \gets \paren{\sum_{\ell \in [L]} M_\ell}_{\zground, *}$ \label{line:unground2}\tcp*{$C$ is the explicitly known portion of $A$} 
$\psi \gets \alpha \nabla \breg{\cU}{\cdot} + \nabla_\pm f_C$ \; 
$w \gets \prox_{z}^\tau(\nabla_\pm f_{B}(z) + \nabla_\pm \psi; \cB_{c, \zground}^{\cS})$ \label{line:first-step-stronglymonotone}\;
$z' \gets \prox_{z, w}^{\tau, \alpha}\paren{ (\nabla_\pm f_{B}  + \psi)(w); \cB_{c, \zground}^{\cS}}$ \label{line:second-step-stronglymonotone}\;
$z_1 \gets (w\x - z'\x, w\y - z\y)$, $z_2 \gets (z\x -w\x, w\y - z'\y)$ \; 
$z_{(1)} \gets (z_1)_{\zground}$, $z_{(2)} \gets (z_2)_{\zground}$\; 
$(\verdict, \cP') \gets \judge(\cP, 2\pi(c) \cdot \tau, z_{(1)}, z_{(2)})$ \tcp*{$\pi(c)$ as defined in Definition~\ref{def:consistency}}
\Return{$(z', \verdict, \cP')$}
\end{algorithm2e}

\begin{lemma}\label{lemma:step-guarantee} Let $z^\star \defeq \prox_\cU^\alpha(\nabla_\pm f_A; \cB_{c, \zground}^{\cS})$, $\tau > 0$, $c > 1$, $z \in \cB_{c, \zground}^{\cS}$, and $(z', \verdict, \cP') \gets \Step(\cU, \cP, c, \alpha, \zground, z)$. Then, $\cP'$ is a matrix-approximation path to $\zground$, $\size(\cP') \leq \size(\cP)$, and either 
\begin{itemize}
    \item $\verdict = \smooth$ and $\breg{z^\star}{z'} \leq \paren{1 + \frac{\alpha}{\tau}}^{-1} \breg{z^\star}{z}$, or else 
    \item $\verdict = \guilty$ and $\size(\cP') \leq \size(\cP) - (2\pi(c) \cdot \tau)^2/L$. 
\end{itemize}
The algorithm can be implemented with $O(L)$ matvecs to $A$.  
\end{lemma}

Our proof of Lemma~\ref{lemma:step-guarantee} uses the following technical lemma. 

\begin{lemma}\label{lemma:standalone-guarantee} If $c, \zground, z, w, z'$, $B$, $z_1, z_2, z_{(1)}$, $z_{(2)}$ are as in the pseudocode of Algorithm~\ref{alg:mirror-prox-step} and 
\begin{align*}
    \inangle*{w\y -z\y, B (w\x - z'\x)} + \inangle*{w\y - z'\y, B (z\x - w\y)} > \tau (\breg{w}{z'} + \breg{z}{w})\,,
\end{align*}
then, $\verdict = \guilty$. 
\end{lemma}
\begin{proof} By Fact~\ref{lemma:ungrounding}, we have
\begin{align*}
    &\inangle*{w\y -z\y, B (w\x - z'\x)} + \inangle*{w\y - z'\y, B (z\x - w\y)} \\
    =& \inangle*{{z_{(1)}}\y, \ground{B}{\zground} {{z_{(1)}}}\x} + \inangle*{{z_{(2)}}\y, \ground{B}{\zground} {{z_{(2)}}}\x} \\
    =& \inangle*{{z_{(1)}}\y, \sum_{\ell \in [L]} (\Delta_\ell - M_\ell) {{z_{(1)}}}\x} + \inangle*{{z_{(2)}}\y, \sum_{\ell \in [L]} (\Delta_\ell - M_\ell) {{z_{(2)}}}\x}. 
\end{align*}
Now, by Definition~\ref{def:best-response-stability} and Lemma~\ref{lemma:ungrounding}, we have 
\begin{align*}
   (\breg{w}{z'} + \breg{z}{w}) &\geq \pi(c) \cdot \paren{ \normInline{w-z'}_{\zground}^2 + \normInline{w-z}_{\zground}^2 } \\
    &= \pi(c) \cdot \paren{ \norm{\ground{w-z'}{\zground}}_2^2 + \norm{\ground{w-z}{\zground}}_2^2 }. 
\end{align*}
Thus, by splitting into the components in $\cX$ and $\cY$, we have 
\begin{align*}
    &(\breg{w}{z'} + \breg{z}{w}) \\
    \geq& \pi(c) \cdot \paren{ \norm{\ground{w-z'}{\zground}}_2^2 + \norm{\ground{w-z}{\zground}}_2^2 } \\
   =& \pi(c) \cdot \paren{ \norm{ {\ground{w-z'}{\zground}}\x }_2^2 + \norm{ {\ground{w-z}{\zground}}\x}_2^2 + \norm{ {\ground{w-z'}{\zground}}\y }_2^2 + \norm{ {\ground{w-z}{\zground}}\y}_2^2} \\
   \geq& 2\pi(c) \cdot \paren{ \norm{ {\ground{w-z'}{\zground}}\x }_2 \norm{ {\ground{w-z}{\zground}}\y}_2 + \norm{ {\ground{w-z}{\zground}}\x}_2 \norm{ {\ground{w-z'}{\zground}}\y }_2 } \\
   \geq& 2\pi(c) \cdot \paren{\norm{{z_{(1)}}\x}_2 \norm{{z_{(1)}}\y}_2 + \norm{{z_{(2)}}\x}_2 \norm{{z_{(2)}}\y}_2}, 
\end{align*}
where the second-to-last step used that for any $a, b \geq 0$ we have $a^2 + b^2 \geq 2ab$. Thus, we must have that either 
\begin{align*}
    \inangle*{{z_{(1)}}\y, \sum_{\ell \in [L]} (\Delta_\ell - M_\ell) {{z_{(1)}}}\x}  > 2\pi(c) \cdot \tau \cdot \norm{{z_{(1)}}\x}_2 \norm{{z_{(1)}}\y}_2, 
\end{align*}
or else 
\begin{align*}
    \inangle*{{z_{(2)}}\y, \sum_{\ell \in [L]} (\Delta_\ell - M_\ell) {{z_{(2)}}}\x} > 2\pi(c) \cdot \tau \cdot \norm{{z_{(2)}}\x}_2 \norm{{z_{(2)}}\y}_2. 
\end{align*}
Consequently, by Lemma~\ref{lemma:judge-guarantee}, we must have that $\verdict = \guilty$. 
\end{proof}

We now prove Lemma~\ref{lemma:step-guarantee}. The proof is very similar to the proof of Lemma 5.6 in \citep{karmarkar2025solvingzerosumgames} (and perhaps other well-known proofs of strongly monotone mirror prox). The main difference relative to the proof of Lemma 5.6 of \citep{karmarkar2025solvingzerosumgames} is that our version needs to handle prox steps with regularization to each $u \in \cU$, whereas the version in \citep{karmarkar2025solvingzerosumgames} considered only regularization with respect to a single point. 

\begin{proof}[Proof of Lemma~\ref{lemma:step-guarantee}, adapted from Proof of Lemma 5.6 of \citep{karmarkar2025solvingzerosumgames}] If $\verdict = \guilty$, then the second bullet holds due to Lemma~\ref{lemma:judge-guarantee}. Thus, suppose that $\verdict = \smooth$ and observe that by Lemma~\ref{lemma:standalone-guarantee} we have that $A = B + C$. Consequently, by Lemma~\ref{lemma:standalone-guarantee}, we have that 
\begin{align}
\begin{split}\label{eq:lipschitz-condition-holds-case}
    \inangle*{\nabla_\pm \bilinear{B}(w) - \nabla_\pm \bilinear{B}(z), w - z'} &= \inangle*{w\y -z\y, B w\x - z'\x} + \inangle*{w\y - z'\y, B z\x - w\y} \\
    &\leq \tau (\breg{w}{z'} + \breg{z}{w}). 
\end{split}
\end{align}

Next, we apply the optimality conditions from Definition~\ref{def:proximal-mappings} to each of the composite proximal steps (Lines~\ref{line:first-step-stronglymonotone} and~\ref{line:second-step-stronglymonotone}). We have that for all $u, u' \in \cB_{c, \zground}^{\cS}$,  
\begin{align*}
\begin{split}
    \inangle*{\nabla_\pm f_B(z), w-u'} + \inangle*{\psi(w), w-u'} &\leq \tau \paren{\breg{z}{u'} - \breg{w}{u'} - \breg{z}{w}}\,\\ 
    \inangle*{\nabla_\pm f_B(w), z'-u} + \inangle*{\psi(w), z' - u} &\leq \tau \paren{\breg{z}{u} - \breg{z'}{u} - \breg{z}{z'}} + \alpha \paren{ \breg{w}{u} + \breg{z'}{u}}, 
\end{split}
\end{align*}
(the second line used non-negativity of the Bregman divergence). Setting $u' = z'$ in the above display and summing both equations, we have, for all $u \in \cB_{c, \zground}^{\cS}$,
\begin{align*}
    &\inangle*{\nabla_\pm f_B(z), w-z'} + \inangle*{\nabla_\pm f_B(w), z'-u} +  \inangle*{\psi(w), w-u}  \\
    &\leq \tau \paren{\breg{z}{u} - \breg{z'}{u} - ( V_{w}^r(z') + \breg{z}{w} )} + \alpha \paren{ \breg{w}{u} + \breg{z'}{u}}. 
\end{align*}
Dividing through the above display and \eqref{eq:lipschitz-condition-holds-case} by $\tau$, we find that for all $u \in \cB_{c, \zground}^{\cS}$,
\begin{align*}
    &\frac{1}{\tau} \Brac{ \inangle*{\nabla_\pm f_B(z), w-z'} + \inangle*{\nabla_\pm f_B(w), z'-u} + \inangle*{\psi(w), w-u}} \\
    \leq& \breg{z}{u} - \breg{z'}{u} - ( V_{w}^r{z'} + \breg{z}{w} ) + \frac{\alpha}{\tau} \paren{ \breg{w}{u} + \breg{z'}{u}}. 
    \\
    \leq& \breg{z}{u} - \breg{z'}{u} -\frac{1}{\tau} \inangle*{\nabla_\pm f_B(w) - \nabla_\pm f_B(z), w - z'} + \frac{\alpha}{\tau} \paren{ \breg{w}{u} - \breg{z'}{u}}. 
\end{align*}
Rearranging terms, we have that for all $u \in \cB_{c, \zground}^{\cS}$, 
\begin{align*}
       \frac{1}{\tau} \inangle*{\nabla_\pm \bilinear{B}(w) +\psi(w), w-u} &= \frac{1}{\tau} \Brac{\inangle*{\nabla_\pm f_B(z), w-z'} + \inangle*{\nabla_\pm f_B(w), z'-u} + \inangle*{\psi(w), w-u}}  \\ 
       &~~+  \frac{1}{\tau} \inangle*{\nabla_\pm f_B(w) - \nabla_\pm f_B(z), w - z'} + \frac{\alpha}{\tau} \paren{ \breg{w}{u} - \breg{z'}{u}} \\
       &\leq \breg{z}{u} - \breg{z'}{u} + \frac{\alpha}{\tau} \paren{ \breg{w}{u} - \breg{z'}{u}}. 
\end{align*}
Thus, for $u = z^\star$, we have 
\begin{align*}
   \frac{1}{\tau} {\inangle*{\nabla_\pm \bilinear{B}(w)+\psi(w), w-z^\star}} \leq  \breg{z}{z^\star} - \breg{z'}{z^\star} + \frac{\alpha}{\tau} \paren{ \breg{w}{z^\star} - \breg{z'}{z^\star}}. 
\end{align*}
Consequently, subtracting $\frac{\alpha}{\tau} \breg{w}{z^\star}$ from both sides,
\begin{align}\label{eq:right-side-boundedness}
    \frac{1}{\tau} \inangle*{\nabla_\pm \bilinear{B}(w)+\psi(w), w-z^\star} - \frac{\alpha}{\tau} \breg{w}{z^\star} \leq \breg{z}{z^\star} - \paren{1 + \frac{\alpha}{\tau}} \breg{z'}{z^\star}. 
\end{align}
To complete the proof, it suffices to lower bound the left hand side of \eqref{eq:right-side-boundedness} by 0. To this end, note that, by the definition of $z^\star$, we have that $\frac{1}{\tau} \inangle*{\nabla_\pm \bilinear{B}(z^\star)+\psi(z^\star), z^\star-w} \leq 0$. Consequently,
\begin{align*}
     &\frac{1}{\tau} \inangle*{\nabla_\pm \bilinear{B}(w)+\psi(w),  w-z^\star} - \frac{\alpha}{\tau} \breg{w}{z^\star} \\
     \geq& \frac{1}{\tau} \inangle*{(\nabla_\pm \bilinear{B}(w)+\psi(w)) - (\nabla_\pm \bilinear{B}(z^\star)+\psi(z^\star)), w-z^\star} - \frac{\alpha}{\tau} \breg{w}{z^\star}. 
\end{align*}
Now, by $\alpha$-strong monotonicity of the operator $\nabla_\pm f_B + \nabla \psi$, 
\begin{align*}
        \frac{1}{\tau} \inangle*{(\nabla_\pm \bilinear{B}(w)+\psi(w)) - (\nabla_\pm \bilinear{B}(z^\star)+\psi(z^\star)), w-z^\star} \geq \frac{\alpha}{\tau} \breg{w}{z^\star} 
\end{align*}
and hence from the preceding two displays we can conclude that 
\begin{align*}
     \frac{1}{\tau} \inangle*{\nabla_\pm \bilinear{B}(w)+\psi(w),  w-z^\star} - \frac{\alpha}{\tau} \breg{w}{z^\star} \geq 0.
\end{align*}
Thus, taking \eqref{eq:right-side-boundedness} and dividing through by $\paren{1 + \frac{\alpha}{\tau}}$ we obtain $\breg{z'}{z^\star} \leq \paren{1 + \frac{\alpha}{\tau}}^{-1} \breg{z}{z^\star}.$ Finally, the matvec complexity is evident from Lemma~\ref{lemma:judge-guarantee}. 
\end{proof}

\subsection{Implementing an divergence-bounded solution oracle}\label{sec:solver}

Here we discuss Algorithm~\ref{alg:subsolver}, which is our ultimate smooth-until-proven-guilty mirror prox algorithm for implementing a divergence-bounded solution oracle (Definition~\ref{def:oracle}). 

\RestyleAlgo{ruled}\label{alg:subsolver}
\SetKwComment{Comment}{/* }{ */}
\begin{algorithm2e}[ht]
\caption{Smooth-until-proven-guilty solver $\SUG(\cU, c, \alpha, \zground, \cP, \epsilon)$}
\KwInput{A constrained prox multi-point problem $(\cU, c, \alpha, \zground)$ (Definition~\ref{def:subproblem}), a matrix approximation path $\cP$ to $\zground$, and a target accuracy $\epsilon>0$.}
\KwParameter{A smoothness threshold $\tau > 0$}
$\cP^{(0)} = \{\Delta_\ell, M_\ell^{(0)}\}_{\ell \in [L]} \gets \cP$\;
$z^{(0)} \gets \argmin_{z \in \cB_{c, \zground}^{\cS}} r(z)$\; 
$j \gets 0, k \gets 0$\;
\While{$j \leq J$ where $J = \ceil{(1 + \tau/\alpha) \log(\Gamma_\cS/\epsilon)}$ }{
    $(z^{(j+1)}, \verdict, \cP^{(j)}) \gets \Step(\cU, \cP^{(j)}, c, \alpha, \zground, z^{(j)})$\;
    \lIf{$\verdict = \guilty$}{ $k \gets k+1$ \label{line:path-update-iter} }
    \lElse{$\cP^{(j+1)} \gets \cP^{(j)}$ and then $j \gets j+1$ \label{line:progress-iter}} 
}
\Return{$(z', \cP^{(J)})$}
\end{algorithm2e}

\begin{theorem}\label{thm:sug-main} For any $\tau > 0$, $\SUG$
(Algorithm~\ref{alg:subsolver}) is a divergence-bounded solution oracle for $\cS$ (Definition~\ref{def:oracle}). Moreover, for any constrained prox multi-point problem $(\cU, c, \alpha, \zground, \cS)$ (Definition~\ref{def:subproblem}) and matrix approximation path $\cP$ to $\zground$, letting $(z', \cP') \defeq \SUG(\cU, c, \alpha, \zground, \cP, \epsilon)$, the algorithm makes at most 
\begin{align*}
    L \left\lceil{1 + \frac{\tau}{\alpha}}  \log\paren{\frac{\Gamma_\cS}{\epsilon}} \right\rceil + \frac{L^2}{(2\pi(c) \cdot \tau)^2}[\size(\cP) - \size(\cP')] \text{ matvecs to } A. 
\end{align*}
\end{theorem}

\begin{proof} First, note that by Lemma~\ref{lemma:step-guarantee}, the algorithm maintains the invariant that $\cP^{(j)}$ is always a matrix-approximation path to $\zground.$ Now, on every iteration of the while loop of Algorithm~\ref{alg:subsolver}, we have that either $k$ or $j$ is incremented. We refer to iterations wherein $k$ is iterated as \emph{path update steps} and refer to iterations where $j$ is updated as \emph{convergence progress steps}. 

First, we analyze the convergence progress steps. By Lemma~\ref{lemma:step-guarantee} we have that letting $z^\star \defeq \prox_{\cU}^\alpha(\nabla_\pm f_A; \cB_{c, \zground}^{\cS})$, for each $j \geq 0$,
\begin{align*}
    \breg{z^\star}{z^{(j+1)}} \leq \paren{1 + \frac{\alpha}{\tau}}^{-1} \breg{z^\star}{z^{(j)}}. 
\end{align*}
Consequently, by induction, 
\begin{align*}
    \breg{z^\star}{z^{(J)}} \leq  \paren{1 + \frac{\alpha}{\tau}}^{-J} \breg{z^\star}{z^{(0)}} \leq \paren{1 + \frac{\alpha}{\tau}}^{-J} \Gamma_\cS \leq \epsilon. 
\end{align*}

To bound the matvec complexity, recall from Lemma~\ref{lemma:step-guarantee} that each call to $\Step$ runs in $O(L)$ matvecs to $A$. The total number of convergence progress steps is $J$ and each convergence progress step $j \geq 0$ maintains $\size(\cP^{(j+1)}) \leq \size(\cP^{(j)})$ (by Lemma~\ref{lemma:step-guarantee}).

Meanwhile, for each path update step, Lemma~\ref{lemma:step-guarantee} guarantees that $\size(\cP') \leq \size(\cP) - (2\pi(c) \cdot \tau)^2/L$. Thus, by induction, letting $K$ denote the total number of path update iterations, we have 
\begin{align*}
    \size(\cP^{(j)}) \leq \size(\cP^{(0)}) - \frac{K (2\pi(c) \cdot \tau)^2}{L}, 
\end{align*}
Consequently, rearranging the above expression yields the result as
\begin{align*}
    K \leq \frac{L}{(2\pi(c) \cdot \tau)^2}[\size(\cP^{(0)}) - \size(\cP^{(j)})]\,. 
\end{align*}
\end{proof}
 
%

\section{Main results}\label{sec:putting-together}

In this section, we show how to apply the machinery developed in the previous sections to obtain our main results. In Section~\ref{sec:general-analysis}, we describe a general result for matrix games under  several assumptions introduced in the previous sections. Then, in
Section~\ref{sec:applications}, we verify these assumptions for the setups associated with $\ellOneEllOne$ and $\ellTwoEllOne$ matrix games and prove Theorem~\ref{thm:final-result-l1-l1-aka-zero-sum} and Theorem~\ref{thm:final-result-l2-l1-aka-SVM}. 

%

\subsection{Complexity analysis of general framework}\label{sec:general-analysis}

Here, we discuss how to combine the results from the previous sections to obtain a general algorithm for solving matrix games (recall \eqref{eq:intro-general-matrix-game}) under appropriate assumptions on the setup and bound its matvec complexity. 

\begin{assumptions}
In the remainder of Section~\ref{sec:general-analysis}, we fix arbitrary $\tau, \beta, \epsilon, \rho > 0$, $A \in \R^{m \times n}$, and dgf setups $\dgfsetup\x = (\xset \subset \R^n, \rx)$ and $\dgfsetup\y = (\yset \subset \R^m, \ry)$ with $\dgfsetup = (\zset \subset \R^d,r) \defeq \prodsetup(\dgfsetup\x, \dgfsetup\y)$ (recall Definition~\ref{def:product-dgf-setups}). We assume that $\Gamma_\cS$ is an upper bound on the range of $r$ so that $\sup_{z \in \cZ} \breg{z'}{z} \leq \Gamma_\cS$ for $z' \defeq \argmin_{z\in\cZ} r(z)$. We also assume that for any $z \in \cZ$ and $c > 1$, $\cB^\cS_{c, z}$ is closed and convex. Moreover, we assume that $\cS$ is $\pi$-locally bounded (recall Definition~\ref{def:consistency}) and $(\iota, \rho)$-stable with respect to a mapping $(\alpha > 0, \cU \subseteq \cZ) \mapsto \bestresponse(\alpha, \cU)$ (recall Definition~\ref{def:best-response-stability}) and that for any $\alpha > 0$ and finite nonempty $\cU \subseteq \cZ$, $\bestresponse(\alpha, \cU)$ can be computed with $O(1)$ matvecs to $A$. Furthermore, we assume that $\cS$ is $\zeta$-compatible with respect to $A$ (recall Definition~\ref{def:zeta-compatible-mapping}), and that for every $\epsilon, \delta > 0$, $\cS$ is $(\epsilon, \delta, \rho, \kappa(\epsilon, \delta))$-robust for some function $\kappa: \R_{>0} \times \R_{>0} \to \R_{>0}$ (recall Definition~\ref{def:robustness}). Finally, (as in Theorem~\ref{thm:dmp-basic}) we assume that $M$ and $\theta_r > \beta$ are fixed finite values such that for any finite nonempty multiset $\cU \subseteq \cZ$ and $\alpha > 0$, letting $z_\alpha^\star \defeq \prox_\cU^\alpha(\nabla_\pm f_A; \cZ)$ and $h(\alpha) \defeq \breg{\cU}{z^\star_\alpha}$, $h$ is $M$-Lipchitz over $[\beta, \theta_r]$ with $h(\theta_r) < 2.4 \theta_r^\rho$.
\end{assumptions}

As our first general complexity guarantee, we bound the matvec complexity of the $\MDMPImp$ subroutine in Algorithm~\ref{alg:DMP-implementation-matrix-games} by using the divergence-bounded solution oracle presented in Section~\ref{sec:sug-solver} to instantiate an approximate solution oracle $\OAPPROX$ (recall Definition~\ref{def:approx-solution} and~\ref{def:oracle}).

\begin{theorem}[Complexity of $\MDMPImp$ using $\SUG$ to implement $\OAPPROX$]\label{thm:mdmp-imp-complexity} Consider $\MDMPImp$ (Algorithm~\ref{alg:DMP-implementation-matrix-games}) instantiated with
\begin{align}\label{eq:oas-instantiate}
    \OAPPROX(\cU, c, \alpha, \bar{z}, \epsilon, \bar{\cP}, \epsilon, \delta) \gets \SUG(\cU, c, \alpha, \bar{z}, \epsilon, \bar{\cP}, \kappa(\epsilon, \delta))
\end{align}
for every constrained prox multi-point problem $(\cU, c, \alpha, \bar{z})$, matrix approximation path $\bar{\cP}$ to $\bar{z} \in \cU$, and $\epsilon, \delta > 0$. Then, $\OAPPROX$ is an $\rho$-approximate solution oracle (Definition~\ref{def:approx-solution} and~\ref{def:oracle}) and $\MDMPImp$ is a $(\beta, 2, \rho)$-kinetic $\epsilon$-$\MDMP$ (Definition~\ref{def:MDMP}.) 

Moreover, letting $(z_*, \alpha_*, \cP_*) = \MDMPImp(\cU, \cP)$ for any finite nonempty multiset $\cU \subseteq \cZ$ and a matrix approximation path $\cP$ to $z \in \cU$, $\MDMPImp(\cU, \cP)$ makes
\begin{align*}
    O\paren{\log\paren{\frac{\theta_r}{\min\{\epsilon', \beta\}}} \paren{L \left\lceil{1 + \frac{\tau}{\beta}}  \log\paren{\frac{\Gamma_\cS}{\kappa(\epsilon, \delta)}} \right\rceil + \frac{L^2}{(\pi(\iota(5)) \tau)^2}\{[\size(\cP) - \size(\cP')] + \zeta (2\alpha_*)^\rho\}}}
\end{align*}
matvecs to $A$, where $\epsilon'$ is as defined in \eqref{eq:epsilonprime}. 
\end{theorem}

\begin{proof} First, recall that by the definition of robustness (Definition~\ref{def:robustness}), $\OAPPROX$ is a $\rho$-approximate solution oracle (Definitions~\ref{def:approx-solution} and~\ref{def:oracle}). Thus, by Theorem~\ref{thm:dmp-basic}, we have that $\MDMPImp$ us an $\epsilon$-$\MDMP$. This completes the proof of the first two claims. 

To prove the final claim, we first bound the matvec complexity of a single call to the subroutine $\Validate(\alpha).$ To this end, consider a single call to $\Validate(\alpha)$ for arbitrary $\alpha > 0$. Letting $\cP$ and $\cP'$ denote the matrix-approximation path to $z$ before and after (respectively) the in-place update in Line~\ref{line:inplace}, we claim that $\Validate(\alpha)$ makes at most 
\begin{align}\label{eq:validate-complexity}
    O(1) + (L+1) \left\lceil{1 + \frac{\tau}{\alpha}}  \log\paren{\frac{\Gamma_\cS}{\kappa(\epsilon, \delta)}} \right\rceil + \frac{(L+1)^2}{(\pi(\iota(5)) \tau)^2}\{[\size(\cP) - \size(\cP')] + 3\zeta \alpha^\rho\}
\end{align}
matvecs to $A$. 

To prove this, we split into two cases. First, if $\Validate(\alpha)$ returns on Line~\ref{line:safety-check}, then it runs in $O(1)$ matvecs to $A$ (which is the cost of computing $\tilde{z} = \bestresponse(\alpha, \cU)$), thus the claimed bound in \eqref{eq:validate-complexity} is trivially true. Otherwise, by Theorem~\ref{thm:sug-main}, the call to $\OAPPROX$ in Line~\ref{line:wrap} runs in 
\begin{align}\label{eq:complexity-of-oas}
    (L+1) \left\lceil{1 + \frac{\tau}{\alpha}}  \log\paren{\frac{\Gamma_\cS}{\kappa(\epsilon, \delta)}} \right\rceil + \frac{(L+1)^2}{(\pi(\iota(5)) \tau)^2}\left[\size(\{\Delta_\ell, M_\ell\})_{\ell \in [L+1]} - \size(\{\Delta_\ell, M'_\ell\})_{\ell \in [L+1]}\right],
\end{align}
matvecs to $A$, where $\{\Delta_\ell, M'_\ell\}_{\ell \in [L+1]}$ and $\{\Delta_\ell, M_\ell\}_{\ell \in [L+1]}$ are as in Line~\ref{line:wrap}. Now, by the definition of $\size(\cdot)$ (Definition~\ref{def:matrix-approx-path}) and the update in Line~\ref{line:inplace}, we have that 
\begin{align*}
    &\size(\cP) - \size(\cP') = \left[\size(\{\Delta_\ell, M_\ell\})_{\ell \in [L]} - \size(\{\Delta_\ell, M'_\ell\})_{\ell \in [L]}\right] \\
    &= \left[\size(\{\Delta_\ell, M_\ell\})_{\ell \in [L+1]} - \size(\{\Delta_\ell, M'_\ell\})_{\ell \in [L+1]}\right] - \paren{\normInline{\Delta_{L+1} - M_{L+1}}_F^2 - \normInline{\Delta_{L+1} - M'_{L+1}}_F^2}. 
\end{align*}
Further, note that Line~\ref{line:null-model} ensures $M_{L+1} = 0$ and $\normInline{\Delta_{L+1}}_F^2 = \normInline{\ground{A}{\zground} - \ground{A}{z}}_F^2$. Substituting this into the display above, we have that
\begin{align*}
    &\size(\cP) - \size(\cP') \\
     =& \left[\size(\{\Delta_\ell, M_\ell\})_{\ell \in [L+1]} - \size(\{\Delta_\ell, M'_\ell\})_{\ell \in [L+1]}\right] - \paren{\normInline{\Delta_{L+1}}_F^2 - \normInline{\Delta_{L+1} - M'_{L+1}}_F^2} \\
    \geq& \left[\size(\{\Delta_\ell, M_\ell\})_{\ell \in [L+1]} - \size(\{\Delta_\ell, M'_\ell\})_{\ell \in [L+1]}\right] - \normInline{\Delta_{L+1}}_F^2 \\
    =& \left[\size(\{\Delta_\ell, M_\ell\})_{\ell \in [L+1]} - \size(\{\Delta_\ell, M'_\ell\})_{\ell \in [L+1]}\right] - \normInline{\ground{A}{\zground} - \ground{A}{z}}_F^2.
\end{align*}
Finally, recalling that $\cS$ is $\zeta$-compatible with respect to $A$, note that $\normInline{\ground{A}{\zground} - \ground{A}{z}}_F^2 \leq \zeta \breg{z}{\zground}$. Consequently, substituting this bound into the display above and rearranging, 
\begin{align*}
     \left[\size(\{\Delta_\ell, M_\ell\})_{\ell \in [L+1]} - \size(\{\Delta_\ell, M'_\ell\})_{\ell \in [L+1]}\right]  \leq [\size(\cP) - \size(\cP')] + \zeta \breg{z}{\zground}, 
\end{align*}
where by the check in Line~\ref{line:safety-check}, and the fact that $z \in \cU$, we have that $\breg{z}{\zground} \leq \breg{\cU}{\zground} \leq 3\alpha^\rho$ and consequently, 
\begin{align*}
    \left[\size(\{\Delta_\ell, M_\ell\})_{\ell \in [L+1]} - \size(\{\Delta_\ell, M'_\ell\})_{\ell \in [L+1]}\right]  \leq [\size(\cP) - \size(\cP')] + 3\zeta\alpha^\rho. 
\end{align*}
Thus, the bound in \eqref{eq:validate-complexity} holds by substituting the above bound into \eqref{eq:complexity-of-oas}. 

Now, let $(z_*, \alpha_*, \cP_*) = \MDMPImp(\cU, \cP)$ for any finite nonempty multiset $\cU \subseteq \cZ$ and a matrix approximation path $\cP$ to $z \in \cU$. By Theorem~\ref{thm:dmp-basic}, $\Validate(\alpha)$ is only ever called for $\alpha \in [\beta, \min\{2\alpha_*, \theta_r\}]$. Consequently, by \eqref{eq:validate-complexity} and Theorem~\ref{thm:dmp-basic}$, \MDMPImp(\cU, \cP)$ makes at most 
\begin{align*}
    O\paren{\log\paren{\frac{\theta_r}{\min\{\epsilon', \beta\}}} \paren{L \left\lceil{1 + \frac{\tau}{\beta}}  \log\paren{\frac{\Gamma_\cS}{\kappa(\epsilon, \delta)}} \right\rceil + \frac{L^2}{(\pi(\iota(5)) \tau)^2}\{[\size(\cP) - \size(\cP')] + \zeta (2\alpha_*)^\rho\}}}
\end{align*}
matvecs to $A$. 
\end{proof}

Combining this complexity guarantee with Theorem~\ref{thm:matrix-games-outer-loop-guarantee}, we obtain the following general result. 

\begin{theorem}[General framework complexity guarantee for matrix games]\label{thm:main-general-result} Consider Algorithm~\ref{alg:final-algo-outer-loop} instantiated with $\OAPPROX$ as in \eqref{eq:oas-instantiate} and $\OMDMP(\cU, \cP) \gets \MDMPImp(\cU, \cP)$ (Algorithm~\ref{alg:DMP-implementation-matrix-games}) and $K \gets \inceil{5 \log_2 \inparen{\Gamma_\dgfsetup (\beta \epsilon^{-1} + 2^{- \frac{1}{\rho+1}} \epsilon^{- \frac{\rho}{\rho + 1}}) + 2} } + 5$. Then Algorithm~\ref{alg:final-algo-outer-loop} makes
\begin{align*}
    {O} \Bigg(&\paren{K \Gamma_\dgfsetup (\beta \epsilon^{-1} + 2^{- \frac{1}{\rho+1}} \epsilon^{- \frac{\rho}{\rho + 1}})} \cdot \log\paren{\frac{\theta_r}{\min\{\epsilon', \beta\}}} \cdot \paren{K \left\lceil{1 + \frac{\tau}{\beta}}  \log\paren{\frac{\Gamma_\cS}{\kappa(\epsilon, \delta)}}\right\rceil} \\
    &~~+ \log\paren{\frac{\theta_r}{\min\{\epsilon', \beta\}}} \cdot \frac{K^2}{(\pi(\iota(5)) \tau)^2} \left( \zeta 2^\rho \left(\frac{K\Gamma_\cS}{2} + T\beta^\rho\right) + \left(\normInline{\ground{A}{z^{(0)}}}_F^2 + \zeta K \Gamma_\cS\right)\right)\Bigg)
\end{align*}
matvecs to $A$ and the output $\zbar$ is a $2 \epsilon$-solution of \eqref{eq:intro-general-matrix-game}. 
\end{theorem}
\begin{proof} This result follows from summing the complexity bound from Theorem~\ref{thm:mdmp-imp-complexity} and applying the bounds in \eqref{eq:bound-on-sum-alpha-and-path-diffs} from Theorem~\ref{thm:matrix-games-outer-loop-guarantee}. Indeed, by Theorem~\ref{thm:mdmp-imp-complexity}, the overall query complexity can be bounded (up to big-$O$) as 
\begin{align*}
    &\log\paren{\frac{\theta_r}{\min\{\epsilon', \beta\}}} \sum_{t \in [T]} K \left\lceil{1 + \frac{\tau}{\beta}}  \log\paren{\frac{\Gamma_\cS}{\kappa(\epsilon, \delta)}} \right\rceil + \frac{K^2}{(\pi(\iota(5)) \tau)^2}\left([\size({\cP}^{(t)}) - \size({\cP'}^{(t)})] + \zeta (2\alpha^{(t)})^\rho\right) \\
    &= \log\paren{\frac{\theta_r}{\min\{\epsilon', \beta\}}} \cdot TK \left\lceil{1 + \frac{\tau}{\beta}}  \log\paren{\frac{\Gamma_\cS}{\kappa(\epsilon, \delta)}}\right\rceil \\
    &~~+ \log\paren{\frac{\theta_r}{\min\{\epsilon', \beta\}}} \cdot \frac{K^2}{(\pi(\iota(5)) \tau)^2} \paren{ \sum_{t \in [T]} [\size(\cP^{(t)}) - \size({\cP'}^{(t)})] + \sum_{t \in [T]} 2^\rho (\alpha^{(t)})^\rho }. 
\end{align*}
The theorem now follows immediately from the bounds on $\sum_{t \in [T]} [\size(\cP^{(t)}) - \size({\cP'}^{(t)})]$ and $\sum_{t \in [T]}  (\alpha^{(t)})^\rho$ from \eqref{eq:bound-on-sum-alpha-and-path-diffs} in Theorem~\ref{thm:matrix-games-outer-loop-guarantee}. 
\end{proof}

\subsection{Applications to $\ellOneEllOne$ and $\ellTwoEllOne$ Matrix Games}\label{sec:applications}

In order to prove our main results Theorem~\ref{thm:final-result-l1-l1-aka-zero-sum} and \ref{thm:final-result-l2-l1-aka-SVM}, we first define the canonical setups that we consider in this paper. 

\begin{definition}[$\ellOneEllOne$ and $\ellTwoEllOne$ setups] \label{def:matrix-games-setups}
With $d \defeq n + m$, we refer to the tuples  $(\xset, \yset, \xtrunc, \ytrunc, \cZint, \rx : \xset \to \R, \ry : \yset \to \R, \Gamma_{\dgfsetup})$ defined in Table \ref{table:matrix-games-setups} as the \emph{$\ellOneEllOne$ and $\ellTwoEllOne$ setups} respectively. In the context of these setups, we further define $\zset \defeq \xset \times \yset$ and $r : \zset \to \R$ via $r(z) \defeq \rx(z\x) + \ry(z\y)$. Furthermore, we define what we call \emph{truncated} domains, which restrict simplex-constrained coordinates to be at least $\nu$, with $\ztrunc \defeq \xtrunc \times \ytrunc$. Finally, we make the standard (see, e.g., \cite{kornowski2024oracle, carmon2019variance, carmon2024whole, karmarkar2025solvingzerosumgames}) normalization assumptions $\inmaxnorm{A} \le 1$ in the $\ellOneEllOne$ setup and $\innorm{A}_{2 \to \infty} \le 1$ in the $\ellTwoEllOne$ setup.
\end{definition}

\begin{table*}[ht]
\centering
\begin{tabular}{ c c c }
\hline
 & \textbf{$\ell_1$-$\ell_1$} & \textbf{$\ell_2$-$\ell_1$}   \\ \hline
$\cX$     & $\Delta^n$     & $\mathbb{B}^n$       \\ 
$\cY$     & $\Delta^m$     & $\Delta^m$     \\ 
$\cX_\nu$ & $\Delta_\nu^n$     & $\mathbb{B}^n$       \\ 
$\cY_\nu$ & $\Delta_\nu^m$     & $\Delta_\nu^m$     \\
$\cZint$ & $\Delta_{>0}^n \times \Delta_{>0}^m$ & $\mathbb{B}^n \times \Delta_{>0}^n $      \\ 
$\rx(x)$     & $\frac{1}{2}\norm{x}_2^2$    & $\frac{1}{2}\norm{x}_2^2$    \\
$\ry(y)$     & $\frac{1}{2}\norm{y}_2^2$    & $\sum_{i \in [m]} [y]_i \log([y]_i)$    \\
%
%
$\Range_{\cS}$     & $\log(mn)$    & $\frac{1}{2} + \log(m)$   \\ 
$V^r_{z'}(z)$ & $\KL(z || z')$ & $\frac{1}{2} \norm{z\x - z'\x}^2_2 + \KL(z\y||z'\y)$\\
\hline
\end{tabular}
\caption{$\ellOneEllOne$ and $\ellTwoEllOne$ setups (Definition~\ref{def:matrix-games-setups}) and associated notation.}
%
\label{table:matrix-games-setups}
\end{table*}

Note that for both the $\ellOneEllOne$ and $\ellTwoEllOne$ setups, we have that ${\dgfsetup\x}_\nu \defeq (\xset_\nu, \rx)$ and ${\dgfsetup\y}_\nu \defeq (\yset, \ry)$ are dgf setups (\Cref{def:dgf-setup}) with $\dgfsetup_\nu \defeq (\zset_\nu, r) = \prodsetup({\dgfsetup\x}_\nu, {\dgfsetup\y}_\nu)$ (\Cref{def:product-dgf-setups}). Furthermore, $\Range_\dgfsetup = \max_{z, z' \in \zset} r(z) - r(z') \ge \max_{z, z' \in \ztrunc} r(z) - r(z')$.



%

In the remainder of this section, we verify the assumptions outlined in Section~\ref{sec:general-analysis} for our application to (appropriately truncated) $\ellOneEllOne$ or $\ellTwoEllOne$ setups (Definition~\ref{def:matrix-games-setups}). First, we show that it suffices to solve the problem constrained the \emph{truncated} setup $\cS_\nu$ for appropriate $\nu > 0$. Then, we show how to instantiate the best-response mapping $\bestresponse$ for these setups for use in the the $\MDMP$ implementation from Section~\ref{sec:MDMP-implementation} and prove stability (Definition~\ref{def:best-response-stability}). Next we show that the mapping $z \mapsto \ground{A}{z}$ defined in Definition~\ref{def:product-dgf-setups} is $O(1)$-compatible in these setups. We then show that these setups are also appropriately locally-bounded (in the sense of Definition~\ref{def:consistency}), allowing use to invoke the inner subproblem solver $\SUG$ from Section~\ref{sec:sug-solver}. Finally, we discuss how, for these truncated setups, we can prove a robustness condition (Definition~\ref{def:robustness}). Combining these results, we prove Theorem~\ref{thm:final-result-l1-l1-aka-zero-sum} and Theorem~\ref{thm:final-result-l2-l1-aka-SVM}. For notational convenience, and to avoid redundancy, we handle the $\ellOneEllOne$ and $\ellTwoEllOne$ setups \emph{jointly} in our analysis, with distinctions between the two setups being deferred to the proofs of intermediate lemmas.  

\paragraph{Assumptions.}
In the remainder of this section, we assume that is  $\cS = (\xset, \yset, \xtrunc, \ytrunc, \cZint, \rx : \xset \to \R, \ry : \yset \to \R, \norm{\cdot} : \R^d \to \R, \Gamma_{\dgfsetup})$ is fixed to be any of the setups defined in Definition~\ref{def:matrix-games-setups} for a fixed but arbitrary $\nu \in (0, 1/d)$. In particular, we may also use the notation ${\dgfsetup\x}_\nu$, ${\dgfsetup\y}_\nu$, and ${\dgfsetup}_\nu$ defined above. Furthermore, we fix $A \in \R^{m\times n}$ to be any matrix satisfying the normalization assumptions in Definition~\ref{def:matrix-games-setups}. 

\paragraph{Truncation.}
First, we restate the following reduction from \citep{karmarkar2025solvingzerosumgames} shows that in order to compute an approximate solution to \eqref{eq:intro-general-matrix-game} 
%
%
%
it suffices to compute an approximate solution to the same problem over the truncated domain $\cZ_\nu = \cX_\nu \times \cY_\nu$ for an appropriate $\nu$ which scales inverse polynomially in the problem parameters ($m, n, \epsilon$). 

\begin{lemma}[Lemma 6.2 of \citep{karmarkar2025solvingzerosumgames}, restated] \label{lem:truncation-for-ell2ell1-ell1ell1}
For $\epsilon > 0$ and $0 < \nu \le \frac{\min \inbraces{\epsilon, 1}}{8 \max \inbraces{m, n}}$, any $\epsilon / 2$-solution $z' \in \ztrunc$ of 
\begin{align}
    \min_{x \in \xset_\nu} \max_{y \in \yset_\nu} \inangle*{y, Ax}. \label{eq:truncated-minimax-prob}
\end{align}
is an $\epsilon$-solution of \eqref{eq:intro-general-matrix-game}.
\end{lemma}
\paragraph{Stability.} Here, we show that the dgf stables considered in the section are $\iota$-stable with respect to the following mapping $(\alpha, \cU) \mapsto \bestresponse(\alpha, \cU)$. The following definition builds upon Definition 6.3 of \citep{karmarkar2025solvingzerosumgames}. 

\begin{definition}[$(\alpha, \cU)$-best-response]\label{def:best-response} Let $\alpha > 0$, $\cU \subseteq \cZ_\nu$ be a finite and nonempty multiset, and $\mean{\cU} \defeq \frac{1}{\abs{\cU}} \sum_{u \in \cU} u$. We define $\bestresponse(\alpha, \cU) = \prox_{\cU}^\alpha(\nabla_\pm f_A(\mean{\cU}; \cZ_\nu)$ (recall Definition~\ref{def:proximal-mappings}). That is, letting $\tilde{z} = (\tilde{x}, \tilde{y}) = \bestresponse(\alpha, \cU)$,
\begin{align*}
    \Tilde{x} = \argmin_{x \in \cX_\nu} \inangle*{{\mean{\cU}}\y,  Ax} + \alpha \xbreg{\cU\x}{x} \text{~~~and~~~}  \Tilde{y} = \argmax_{y \in \cY_\nu} \inangle*{y, A~{\mean{\cU}\x}} - \alpha \ybreg{\cU\y}{y}. 
\end{align*}
In particular, note that for any $\alpha > 0$ and finite nonempty multiset $\cU \subseteq \cZ$, $\bestresponse(\alpha, \cU)$ can be computed with $O(1)$ matvecs to $A$. 
\end{definition}

In the case of zero-sum games (the $\ellOneEllOne$ setup) the $\bestresponse(\alpha, \cU)$ can be interpreted as follows. Each player calculates each player's \emph{best response} (over $\cX_\nu$ and $\cY_\nu$) to their opponent, holding the opponents' strategy to be \emph{fixed} to be the \emph{average} of the strategies in $\cU$, subject to an $\alpha$-regularization penalty for each $u \in \cU$. We now prove the following analog of Lemma 6.4 of \citep{karmarkar2025solvingzerosumgames}.

\begin{lemma}\label{lemma:new-stability} $\cS_\nu$ is $(\iota, 2)$-stable (Definition~\ref{def:best-response-stability}) with respect to the mapping $\bestresponse(\alpha, \cU)$ defined in Definition~\ref{def:best-response} for $\iota: c \mapsto \exp(2 \sqrt{2c})$. 
\end{lemma}

Recall from Definition~\ref{def:best-response-stability} that to prove this result, we need to show that for any $\alpha, c > 0$ and any finite nonempty set $\cU \subseteq \cZ_\nu$, letting $z^\star = \prox_{\cU}^\alpha(\nabla_\pm f_A; \cZ_\nu)$ and $\Tilde{z} = \bestresponse(\alpha, \cU)$, whenever $\breg{\cU}{z^\star} \leq c\alpha^2$, we must have $z^\star \in \cB^\cZ_{\exp(2\sqrt{2c}), \Tilde{z}}$. To prove this, we reduce to Lemma 6.6 of \citep{karmarkar2025solvingzerosumgames}. 

\begin{lemma}[Lemma 6.6 of \citep{karmarkar2025solvingzerosumgames}]
    \label{lem:general-stability-helper}
    For $\alpha > 0$, vectors $\theta, \xi \in \R^d$, and $q \in \simplex^d_\nu$, define
    \begin{align*}
        %
        u_\theta \defeq \argmin_{z \in \simplex^d_\nu} \inangle*{\theta, z} + \alpha \cdot \inKL{q}{z} ~~~\text{and}~~~ u_\xi \defeq \argmin_{z \in \simplex^d_\nu} \inangle*{\xi, z} + \alpha \cdot \inKL{q}{z}.
    \end{align*}
    Then $u_\theta \approx_{\delta} u_\xi$ with $\delta \defeq \exp \inparen*{\frac{2 \norm{\theta - \xi}_\infty}{\alpha}}$.
    %
    %
    %
\end{lemma}
  
\begin{proof}[Proof of Lemma~\ref{lemma:new-stability}] 
Let $z^\star = \prox_{\cU}^\alpha(\nabla_\pm f_A; \cZ_\nu)$ and $\Tilde{z} = \bestresponse(\alpha, \cU)$. Suppose that $\breg{\cU}{z^\star} \leq c \alpha^2$. By the optimality conditions, we have that 
\begin{align*}
    z^\star\x = \argmin_{x \in \cX_\nu} \inangle*{{z\y^\star}, A x} + \alpha V^{\rx}_{\cU\x}(x), ~~~\text{ and }~~~  z^\star\y = \argmax_{y \in \cY_\nu} \inangle*{y, A z\x^\star} - \alpha V^{\ry}_{\cU\y}(y). 
\end{align*}
Corollary~\ref{corr:reduce-sum-to-one-point} ensures the existence of a $\collapsed{\cU} \in \cZ_\nu$ such that 
\begin{align*}
    z^\star\x = \argmin_{x \in \cX_\nu} \inangle*{z\y^\star, A x} + \alpha \cdot |\cU| \cdot V^{\rx}_{{\collapsed{\cU}}\x}(x), &~\text{ and }~ z^\star\y = \argmax_{y \in \cY_\nu} \inangle*{y, A z\x^\star} - \alpha \cdot |\cU| \cdot V^{\ry}_{{\collapsed{\cU}}\y}(y), \\
    \tilde{z}\x = \argmin_{x \in \cX_\nu} \inangle*{{\mean{\cU}}\y, A x} + \alpha \cdot |\cU| \cdot V^{\rx}_{{\collapsed{\cU}}\x}(x), &~\text{ and }~ \Tilde{z}\y = \argmin_{y \in \cY_\nu} \inangle*{y, A{\mean{\cU}\x}} - \alpha \cdot |\cU| \cdot V^{\ry}_{{\collapsed{\cU}}\y}(y). 
\end{align*}
In the $\ell_1$-$\ell_1$ setup (Definition~\ref{def:matrix-games-setups}), by Lemma~\ref{lem:general-stability-helper}, we have that 
\begin{align*}
    {z}\x^\star &\approx_{\delta\x} \Tilde{z}\x \text{ for } \delta\x = \exp\paren{\frac{2\normInline{A^\top\mean{\cU}\y - A^\top z\y^\star}_\infty}{\alpha}}, \\
    {z}\y^\star &\approx_{\delta\y} \Tilde{z}\y \text{ for } \delta\y = \exp\paren{\frac{2\normInline{A\mean{\cU}\x - Az\x^\star}_\infty}{\alpha}}. 
\end{align*}
Now, 
\begin{align*}
    \normInline{A^\top\mean{\cU}\y - A^\top z\y^\star} &\leq \normInline{A}_{\max} \normInline{{\mean{\cU}}\y - z\y^\star}_1 \leq \sqrt{\frac{1}{\abs{\cU}} \sum_{u \in \cU} \normInline{ \mean{\cU}\y- z^\star\y}_1^2} \\
    &\leq \sqrt{\frac{2}{\abs{\cU}} V^{\ry}_{\cU\y}(z^\star\y)} \leq \alpha \sqrt{2c}
\end{align*}
where the second inequality used Jensen's inequality and the convexity of $\normInline{\cdot}^2$, as well as the property that $\normInline{A}_{\max} \leq 1$. Hence, $\delta\x \leq \exp(2\sqrt{2c})$. An identical argument shows $\delta\y \leq \exp(2\sqrt{2c})$. 

Now, consider the $\ell_2$-$\ell_1$ setup (Definition~\ref{def:matrix-games-setups}). Again, by Lemma~\ref{lem:general-stability-helper}, we have that 
\begin{align*}
    {z}\y^\star &\approx_{\delta} \Tilde{z}\y \text{ for } \delta = \exp\paren{\frac{2\normInline{A^\top\mean{\cU}\x - A z\x^\star}_\infty}{\alpha}}. 
\end{align*}
Similar to before,
\begin{align*}
    \normInline{A\mean{\cU}\x - A z\x^\star} &\leq \max_i \norm{
    A_{:, i}
    }_2 \normInline{{\mean{\cU}}\y - z\y^\star}_2 \leq \sqrt{\frac{1}{\abs{\cU}} \sum_{u \in \cU} \normInline{ u\y- z^\star\y}_2^2} \\
    &\leq \sqrt{\frac{2}{\abs{\cU}}  V^{\ry}_{\cU\y}(z^\star\y)} \leq \alpha \sqrt{2c}
\end{align*}
where the second inequality used Jensen's inequality and the convexity of norms, and the property that $\max_i \normInline{A_{:, i}}_2 \leq 1$. Hence, $\delta \leq \exp(2\sqrt{2c})$. 
\end{proof}

\paragraph{Compatibility.} Here, we show that the setup $(\cZ_\nu, \normInline{\cdot}, r)$ is $2$-compatible with respect to $A$ (Definition~\ref{def:zeta-compatible-mapping}). To aid in the proof, for any $z_1, z_2 \in \R^d$ we let $H^2(z_1, z_2) \defeq \sum_{i \in [d]} \paren{\sqrt{[z_1]_i} - \sqrt{[z_2]_i}}^2$ denote the squared \emph{Hellinger distance} between $z_1, z_2 \in \cZ$. 

\begin{lemma}[Compatibility]\label{lemma:compatibility} 
For any $A \in \R^{m \times n}$ satisfying the normalization assumptions of the setup (Definition~\ref{def:matrix-games-setups}), the setup $\cS_\nu$ is $2$-compatible (Definition~\ref{def:zeta-compatible-mapping}) with respect to $A$. 
\end{lemma}

\begin{proof} In the $\ell_1$-$\ell_1$ setup, 
\begin{align*}
    \normInline{\ground{A}{z} - \ground{A}{z'}}_F^2 &= \normInline{\diag(z\y)^{1/2} \cdot A \cdot \diag(z\x)^{1/2} - \diag(z'\y)^{1/2} \cdot A \cdot \diag(z'\x)^{1/2}}_F^2 \\
    &= \sum_{i \in [m]} \sum_{j \in [n]} \paren{\sqrt{[z\y]_i} A_{ij} \sqrt{[z\x]_j} - \sqrt{[z'\y]_i} A_{ij} \sqrt{[z'\x]_j}}^2 \\
    &\leq \normInline{A}_{\max}^2 \sum_{i \in [m]} \sum_{j \in [n]} \paren{\sqrt{[z\y]_i} \sqrt{[z\x]_j} - \sqrt{[z'\y]_i} \sqrt{[z'\x]_j}}^2
 \end{align*}
Now, using the property that for any real numbers $a, b, c, d$ we have
\begin{align*}
    (ab-cd)^2 = (a(b-d) + d(a-c))^2 \leq 2a^2(b-d)^2 + 2d^2(a-c)^2, 
\end{align*}
we have that (taking $a = \sqrt{[z\y]_i}, b = \sqrt{[z\x]_j}, c = \sqrt{[z'\y]_i}$ and $d = \sqrt{[z'\x]_j}$ above),  
\begin{align*}
    \normInline{\ground{A}{z} - \ground{A}{z'}}_F^2 
    &\leq 2\sum_{i \in [m]} \sum_{j \in [n]} [z\y]_i \paren{\sqrt{[z\x]_i} - \sqrt{[z'\x]_i}}^2 + 2\sum_{i \in [m]} \sum_{j \in [n]} [z'\x]_i \paren{\sqrt{[z\y]_i} - \sqrt{[z'\y]_i}}^2 \\
    &\leq 2H^2(z\x, z'\x) + 2H^2(z\y, z'\y) \\
    &\leq 2\KL(z || z'). 
 \end{align*}
where the second-to-last step uses that $z\y, z\x$ are in the probability simplex, and the last step is true by Fact~\ref{fact:hellinger-trick}. In the $\ell_2$-$\ell_1$ setup,
\begin{align*}
    \normInline{\ground{A}{z} - \ground{A}{z'}}_F^2 &= \normInline{(\diag({z\y})^{1/2} - \diag({z'\y})^{1/2}) A}_F^2 \\
    &= \sum_{i \in [m]} (\sqrt{[z\y]_i} - \sqrt{[z'\y]_i})^2 \sum_{j \in [n]} A_{ij}^2 \\
    &\leq \max_{i \in [n]} \normInline{A_{i, :}}_2^2 \cdot H^2(z\y, z'\y) \leq \KL(z\y || z'\y) \leq \breg{z'}{z}, 
 \end{align*}
 where the last step is true by Fact~\ref{fact:hellinger-trick}. 
\end{proof}


%
%
%
%
%
%
%
%
%
%
%
%
%

\paragraph{Local-boundedness.} Here we verify the local boundedness condition introduced in Definition~\ref{def:best-response}.

\begin{lemma}\label{lemma:local-boundedness} There exists an explicit function $\pi: \R_{>1} \mapsto \R_{>0}$ such that $\cS_\nu$ is $\pi$-locally bounded (Definition~\ref{def:best-response}). Moreover, $\pi$ is a \emph{universal} function independent of any problem parameters. 
\end{lemma}
\begin{proof} Let $z_1, z_2 \in \cB_{c, \tilde{z}}$ for some $\tilde{z} \in \cZ$ and $c > 1$. In the $\ell_2$-$\ell_1$ setup, 
\begin{align*}
    \breg{z_1}{z_2} &= \frac{1}{2}\normInline{{z_1}\x - {z_2}\x}_2^2 + \KL(z_1|| z_2)
\end{align*}
and
\begin{align*}
    \normInline{z_1 - z_2}_{z'}^2 &= \frac{1}{2}\normInline{{z_1}\x - {z_2}\x}_2^2 + \inangle*{(z_1 -z_2)\y, \diag(z'\y)^{-1} (z_1-z_2)\y}, 
\end{align*}
in which case the result is an immediate consequence of Lemma 5.3 of \citep{karmarkar2025solvingzerosumgames}. The argument for $\ell_1$-$\ell_2$ and $\ell_1$-$\ell_1$ setups is analogous. 
\end{proof}
\citep{karmarkar2025solvingzerosumgames} explicitly characterize $\pi$ and show that it is a \emph{universal} function independent of any problem parameters; however, the specific formula is not important for our purposes, and hence, we omit it for brevity.

%

\paragraph{Robustness.} Here, we show that for any $\epsilon, \delta > 0$, the setup $\cS_\nu$ is $(\epsilon, \delta, 2, \kappa(\epsilon,\delta))$-robust (Definition~\ref{def:robustness}) for an appropriately defined $\kappa: \R_{>0} \times \R_{>0} \to \R_{>0}$. 

\begin{lemma}\label{lem:reducetoinner} There exists an absolute constant $C > 0$ such that for any $\epsilon, \delta > 0$ and
\begin{align*}
    %
    \kappa(\epsilon, \delta) \defeq C \min\left\{ \delta^2 \nu^2, \frac{\alpha^{4}}{|\uset|^2  ( 1 + \log^2(\nu^{-1}))}, \frac{\epsilon^2 }{1 + (\alpha\abs{\cU})^2 \nu^{-2}}\right\} \, , 
\end{align*}
the setup $\mathcal{S}_\nu$ is $(\epsilon, \delta, 2, \kappa(\epsilon, \delta))$-robust (Definition~\ref{def:robustness}).
\end{lemma}
\begin{proof}
    Per Lemma~\ref{lem:sufficient-cond-for-robustness}, it suffices to show that for any $(\uset, c, \alpha, z, \dgfsetup_\nu)$-constrained prox multi-point problem where $\uset, c, \alpha, z$ are arbitrary (up to the restrictions of Definition~\ref{def:subproblem}) and with solution $\zopt \defeq \prox^{\alpha}_{\uset}(\gm f_A ; \sball^{\dgfsetup_\nu}_{c, z})$, any $z' \in \sball^{\dgfsetup_\nu}_{c, z}$ with $\breg{\zopt}{z'} \le \kappa(\epsilon, \delta)$ satisfies
    \begin{itemize}
    \item $\abs{\breg{\cU}{z'} - \breg{\cU}{z^\star}} < \alpha^2 / 10$, 
    \item $z' \in \cB^{\cS_\nu}_{1+\delta, z^\star}$, and
    \item if $\prox_\cU^\alpha(\nabla_\pm f_A; \cZ_\nu) \in \cB_{c, z}^{\cS_\nu}$ then $\inangle{\nabla_{\pm}f_A(z') + \alpha \grad \breg{\cU}{z'}, z' -u} \leq \epsilon, \text{ for all } u \in \cZ_\nu$. 
    \end{itemize}
    The first property follows from Corollary~\ref{corr:reduce-sum-to-one-point} and \cite[Lemma B.5]{karmarkar2025solvingzerosumgames}, and the second property follows from \cite[Lemma B.4]{karmarkar2025solvingzerosumgames}. To prove the third property, we will follow the proof of \cite[Lemma 6.15]{karmarkar2025solvingzerosumgames} with minor modifications. Note that by Corollary~\ref{corr:reduce-sum-to-one-point}, there exists some $\qU \in \ztrunc$ such that $\grad \breg{\uset}{w} = |\uset| \grad \breg{\qU}{w}$ for all $w \in \ztrunc$. Then combining this with \eqref{eq:Bregman-three-point-equality}, we have for all $u \in \ztrunc$:
    \begin{align*}
       \inangle*{\nabla_{\pm}f_A(z') + \alpha \grad \breg{\cU}{z'}, z' -u} &= \inangle{\nabla_{\pm}f_A(z') + \alpha |\uset| \grad \breg{\qU}{z'}, z' -u} \\
       &= \inangle{\nabla_{\pm}f_A(z'), z' - u} - \alpha |\uset| \cdot \insquare{\breg{\qU}{u} - \breg{z'}{u} - \breg{\qU}{z'}} \, .
    \end{align*}
    Using analogous manipulations, we have for all $u \in \ztrunc$:
    \begin{align*}
        \inangle{\gm f_A(\zopt), \zopt - u} \le \alpha |\uset| \cdot \insquare{\breg{\qU}{u} - \breg{\zopt}{u} - \breg{\qU}{\zopt}} \, .
    \end{align*}
    The remainder of the proof follows the same steps as the proof of \cite[Lemma 6.15]{karmarkar2025solvingzerosumgames}, except we set $z_{\mathsf{c}}$, $\alpha$, $\gammav$, and $w$ in the proof of \cite[Lemma 6.15]{karmarkar2025solvingzerosumgames} to $\qU$, $\alpha |\uset|$, $\epsilon$, and $z'$ respectively. 
%
%
%
%
%
%
%
%
%
\end{proof}

\paragraph{Lipschitzness bound and binary search range.} Below, we provide a lemma to motivate our instantiation of the range upper bound $\theta_r$ in the implementation of $\MDMPImp$ (Algorithm~\ref{alg:DMP-implementation-matrix-games}). The proof of the following lemma is very similar to that of Lemma B.12 of \citep{karmarkar2025solvingzerosumgames}.

\begin{lemma}
    \label{lem:starting-value-b-search}
For any finite nonempty multiset $\cU \subseteq \zset_\nu$ and $\alpha \ge 1$, $w \defeq \prox_\cU^{\alpha}(\gm f_A; \zset_\nu)$ satisfies $\breg{\cU}{w} \le 12\abs{\cU} \log\paren{\frac{1}{\nu d}}$.
\end{lemma}

\begin{proof} First, note that by the normalization assumptions on the matrix $A$ in Definition~\ref{def:matrix-games-setups}, $|f_A(z)| \le 1$ for all $z \in \zset_\nu$. For the sake of contradiction, suppose that $\breg{\cU}{w} > 12|\cU|\log\paren{\frac{1}{\nu d}}$. Then, we must have that either $\breg{\cU\x}{w\x} > 6|\cU|\log\paren{\frac{1}{\nu d}}$ or $\breg{\cU\y}{w\y} > 6|\cU|\log\paren{\frac{1}{\nu d}}$. If $\breg{\cU\x}{w\x} > 6|\cU|\log\paren{\frac{1}{\nu d}}$, then we have 
\begin{align*}
    w\x = 
    \argmin_{x \in \xset_\nu} f_A(x, w\y) + \alpha \xbreg{\cU\x}{x}.
\end{align*}
Lemmas~\ref{lemma:uncondition-simplex-bound} and \ref{lemma:unconditional-euclidean-bound} guarantee that there exists an $x' \in \cX_\nu$ such that 
\begin{align*}
    f_A(x', w\y) + \alpha \xbreg{\cU\x}{x'} &= f_A(x', w\y) \le 1 + 4\alpha \abs{\cU}\log\paren{\frac{1}{\nu d}} \\
    &\leq 5\alpha \abs{\cU}\log\paren{\frac{1}{\nu d}} < f_A(w\x, w\y) + \alpha \xbreg{\cU\x}{w\x}, 
\end{align*}
which is a contradiction. A similar argument holds if $\breg{\cU\y}{w\y} > 6\alpha|\cU|\log\paren{\frac{1}{\nu d}}$.
\end{proof}

\begin{corollary}
    \label{corr:starting-value-b-search}
For any finite nonempty multiset $\cU \subseteq \zset_\nu$ and $\alpha \geq \frac{1}{2} \sqrt{12 \abs{\cU} \log\paren{\frac{1}{\nu d}}}$, we have that $w \defeq \prox_\cU^{\alpha}(\gm f_A; \zset_\nu)$ satisfies $\breg{\cU}{w} < 2\alpha^2$.
\end{corollary}
\begin{proof} By Lemma~\ref{lem:starting-value-b-search}, for any $\alpha \geq 1$, we have that 
\begin{align*}
    \breg{\cU}{w} \leq 12 \abs{\cU} \log\paren{\frac{1}{\nu d}}. 
\end{align*}
Consequently, for $\alpha \geq \frac{1}{2} \sqrt{12 \abs{\cU} \log\paren{\frac{1}{\nu d}}}$, the bound holds. 
\end{proof}


We prove the following guarantee regarding the Lipschitzness of the function $h$ defined in Theorem~\ref{thm:dmp-basic}. 

\begin{restatable}{lemma}{lip}\label{lem:Lipschitzness-of-h} For any $z \in \cZ_\nu$, finite nonempty multiset $\cU \subseteq \cZ_\nu$, and $\alpha > 0$, define $h: \R_{>0} \to \R$ by $h(\alpha) \defeq \breg{\cU}{\prox_{\cU}^\alpha(\nabla_\pm f_A; \cZ_\nu}$. Then, for any $0 < \theta_\ell < \theta_r$ and $\alpha, \alpha' \in [\theta_\ell, \theta_r]$ there exists an absolute constant $M > 0$ such that 
\begin{align*}
    \abs{h(\alpha) - h(\alpha')} \leq M \abs{\cU}^3 \paren{\frac{1 + \log(\nu^{-1})}{\nu \theta_\ell} + \abs{\cU}\theta_r} \cdot \abs{\alpha - \alpha'}. 
\end{align*}
\end{restatable}

Our proof leverages the following lemma of \citep{karmarkar2025solvingzerosumgames}.

\begin{lemma}[Lemma B.11 of \citep{karmarkar2025solvingzerosumgames}, restated]
    \label{lem:Lipschitzness-of-h-og}
For a fixed $q \in \ztrunc$ and parameter $\alpha > 0$, let $w_\alpha \defeq \prox_q^{\alpha}(\gm f_A; \ztrunc)$ (namely, $w_\alpha$ is parameterized by $\alpha$) and define $\vartheta : \R_{>0} \to \R$ via $\vartheta(\alpha) \defeq \breg{q}{w_\alpha} - 2 \alpha^2$. Then for any $\alpha, \alpha' \in [b, c]$ for some $c > b > 0$, there exists an absolute constant $M' > 0$ such that
\begin{align*}
    |\vartheta(\alpha) - \vartheta(\alpha')| \le M' \inparen*{\frac{1 + \log \nu^{-1}}{\nu b} + c} \cdot |\alpha - \alpha'|.
\end{align*}
\end{lemma}

\begin{proof}[Proof of Lemma~\ref{lem:Lipschitzness-of-h}] Let $w_\alpha \defeq \prox_{\cU}^\alpha(\nabla_\pm f_A; \cZ_\nu)$ and $w_{\alpha'} \defeq \prox_{\cU}^{\alpha'}(\nabla_\pm f_A; \cZ_\nu)$. By Corollary~\ref{corr:reduce-sum-to-one-point}, there is a point $\collapsed{\cU}$ such that $w_\alpha =\prox_{\collapsed{\cU}}^{\alpha\abs{\cU}}(\gm f_A; \ztrunc)$ 
and $w_{\alpha'} = \prox_{\collapsed{\cU}}^{\alpha\abs{\cU}}(\gm f_A; \ztrunc)$. Consequently, by Corollary~\ref{corr:reduce-sum-to-one-point}, we have that 
\begin{align*}
    \abs{h(\alpha) - h(\alpha')} &= \abs{\abs{\cU} \cdot \breg{\collapsed{\cU}}{w_\alpha} - \abs{\cU}\breg{\collapsed{\cU}}{w_{\alpha'}}} \\
    &= \abs{\abs{\cU} \cdot \breg{\collapsed{\cU}}{w_\alpha} - \alpha^2 - \abs{\cU}\breg{\collapsed{\cU}}{w_{\alpha'}} + {\alpha'}^2 + (\alpha^2 - {\alpha'}^2)} \\
    &= \abs{\cU}\abs{\vartheta(\abs{\cU}\alpha) - \vartheta(\abs{\cU}\alpha')} + 4R|\alpha - \alpha'| \\
    &\leq M' \abs{\cU}^2 \paren{\frac{1 + \log(\nu^{-1})}{\nu \abs{\cU}L} + \abs{\cU} R} \cdot \abs{\abs{\cU} (\alpha - \alpha')} + 4R (\alpha - \alpha') \\
    &\leq M' \abs{\cU}^4 \paren{\frac{1 + \log(\nu^{-1})}{\nu \abs{\cU}b} + 5R} \abs{\alpha - \alpha'}, 
\end{align*}
where in the third line we used the fact that the function $x \mapsto x^2$ is $4R$-Lipschitz on $[L, R]$ and in the fourth line we used Lemma~\ref{lem:Lipschitzness-of-h-og}. 
\end{proof}

\paragraph{Proof of main results.} Finally, we conclude by proving our main results. In the proof of the main results, we fix the following: 
\begin{itemize}
    \item Setup: $\cS_\nu$ as defined in Definition~\ref{def:matrix-games-setups}; 
    \item $\nu := \min\left\{\frac{\min\{\epsilon, 1\}}{8 \max\{m, n\}}, \frac{1}{d}\right\}$;
    \item Compatibility (Definition~\ref{def:zeta-compatible-mapping}): $\zeta = 2$; 
    \item Stability mapping (Definition~\ref{def:best-response-stability}): $\iota: c \mapsto \exp(2\sqrt{2c})$ as in Lemma~\ref{lemma:new-stability};
    \item Local boundedness mapping (Definition~\ref{def:consistency}): $\pi$ as guaranteed by Lemma~\ref{lemma:local-boundedness}; 
    \item $\MDMP$ parameters: $\beta = \epsilon^{1/3}, \rho = 2, \gamma = 2$; 
    \item Approximation parameters: $\delta = \frac{1}{2} \paren{\exp(2\sqrt{10} - 4\sqrt{2}) - 1} \approx .47$ (it is easy to verify that this satisfies the constraints of Line~\ref{line:delta2}) and $\kappa(\epsilon, \delta)$ as defined in Lemma~\ref{lem:reducetoinner};
    \item $\SUG$ parameter: $\tau = \beta$; 
    \item Bisection-search parameters: $\theta_\ell = \beta, \theta_r = 1/2\cdot\sqrt{12 \abs{\cU} \log(1/(\nu d)}$ as in Corollary~\ref{corr:starting-value-b-search}, and $M$ as in Lemma~\ref{lem:Lipschitzness-of-h}. 
\end{itemize}

We also use the following bound from \citep{karmarkar2025solvingzerosumgames}. 
\begin{lemma}[Remark 6.12 of \citep{karmarkar2025solvingzerosumgames}, restated]\label{lemma:uniform-bound-start} For any $A \in \R^{m \times n}$ satisfying the assumptions of the setup (Definition~\ref{def:matrix-games-setups}) and any $z \in \cZ_\nu$, we have that $\normInline{\ground{A}{z}}_F^2 \leq 1$. 
\end{lemma}

\begin{proof}[Proof of Theorem~\ref{thm:final-result-l1-l1-aka-zero-sum} and Theorem~\ref{thm:final-result-l2-l1-aka-SVM}] The proof follows immediately from Theorem~\ref{thm:main-general-result} and Lemma~\ref{lem:truncation-for-ell2ell1-ell1ell1}. Indeed, note that the parameters $K, \Gamma_\cS = \tilde{O}(1)$, the parameters $\pi(\iota(5)), \rho, \cT_{\max}, \zeta$ are absolute constants, and $\kappa(\epsilon
, \delta), \epsilon', \beta$ are all inverse polynomial in the problem parameters. Consequently, Theorem~\ref{thm:main-general-result} guarantees a matvec complexity of 
\begin{align*}
    \tilde{O}\paren{ \beta\epsilon^{-1} \paren{1 + \frac{\tau}{\beta}} + \frac{1 + \beta^\rho}{\tau^2}}. 
\end{align*}
Substituting $\tau = \beta = \epsilon^{1/3}$ and $\rho = 2$ now yields the final complexity of $\tilde{O}(\epsilon^{-2/3})$ matvecs to $A$, as desired. 
\end{proof}

%
%
%
%
%
%
%
%
%
%

%
%
 
\section*{Acknowledgements}

%
%
%
%
Ishani Karmarkar was funded in part by NSF Grant CCF-1955039, and a PayPal research award.
Liam O'Carroll was funded in part by NSF Grant CCF-1955039.
Aaron Sidford was funded in part by a Microsoft Research Faculty Fellowship, NSF Grant CCF1955039, and a PayPal research award.

\newpage

\bibliographystyle{plainnat}
%
%
%
\begin{thebibliography}{37}
\providecommand{\natexlab}[1]{#1}
\providecommand{\url}[1]{\texttt{#1}}
\expandafter\ifx\csname urlstyle\endcsname\relax
  \providecommand{\doi}[1]{doi: #1}\else
  \providecommand{\doi}{doi: \begingroup \urlstyle{rm}\Url}\fi

\bibitem[Adler(2013)]{Adler2013}
Ilan Adler.
\newblock The equivalence of linear programs and zero-sum games.
\newblock \emph{International Journal of Game Theory}, 42\penalty0
  (1):\penalty0 165--177, feb 2013.

\bibitem[Axelrod et~al.(2019)Axelrod, Liu, and
  Sidford]{Axelrod2019NearoptimalAD}
Brian Axelrod, Yang~P. Liu, and Aaron Sidford.
\newblock Near-optimal approximate discrete and continuous submodular function
  minimization.
\newblock In \emph{ACM-SIAM Symposium on Discrete Algorithms}, 2019.

\bibitem[Bachoc et~al.(2022)Bachoc, Cesari, Colomboni, and
  Paudice]{bachoc2022nearoptimalalgorithmunivariatezerothorder}
François Bachoc, Tommaso Cesari, Roberto Colomboni, and Andrea Paudice.
\newblock A near-optimal algorithm for univariate zeroth-order budget convex
  optimization, 2022.

\bibitem[Beck and Teboulle(2003)]{beck2003mirrordescent}
Amir Beck and Marc Teboulle.
\newblock Mirror descent and nonlinear projected subgradient methods for convex
  optimization.
\newblock \emph{Operations Research Letters}, 31\penalty0 (3):\penalty0
  167--175, 2003.
\newblock ISSN 0167-6377.

\bibitem[Carmon et~al.(2019)Carmon, Jin, Sidford, and Tian]{carmon2019variance}
Yair Carmon, Yujia Jin, Aaron Sidford, and Kevin Tian.
\newblock Variance reduction for matrix games.
\newblock In \emph{\cNIPS{2019}}, 2019.

\bibitem[Carmon et~al.(2020{\natexlab{a}})Carmon, Jambulapati, Jiang, Jin, Lee,
  Sidford, and Tian]{carmon2020acceleration}
Yair Carmon, Arun Jambulapati, Qijia Jiang, Yujia Jin, Yin~Tat Lee, Aaron
  Sidford, and Kevin Tian.
\newblock Acceleration with a ball optimization oracle.
\newblock In \emph{\cNIPS{2020}}, 2020{\natexlab{a}}.

\bibitem[Carmon et~al.(2020{\natexlab{b}})Carmon, Jin, Sidford, and
  Tian]{carmon2020coordinate}
Yair Carmon, Yujia Jin, Aaron Sidford, and Kevin Tian.
\newblock Coordinate methods for matrix games.
\newblock In \emph{\cFOCS{2020}}, 2020{\natexlab{b}}.

\bibitem[Carmon et~al.(2021)Carmon, Jambulapati, Jin, and
  Sidford]{carmon2021thinking}
Yair Carmon, Arun Jambulapati, Yujia Jin, and Aaron Sidford.
\newblock Thinking inside the ball: Near-optimal minimization of the maximal
  loss.
\newblock In \emph{\cCOLT{2021}}, 2021.

\bibitem[Carmon et~al.(2024)Carmon, Jambulapati, Jin, and
  Sidford]{carmon2024whole}
Yair Carmon, Arun Jambulapati, Yujia Jin, and Aaron Sidford.
\newblock A whole new ball game: A primal accelerated method for matrix games
  and minimizing the maximum of smooth functions.
\newblock In \emph{\cSODA{2024}}, 2024.

\bibitem[Carmon et~al.(2025)Carmon, Jambulapati, O'Carroll, and
  Sidford]{carmon2025extractingdualsolutionsprimal}
Yair Carmon, Arun Jambulapati, Liam O'Carroll, and Aaron Sidford.
\newblock Extracting dual solutions via primal optimizers.
\newblock In \emph{16th Innovations in Theoretical Computer Science Conference
  (ITCS 2025)}, Leibniz International Proceedings in Informatics (LIPIcs).
  Schloss Dagstuhl -- Leibniz-Zentrum f{\"u}r Informatik, 2025.

\bibitem[Clarkson et~al.(2012)Clarkson, Hazan, and
  Woodruff]{clarkson2012sublinear}
Kenneth~L Clarkson, Elad Hazan, and David~P Woodruff.
\newblock Sublinear optimization for machine learning.
\newblock In \emph{Journal of the ACM (JACM)}, 2012.

\bibitem[Dantzig(1953)]{Dantzig1953}
G.~B. Dantzig.
\newblock \emph{Linear Programming and Extensions}.
\newblock Princeton University Press, Princeton, NJ, 1953.

\bibitem[Freund and Schapire(1999)]{freund1999adaptive}
Yoav Freund and Robert~E Schapire.
\newblock Adaptive game playing using multiplicative weights.
\newblock In \emph{Games and Economic Behavior}, 1999.

\bibitem[Grigoriadis and Khachiyan(1995)]{grigoriadis1995sublinear}
Michael~D Grigoriadis and Leonid~G Khachiyan.
\newblock A sublinear-time randomized approximation algorithm for matrix games.
\newblock In \emph{Operations Research Letters}, 1995.

\bibitem[Hadiji et~al.(2024)Hadiji, Sachs, van Erven, and
  Koolen]{hadiji2024towards}
H{\'e}di Hadiji, Sarah Sachs, Tim van Erven, and Wouter~M Koolen.
\newblock Towards characterizing the first-order query complexity of learning
  (approximate) nash equilibria in zero-sum matrix games.
\newblock In \emph{\cNIPS{2024}}, 2024.

\bibitem[Joulani et~al.(2017)Joulani, Gy{\"o}rgy, and
  Szepesv{\'a}ri]{joulani2017modular}
Pooria Joulani, Andr{\'a}s Gy{\"o}rgy, and Csaba Szepesv{\'a}ri.
\newblock A modular analysis of adaptive (non-)convex optimization: Optimism,
  composite objectives, and variational bounds.
\newblock In \emph{Proceedings of the International Conference on Algorithmic
  Learning Theory}, volume~76 of \emph{Proceedings of Machine Learning
  Research}, pages 681--720. PMLR, 2017.

\bibitem[Karimireddy et~al.(2018)Karimireddy, Stich, and
  Jaggi]{karimireddy2018globallinearconvergencenewtons}
Sai~Praneeth Karimireddy, Sebastian~U. Stich, and Martin Jaggi.
\newblock Global linear convergence of newton's method without strong-convexity
  or lipschitz gradients.
\newblock In \emph{arXiv preprint arXiv:1806.00413}, 2018.

\bibitem[Karmarkar et~al.()Karmarkar, O'Carroll, and
  Sidford]{karmarkar2025solvingzerosumgames}
Ishani Karmarkar, Liam O'Carroll, and Aaron Sidford.
\newblock Solving zero-sum games with fewer matrix-vector products.
\newblock In \emph{\cFOCS{2025}}.

\bibitem[Kornowski and Shamir(2025)]{kornowski2024oracle}
Guy Kornowski and Ohad Shamir.
\newblock The oracle complexity of simplex-based matrix games: Linear
  separability and nash equilibria.
\newblock In \emph{\cCOLT{2025}}, 2025.

\bibitem[Makur(2015)]{anuran2015studyoflocalapproximationsininfotheory}
Anuran Makur.
\newblock A study of local approximations in information theory.
\newblock Master's thesis, Massachusetts Institute of Technology, Cambridge,
  MA, June 2015.
\newblock Submitted to the Department of Electrical Engineering and Computer
  Science.

\bibitem[Martinet(1970)]{martinet1970regularisation}
B.~Martinet.
\newblock R{\'e}gularisation d'in{\'e}quations variationnelles par
  approximations successives.
\newblock \emph{Revue fran{\c c}aise d'informatique et de recherche
  op{\'e}rationnelle, S{\'e}rie rouge}, 4\penalty0 (R3):\penalty0 154--158,
  1970.
\newblock \doi{10.1051/m2an/197004R301541}.

\bibitem[McCulloch and Pitts(1943)]{mcculloch1943logical}
Warren~S McCulloch and Walter Pitts.
\newblock A logical calculus of the ideas immanent in nervous activity.
\newblock In \emph{The bulletin of mathematical biophysics}, 1943.

\bibitem[Nemirovski(2004)]{nem04}
Arkadi Nemirovski.
\newblock Prox-method with rate of convergence o(1/t) for variational
  inequalities with lipschitz continuous monotone operators and smooth
  convex-concave saddle point problems.
\newblock \emph{SIAM Journal on Optimization}, 15\penalty0 (1):\penalty0
  229--251, 2004.

\bibitem[Nemirovskij and Yudin(1983)]{nemirovskij1983problem}
Arkadij~Semenovi{\v{c}} Nemirovskij and David~Borisovich Yudin.
\newblock Problem complexity and method efficiency in optimization.
\newblock In \emph{Wiley-Interscience}, 1983.

\bibitem[Nesterov(2005)]{nesterov2005smooth}
Yu~Nesterov.
\newblock Smooth minimization of non-smooth functions.
\newblock In \emph{Mathematical programming}, 2005.

\bibitem[Nesterov(2007)]{Nesterov2007dualextrapolation}
Yurii Nesterov.
\newblock Dual extrapolation and its applications to solving variational
  inequalities and related problems.
\newblock \emph{Mathematical Programming}, 109\penalty0 (2):\penalty0 319--344,
  Mar 2007.

\bibitem[Rakhlin and Sridharan(2013)]{Rakhlin2013online}
A.~Rakhlin and K.~Sridharan.
\newblock Online learning with predictable sequences.
\newblock In \emph{Proceedings of the 26th Annual Conference on Learning
  Theory}, volume~30 of \emph{Proceedings of Machine Learning Research}, pages
  993--1019. PMLR, 2013.

\bibitem[Rockafellar(1976)]{rockafellar1976monotone}
R.~Tyrrell Rockafellar.
\newblock Monotone operators and the proximal point algorithm.
\newblock \emph{SIAM Journal on Control and Optimization}, 14\penalty0
  (5):\penalty0 877--898, 1976.
\newblock \doi{10.1137/0314056}.

\bibitem[Rosenblatt(1958)]{rosenblatt1958perceptron}
Frank Rosenblatt.
\newblock The perceptron: a probabilistic model for information storage and
  organization in the brain.
\newblock In \emph{Psychological review}, 1958.

\bibitem[Shalev-Shwartz(2012)]{shwartz2012onlinelearning}
Shai Shalev-Shwartz.
\newblock Online learning and online convex optimization.
\newblock \emph{Found. Trends Mach. Learn.}, 4\penalty0 (2):\penalty0 107--194,
  February 2012.
\newblock ISSN 1935-8237.

\bibitem[Shalev-Shwartz and Ben-David(2014)]{shwartz2014understandingML}
Shai Shalev-Shwartz and Shai Ben-David.
\newblock \emph{Understanding Machine Learning: From Theory to Algorithms}.
\newblock Cambridge University Press, 2014.

\bibitem[Soheili and Pen{\~n}a(2012)]{soheili2012smoothperceptron}
Negar Soheili and Javier Pen{\~n}a.
\newblock A smooth perceptron algorithm.
\newblock \emph{SIAM Journal on Optimization}, 22\penalty0 (2):\penalty0
  728--737, 2012.

\bibitem[Steinhardt and Liang(2014)]{steinhardt2014adaptivity}
Jacob Steinhardt and Percy Liang.
\newblock Adaptivity and optimism: An improved exponentiated gradient
  algorithm.
\newblock In Eric~P. Xing and Tony Jebara, editors, \emph{Proceedings of the
  31st International Conference on Machine Learning (ICML)}, volume~32 of
  \emph{Proceedings of Machine Learning Research}, pages 1593--1601. PMLR,
  2014.

\bibitem[Tsybakov(2008)]{tsybakov2008nonparametric}
Alexandre~B Tsybakov.
\newblock Nonparametric estimators.
\newblock In \emph{Introduction to Nonparametric Estimation}. Springer, 2008.

\bibitem[von Neumann(1928)]{vonNeumann1928}
John von Neumann.
\newblock Zur theorie der gesellschaftsspiele.
\newblock \emph{Mathematische Annalen}, 100:\penalty0 295--320, 1928.

\bibitem[Wang et~al.(2023)Wang, Hanashiro, Guha, and
  Abernethy]{wang2023accelerated}
Guanghui Wang, Rafael Hanashiro, Etash~Kumar Guha, and Jacob Abernethy.
\newblock On accelerated perceptrons and beyond.
\newblock In \emph{The Eleventh International Conference on Learning
  Representations}, 2023.

\bibitem[Yu et~al.(2014)Yu, Kilinc-Karzan, and
  Carbonell]{yu2014saddlepointsacceleratedperceptron}
Adams~Wei Yu, Fatma Kilinc-Karzan, and Jaime Carbonell.
\newblock Saddle points and accelerated perceptron algorithms.
\newblock In Eric~P. Xing and Tony Jebara, editors, \emph{Proceedings of the
  31st International Conference on Machine Learning}, volume~32 of
  \emph{Proceedings of Machine Learning Research}, pages 1827--1835, Beijing,
  China, 22--24 Jun 2014. PMLR.

\end{thebibliography}
 %


\newpage

\addtocontents{toc}{\protect\setcounter{tocdepth}{0}} %

\appendix

%
\section{Technical results regarding variational equalities, Bregman divergences, and the truncated domains}

%
%

%
%
%

%
%
%
%
%
%
%
%
%
%
%
%
%
%
%
%
%

%

%
%
%
%
%
%
%

%
%
%
%
 
%

\subsection{Properties of KL divergence and Hellinger distance}

\begin{restatable}[Equation 2.27 of \citep{tsybakov2008nonparametric}, restated]{fact}{hellinger}\label{fact:hellinger-trick} Let $z_1, z_2 \in \Delta^n$. Then $\KL(z_1 || z_2) \geq H(z_1,z_2)$. 
\end{restatable}
\begin{proof} The function $g(s) = -\log(s)$ is convex, hence, for any $s > 0$, 
\begin{align*}
    0 = g(1) \geq g(s) + g'(s) (1-s) = -\log(s) + \frac{-1}{s} (1-s). 
\end{align*}
Thus, $\log(s) \geq 1 - \frac{1}{s}$ and consequently $s^2 \log(s^2) \geq 2s^2 - 2s.$ Taking $r = s^2$, we have that for any $r > 0$,
\begin{align*}
    r\log(r) \geq 2r - 2\sqrt{r}
\end{align*}
and 
\begin{align*}
    r\log(r) - (r-1) \geq r - 2\sqrt{r} + 1 = (1-\sqrt{r})^2. 
\end{align*} 
Taking $r_i = [z_1]_i/[z_2]_i$, we have 
\begin{align*}
    \breg{z_2}{z_1} &= \sum_{i} [z_1]_i \log(r_i) = \sum_i [z_2]_ir_i\log(r_i) = \sum_i [z_2]_i[r_i\log(r_i) - (r_i - 1)] \\&\geq \sum_i [z_2]_i (\sqrt{r}_i - 1)^2 
    = \sum_i (\sqrt{[z_2]_i} \sqrt{r}_i - \sqrt{[z_2]_i})^2 = H^2(z_1,z_2). 
\end{align*} 
where the third equality on the first line uses that $[z_2]_i r_i = [z_1]_i$ and hence $\sum_{i \in [d]} [z_2]_i r_i = 1$.  
\end{proof}

\subsection{Collapsing sums of divergences to the divergence from a single point}
\begin{lemma}\label{lemma:reduce-sum-to-one-point}
Let $\cV = \{v^1, \dots, v^k\} \subset \simplex^d$ be a multiset and $\geomean{\cV} \in \simplex^d$ be defined via 
\begin{align*}
    [\geomean{\cV}]_j \defeq \frac{g_j}{\sum_{j \in[d]} g_j}, \text{ where } g_j \defeq \prod_{i \in [k]} [v^i]_j^{1/k} \text{ for each } j \in [d]. 
\end{align*}
Then for all $w \in \simplex^d$, we have
\begin{align*}
    \sum_{i \in [k]} \inKL{v^i}{w} = k \cdot \inKL{\geomean{\cV}}{w} + C,
\end{align*}
where $C$ is a quantity which does not depend on $w$. Moreover, if for some $\nu > 0$, $\cV \subset \Delta_\nu^d$, then $\geomean{\cV} \in \Delta_\nu^d$.
\end{lemma}
\begin{proof}
Note that using standard properties of the negative entropy function (e.g., Lemmas 10 and 21 in \citet{carmon2025extractingdualsolutionsprimal}), we can equivalently express $\geomean{\cV} = \argmin_{q' \in \simplex^d} \frac{1}{k} \sum_{i \in [k]} \inKL{u^i}{q'}$. The remainder of the proof of the first claim uses identical reasoning to the proof of Lemma 2 in \citep{carmon2025extractingdualsolutionsprimal}. For the additional claim that $\cV \in \Delta^d_\nu$ implies $\geomean{\cV} \in \Delta^d_\nu$, note first that $\cV \in \Delta^d_\nu$ implies 
\begin{align*}
    \nu \leq g_j \leq \frac{1}{k} \sum_{i \in [k]} [v^i]_j, 
\end{align*}
where the left inequality holds because the geometric mean is larger than the minimum, and the right inequality holds because of the AM-GM inequality. Consequently, 
\begin{align*}
    \sum_{j \in [d]} g_j \leq \sum_{j \in [d]} \frac{1}{k} \sum_{i \in [k]} [v^i]_j = \frac{1}{k} \sum_{i \in [k]} \sum_{j \in [d]} [v^i]_j = 1, 
\end{align*}
where the last inequality used that each $v^i \in \Delta^d_\nu \subset \Delta^d$. Thus, $[\geomean{\cV}]_j \geq \nu$. 
\end{proof}

\begin{lemma}\label{lemma:reduce-sum-to-one-point-euclidean}
Let $\cV = \{v^1, \dots, v^k\} \subset \B^d$ be a multiset and $\mean{\cV} \defeq \frac{1}{\abs{\cV}} \sum_{v \in \cV} v$. Then for all $w \in \B^d$, we have
\begin{align*}
    \sum_{i \in [k]} \normInline{v^i - w}_2^2 = |\cV| \cdot \normInline{\mean{\cV} -w}_2^2 + C,
\end{align*}
where $C$ is a quantity which does not depend on $w$.
\end{lemma}
\begin{proof} Note that 
\begin{align*}
     \sum_{i \in [k]} \normInline{v^i - w}_2^2 &=  \sum_{i \in [k]} \normInline{v^i}_2^2 - 2\inangle*{v^i, w} + \normInline{w}_2^2 \\
     &= k \normInline{w}_2^2 - 2 \inangle*{ \sum_{i \in [k]} v^i, w} + \sum_{i \in [k]} \normInline{v^i}^2_2 \\
     &= k \normInline{w}_2^2 - 2n \inangle*{\mean{\cV}, w} + \sum_{i \in [k]} \normInline{v^i}_2^2 \\
    &= k \normInline{w}_2^2 - 2k \inangle*{\mean{\cV}, w} + k\normInline{\mean{\cV}}_2^2 + \sum_{i \in [k]} \normInline{v^i}_2^2 - k\normInline{\mean{\cV}}_2^2 \\
    &= k \normInline{w - \mean{\cV}}_2^2 + \sum_{i \in [k]} \normInline{v^i}_2^2 - n\normInline{\mean{\cV}}_2^2, 
\end{align*}
where the second two terms do not depend on $w$. 
\end{proof}

In the following corollary, we let $(\xset, \yset, \xtrunc, \ytrunc, \cZint, \rx : \xset \to \R, \ry : \yset \to \R, \Gamma_{\dgfsetup})$ be any of the setups defined Definition~\ref{def:matrix-games-setups}. 

\begin{corollary}\label{corr:reduce-sum-to-one-point} Consider a finite nonempty multiset $\cU \subset \cZ_\nu$. There exists a point $\collapsed{\cU}$ such that for all $w \in \cZ_\nu$, $\breg{\cU}{w} = |\cU| \cdot \breg{\collapsed{\cU}}{w} + C$ where $C$ is a quantity that does not depend on $w$. 
\end{corollary}
\begin{proof} This follows from Lemma~\ref{lemma:reduce-sum-to-one-point-euclidean} and Lemma~\ref{lemma:reduce-sum-to-one-point}. 
\end{proof}

%

%

%
%
%

\subsection{Range bound for bisection search analysis}\label{sec:range}

Throughout Appendix~\ref{sec:range}, we let $(\xset, \yset, \xtrunc, \ytrunc, \cZint, \rx : \xset \to \R, \ry : \yset \to \R, \Gamma_{\dgfsetup})$ be any of the setups defined Definition~\ref{def:matrix-games-setups}. 

\begin{lemma}\label{lemma:uncondition-simplex-bound} Let $\{u^1, ..., u^k\} \subset \Delta^d_\nu$ be a multiset. Then, 
\begin{align*}
    \min_{x \in \Delta^d_\nu} \sum_{i \in [k]} \KL(x || u^i) \leq k \cdot \log\paren{\frac{1}{\nu d}}
\end{align*}
\end{lemma}
\begin{proof} Consider $x = [\geomean{\cV}]$ where, we recall from Lemma~\ref{lemma:reduce-sum-to-one-point} that
\begin{align*}
        [\geomean{\cU}]_j \defeq \frac{g_j}{\sum_{j \in[d]} g_j}, \text{ where } g_j \defeq \prod_{i \in [k]} [u^i]_j^{1/k} \text{ for each } j \in [d]. 
\end{align*}
For notational convenience, let $G \defeq \sum_{j \in[d]} g_j$. Then, 
\begin{align*}
    \sum_{i \in [k]} \KL(x || u) &= \sum_{i \in [k]} \sum_{j \in [d]} [x]_j \log\paren{\frac{[x]_j}{[u^i]_j}} = \sum_{j \in [d]} [x]_j \paren{k \log([x]_j) - \sum_{i \in [k]} \log([u^i]_j)} \\
    &= \sum_{j \in [d]} [x]_j \paren{k \log([x]_j) - k \log([g]_j)} \\
    &= \sum_{j \in [d]} [x]_j k \paren{\log\paren{\frac{[g]_j}{G}} - \log([g]_j)} \\
    &= \sum_{j \in [d]} [x]_j k \log\paren{\frac{1}{G}} \\
    &= k \log\paren{\frac{1}{G}} 
    \leq k \log\paren{\frac{1}{\nu d}}
\end{align*} 
where the second line used the fact that 
\begin{align*}
    g_j \defeq \prod_{i \in [k]} [u^i]_j^{1/k}~~\text{ if and only if }~~\log(g_j) = \frac{1}{k} \sum_{i \in [k]} \log([u^i]_j), 
\end{align*}
the second-to-last line used that $x \in \Delta^d$, and the last line used that $g_j \geq \nu$ (since the geometric mean is lower bounded by the minimum) implies that $G \geq \nu d$. 
\end{proof}

\begin{lemma}\label{lemma:unconditional-euclidean-bound} Let $\{u^1, ..., u^k\} \subset \B^d_\nu$ be a multiset. Then, 
\begin{align*}
    \min_{x \in \B^n_{\nu}} \sum_{i \in [k]} \normInline{x - u^i}_2^2 \leq k. 
\end{align*}
\end{lemma}
\begin{proof} The minimizer is achieved by the mean, in which case the minimum value is given by 
\begin{align*}
    \sum_{i \in [k]} \normInline{u^i - \mean{\cU}}_2^2 \leq \sum_{i \in [k]} (\normInline{u^i}_2 + \normInline{\mean{\cU}}_2)^2 \leq \sum_{i \in [k]} 4 = 4k, 
\end{align*}
where the first inequality uses triangle inequality, and the second uses that $u^i, \mean{\cU} \in \B^d$. 
\end{proof} 
%


\end{document}